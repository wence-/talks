% -*- TeX-engine: luatex -*-
\documentclass[presentation,aspectratio=43, 10pt]{beamer}
\usepackage{pifont}
\newcommand{\cmark}{\ding{51}}
\newcommand{\xmark}{\ding{55}}

\usepackage{booktabs}
\usepackage{transparent}
\titlegraphic{\hfill{\transparent{0.2}\includegraphics[height=1.25cm]{durham-logo}}}
\usepackage{appendixnumberbeamer}
\usepackage{amsmath}
\usepackage{amssymb}
\usepackage{amsfonts}
\usepackage{bm}
\usepackage{mathtools}
\usepackage{hyperref}
\usepackage{xspace}
\newcommand{\arxivlink}[2]{{\texttt{arXiv:\,\href{https://arxiv.org/abs/#1}{#1\,[#2]}}}}

\newcommand{\honev}{\ensuremath{{H}^1(\Omega; \mathbb{R}^d)}\xspace}
\newcommand{\ltwov}{\ensuremath{{L}^2(\Omega; \mathbb{R}^d)}\xspace}
\newcommand{\ltwo}{\ensuremath{{L}^2(\Omega)}\xspace}
\newcommand{\inner}[1]{\left\langle #1 \right \rangle}
\newcommand{\dx}{\,\text{d}x}
\newcommand{\ds}{\,\text{d}s}

\usepackage{emoji}
\usepackage{minted}
\usepackage[url=false,
doi=true,
isbn=false,
style=authoryear,
maxnames=5,
giveninits=true,
uniquename=init,
backend=biber]{biblatex}
\renewcommand{\bibfont}{\fontsize{7}{7}\selectfont}
\addbibresource{references.bib}

\setlength{\bibitemsep}{1ex}
\setlength{\fboxsep}{1pt}

\renewbibmacro{in:}{}
\DeclareFieldFormat[article]{volume}{\textbf{#1}}
\DeclareFieldFormat{doi}{%
  doi\addcolon%
  {\scriptsize\ifhyperref{\href{http://dx.doi.org/#1}{\nolinkurl{#1}}}
    {\nolinkurl{#1}}}}
\AtEveryBibitem{%
\clearfield{pages}%
\clearfield{issue}%
\clearfield{number}%
}

\DeclareMathOperator{\grad}{grad}
\let\div\relax
\DeclareMathOperator{\div}{div}
\DeclareMathOperator{\curl}{curl}
\DeclareMathOperator{\range}{range}
\DeclareMathOperator{\sym}{sym}
\usetheme{metropolis}
\setbeamertemplate{title graphic}{
  \vbox to 0pt {
    \vspace*{1em}
    \inserttitlegraphic%
  }%
  \nointerlineskip%
}
\metroset{background=light,progressbar=frametitle,numbering=counter,block=fill}

% https://www.dur.ac.uk/marketingandcommunications/marketing/branding/colourpalette/
% Most of these are indistinguishable to those suffering colour blindness
\definecolor{purple}{HTML}{68246D}
\definecolor{blue}{HTML}{002A41}
\definecolor{red}{HTML}{BE1E2D}
\definecolor{cyan}{HTML}{00AEEF}
\definecolor{yellow}{HTML}{FFD53A}

\newenvironment{variableblock}[3]
{\setbeamercolor{block body}{#2}
\setbeamercolor{block title}{#3}
\begin{block}{#1}}%
{\end{block}}
  
\newenvironment{challenge}[1]%
{\begin{variableblock}{#1}{bg=red!20,fg=black}{bg=red,fg=white}}%
{\end{variableblock}}

\newenvironment{answer}[1]%
{\begin{variableblock}{#1}{bg=cyan!20,fg=black}{bg=cyan,fg=white}}%
{\end{variableblock}}

\renewenvironment{exampleblock}[1]%
{\begin{variableblock}{#1}{bg=yellow!20,fg=black}{bg=yellow,fg=white}}%
{\end{variableblock}}

\setbeamercolor{normal text}{
  fg=black,
  bg=white
}
\setbeamercolor{alerted text}{
  fg=red
}
\setbeamercolor{example text}{
  fg=blue
}

\setbeamercolor{palette primary}{%
  use=normal text,
  fg=normal text.bg,
  bg=purple,
}

\usetheme{metropolis}

\author{Lawrence Mitchell\inst{1,*}}
\institute{
  \inst{1}\textcolor{black!20}{Department of Computer Science, Durham University}\\
  \inst{*}\texttt{lawrence@wence.uk}}

\title{Composition and DSLs for finite element computation: pros and cons}

\usepackage{tikz}
\usetikzlibrary{trees,calc,positioning}
\usetikzlibrary{shapes, shapes.geometric}
\usetikzlibrary{arrows,chains,positioning,fit,backgrounds,calc,shapes,
  shadows,scopes,decorations.markings,plotmarks}

\newcommand*{\tettextsize}{\footnotesize}
\tikzstyle{line} = [draw, -, thick]
\tikzstyle{nodraw} = [draw, fill, circle, minimum width=0pt, inner sep=0pt]
\tikzstyle{sieve} = [line, circle, font=\tettextsize, inner sep=0pt,
  minimum size=12pt]

\tikzstyle{cell} = [sieve, fill=blue!60]
\tikzstyle{facet} = [sieve, fill=green!35]
\tikzstyle{edge} = [sieve, fill=red!35]
\tikzstyle{vertex} = [sieve, fill=blue!35]

% https://tex.stackexchange.com/questions/27171/padded-boundary-of-convex-hull
\newcommand{\convexpath}[2]{
  [
  create hullcoords/.code={
    \global\edef\namelist{#1}
    \foreach [count=\counter] \nodename in \namelist {
      \global\edef\numberofnodes{\counter}
      \coordinate (hullcoord\counter) at (\nodename);
    }
    \coordinate (hullcoord0) at (hullcoord\numberofnodes);
    \pgfmathtruncatemacro\lastnumber{\numberofnodes+1}
    \coordinate (hullcoord\lastnumber) at (hullcoord1);
  },
  create hullcoords
  ]
  ($(hullcoord1)!#2!-90:(hullcoord0)$)
  \foreach [
  evaluate=\currentnode as \previousnode using \currentnode-1,
  evaluate=\currentnode as \nextnode using \currentnode+1
  ] \currentnode in {1,...,\numberofnodes} {
    let \p1 = ($(hullcoord\currentnode) - (hullcoord\previousnode)$),
    \n1 = {atan2(\y1,\x1) + 90},
    \p2 = ($(hullcoord\nextnode) - (hullcoord\currentnode)$),
    \n2 = {atan2(\y2,\x2) + 90},
    \n{delta} = {Mod(\n2-\n1,360) - 360}
    in
    {arc [start angle=\n1, delta angle=\n{delta}, radius=#2]}
    -- ($(hullcoord\nextnode)!#2!-90:(hullcoord\currentnode)$)
  }
}

\graphicspath{{./\jobname.figures/}{../pictures/}}

\begin{document}
\maketitle

\begin{frame}
  \frametitle{Setting}
  \begin{columns}
    \begin{column}{0.8\textwidth}
      \begin{quote}
        Firedrake \url{www.firedrakeproject.org} {\normalfont
          [\ldots]} is an automated system for the solution of
        partial differential equations using the finite element
        method.
      \end{quote}
    \end{column}
    \begin{column}{0.2\textwidth}
      \includegraphics[width=0.8\textwidth]{firedrake}
    \end{column}
  \end{columns}
  \begin{itemize}
  \item \alert{Uses \emph{embedded} domain specific language, UFL
    \parencite{Alnaes:2014} from the FEniCS project}
  \item \alert{Form compiler and ``symbolic'' element library}
  \item PETSc for meshes and (algebraic) solvers \nocite{Balay:2016}
  \end{itemize}

  {\raggedleft
    \scriptsize \textcite{Rathgeber:2016} \arxivlink{1501.01809}{cs.MS}\par}
\end{frame}

\begin{frame}
  \frametitle{Why DSLs at all?}

  \begin{block}{Contention}
    DSLs (in various forms) pervade much of scientific software
    because the ergonomics of metaprogramming in (almost?) all
    languages is \emph{terrible}.

    Especially if you want arbitrary composition and want to avoid
    accidental performance pitfalls.
  \end{block}
  \pause
  \begin{answer}{Corollary}
    Composition in libraries is \emph{algebraic} not symbolic

    \pause

    Or you design representations of symbolic structure and then have
    a DSL.
  \end{answer}
\end{frame}

\begin{frame}[fragile]
  \frametitle{DSLs in Firedrake: UFL}
  \begin{columns}
    \begin{column}{0.5\framewidth}
      {\footnotesize
        Find $(u, p, T) \in V\times W\times Q$ s.t.
        \begin{align*}
          \int\!\nabla u \cdot \nabla v + (u \cdot \nabla u) \cdot v \\
          - p\nabla\cdot v + \frac{\text{Ra}}{\text{Pr}} Tg \hat{z} \cdot v\,\text{d}x &= 0 \\
          \int\!\nabla\cdot u q\,\text{d}x&= 0\\
          \int\! (u\cdot \nabla T) S + \text{Pr}^{-1} \nabla T \cdot \nabla
          S\,\text{d}x &= 0\\
          \quad \forall\, (v,q,T) \in V\times W \times Q
        \end{align*}
        }
    \end{column}
      \begin{column}{0.5\framewidth}
\begin{minted}[fontsize=\tiny]{python}
from firedrake import *
mesh = Mesh(...)
V = VectorFunctionSpace(mesh, "CG", 2)
W = FunctionSpace(mesh, "CG", 1)
Q = FunctionSpace(mesh, "CG", 1)
Z = V * W * Q
Ra = Constant(...)
Pr = Constant(...)
upT = Function(Z)
u, p, T = split(upT)
v, q, S = TestFunctions(Z)
bcs = [...]

F = (inner(grad(u), grad(v))
     + inner(dot(grad(u), u), v)
     - inner(p, div(v))
     + (Ra/Pr)*inner(T*g, v)
     + inner(div(u), q)
     + inner(dot(grad(T), u), S)
     + (1/Pr)*inner(grad(T), grad(S)))*dx

solve(F == 0, upT, bcs=bcs)
\end{minted}
      \end{column}
  \end{columns}
\end{frame}

\begin{frame}
  \frametitle{Mostly good!}
  \begin{itemize}
  \item UFL is now >15 years old
  \item Surface has mostly stood test of time
  \item Really it's a symbolic algebra package with some FE-specific
    terminal nodes and builtin FE-relevant DAG rewrites
  \end{itemize}
  \begin{challenge}{Problems}
    \emph{Implementation} is showing its age.

    Difficult to get recognition for reimplementation of the same API
  \end{challenge}
\end{frame}

\begin{frame}
  \frametitle{Can I write down \emph{my} problem?}
  \begin{itemize}
  \item[\emoji{thumbsup}] If you can express your problem in UFL, great!
  \item[\emoji{thumbsdown}] If you can't, it's a mathematical software
    research problem to figure out what to do, rather than a SMOP
  \end{itemize}
  \pause
  \begin{answer}{Problem}
    Graceful degradation with a DSL is \emph{hard}

    \pause
    How do you enable it without forcing model developer to abandon
    all the nice stuff?

    \pause
    Ongoing work in UFL/Firedrake with \texttt{ExternalOperator}
    (Nacime Bouziani, David Ham, India Marsden, Reuben Nixon-Hill)
  \end{answer}
\end{frame}

\begin{frame}[fragile]
  \frametitle{DSLs in Firedrake: GEM}
  \begin{challenge}{Motivation}
    Algorithmically optimal FE basis evaluation (exploiting sum
    factorisation and similar)

    Goal: transform coefficients at node points to coefficients at
    quadrature points
    \begin{equation*}
      \tilde{u}_q = B u
    \end{equation*}
  \end{challenge}

  \begin{answer}{Problem}
    \emoji{thumbsdown} If element library gives you $B$ as a dense
    array, can't deliver optimal contraction

    \emoji{thumbsup} Library solution: provide the action $Bu$ as a
    primitive

    \emoji{question} what about composition?
  \end{answer}
\end{frame}
\begin{frame}[fragile]
  \frametitle{FInAT: An algebra for basis evaluation}
  \footnotesize
  \begin{itemize}
  \item Literal array
    \begin{equation*}
      \Psi_i(\xi_q) \mapsto \bm{\Psi}_{iq}
    \end{equation*}
  \item Tensor-valued, components are Kronecker $\delta$s
    \begin{equation*}
      \Psi_{(\alpha, \nu)\kappa}(\xi_q) \mapsto \Psi_{\alpha}^*(\xi_q){\prod_{i=1}^n \delta_{\nu_i \kappa_i}}
    \end{equation*}
  \item Underintegration, $\{\xi_q\}$ matches nodes
    \begin{equation*}
      \Psi_i(\xi_q) \mapsto \delta_{iq}
    \end{equation*}
  \item Enriched (direct sum)
    \begin{equation*}
      {(\Phi \oplus \Psi)}_i(\xi_q) \mapsto
      \begin{cases}
        \Phi_i(\xi_q) &\text{$i < N_\Phi$}\\
        \Psi_{i-N_\Phi}(\xi_q) &\text{otherwise}
      \end{cases}
    \end{equation*}
  \item Tensor product (with tensor-product quad rule)
    \begin{equation*}
      \Psi_{(i_1, \dots, i_n)}(\xi_{(q_1, \dots, q_n)}) \mapsto \prod_{j=1}^{n} \psi_{i_j}(\xi_{q_j})
    \end{equation*}
  \item Rank-promotion ($\mathbf{H}(\text{div}/\text{curl})$)
    \begin{equation*}
      \text{\texttt{HDiv}}(\Psi_i(\xi_q)) \mapsto
      \begin{cases}
        [-\Psi_i(\xi_q), 0]\\
        [0, \Psi_i(\xi_q)]
      \end{cases}
    \end{equation*}
  \end{itemize}
\end{frame}

\begin{frame}
  \frametitle{A compiler can now DTRT}
  \begin{itemize}
  \item With structure exposed, all new elements get fast basis
    evaluation
  \item \dots Once the optimisation pass has been written in the
    compiler \parencite{Homolya:2018}
  \item Language is essentially Einstein notation, plus
    structure-preserving concatenation/reshape \parencite{Homolya:2017a}
  \end{itemize}

  \begin{answer}{Algebraic approach?}
    I think you can do this composition algebraically (c.f.~work by
    Toby Isaac in PETSc \url{https://gitlab.com/tisaac/lans-feec})

    Library is restricted to a static schedule, so might not be
    operation minimal if extents of the tensor product (say) are not
    all the same.
  \end{answer}
\end{frame}

\begin{frame}
  \frametitle{Also for interpolation}
  \begin{itemize}
  \item Other interface library should expose is for interpolation
  \item Given some $f$, want to represent it in the finite element
    space by acting each dual functional on $f$
  \item Same idea: FInaT provides compositional algebra
  \item[\emoji{thumbsdown}] I haven't worked out the composition for
    tensor-product + enriched together yet.
  \item[\emoji{thumbsdown}] will need new optimisation pass in compiler
  \end{itemize}
\end{frame}
\begin{frame}
  \frametitle{What's the catch?}
  \begin{challenge}{How to lower to executable kernels?}
    TSFC just picks an operation minimal loop schedule and emits C
    code (via \url{https://github.com/inducer/loopy})

    \emoji{thumbsdown} bad solution for matrix-matrix as soon as
    extents are bigger than \sim 8

    \emoji{thumbsdown} implementation is (close to) operation-minimal,
    but not low-level optimised
  \end{challenge}
  \begin{answer}{What to do?}
    Scheduling and optimisation at loop level is something for the
    downstream compiler to handle

    I think Kaushik will talk about some of this in loopy? (MS59, 4:50
    PST today)

    \emoji{question} Can one avoid effort by devolving to MLIR linalg
    dialect?
  \end{answer}
\end{frame}

\begin{frame}
  \frametitle{Where are we now?}
  \begin{answer}{Pros}
    \begin{itemize}
    \item If your problem fits the abstraction, most things are golden
    \item It works pretty well!
    \item There's lots of fun computational science to do along the way
    \end{itemize}
  \end{answer}
  \begin{challenge}{Cons}
    \begin{itemize}
    \item I would like if we had to maintain fewer bits of code
    \item \dots especially taking advantage of other people's
      performance optimisations
    \item Firedrake doesn't offer a library interface: so
      non-expressible features are hard to include
    \end{itemize}
  \end{challenge}
\end{frame}

\begin{frame}
  \frametitle{Open questions/conclusions}
  \begin{itemize}
  \item DSLs from scratch are tunable to the domain, is there a common
    target to lower to? Or do we have to replicate effort each time?
  \item Do we need structure-preservation and compilation in this way,
    or is algebraic composition sufficient?
  \item How do we make composition of generated and hand-written code
    ergonomic?
  \item Are all of these problems because Python?
  \end{itemize}

  \begin{center}
    \large Thanks!
  \end{center}
\end{frame}

\appendix
\begin{frame}[allowframebreaks]
  \frametitle{References}
  \printbibliography[heading=none]
\end{frame}

\end{document}
% Name: Lawrence Mitchell
% Affiliation: Durham University
% Address: Department of Computer Science, Upper Mountjoy, Durham, DH1 3LE
% Phone: +44 131 334 4314
% email: lawrence.mitchell@durham.ac.uk
% title: 

% Abstract:

% The Firedrake system uses a multi-layered set of DSLs and
% domain-specific optimising compilers to generate (hopefully) high
% performance code for finite element assembly. The goal is to deliver
% expert-level performance to all users, who may not be experts in the
% necessary low level optimisations, and data structure design. In this
% talk I will discuss some of the design choices, optimisations and
% algorithmic transformations that they enable, and pros and cons of
% this generative approach.
