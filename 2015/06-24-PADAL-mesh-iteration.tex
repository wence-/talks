\documentclass[presentation]{beamer}
\usepackage[utf8]{inputenc}
\usepackage[T1]{fontenc}
\usepackage{fixltx2e}
\usepackage{graphicx}
\usepackage{longtable}
\usepackage{float}
\usepackage{wrapfig}
\usepackage{rotating}
\usepackage[normalem]{ulem}
\usepackage{amsmath}
\usepackage{textcomp}
\usepackage{marvosym}
\usepackage[integrals]{wasysym}
\usepackage{amssymb}
\usepackage{hyperref}
\tolerance=1000
\usepackage{minted}
\usepackage{amsmath}
\usepackage{tikz}
\usepgflibrary{shapes.geometric}
\usetikzlibrary{calc}
\usetikzlibrary{positioning}
\usetikzlibrary{intersections,decorations.pathreplacing,shapes,arrows}
\usetikzlibrary{plotmarks}
\usepackage{amsmath,amssymb}
\usepackage{mathtools}
\DeclareMathOperator{\tr}{tr}
\usepackage{pgfplots}
\institute{Departments of Computing and Mathematics, Imperial College London}
\renewcommand{\vec}{\mathbf}
\usetheme{IC}
\author{Lawrence Mitchell}
\date{24 June 2015}
\title{Abstracting mesh iteration?}
\begin{document}

\maketitle


\begin{frame}
  \frametitle{Motivation}
  \begin{itemize}
\item Numerical solution of PDEs often requires iteration over
    meshes
  \item Can this be expressed without application code ``knowing'' about
    the mesh structure?
  \end{itemize}
\end{frame}

\begin{frame}[fragile]
  \frametitle{Commonalities}
  \begin{itemize}
  \item Restricted to finite element methods, all evaluation
    looks similar
  \end{itemize}
\begin{block}<1 | only@1>{Residual evaluation}
\begin{minted}[fontsize=\scriptsize]{python}
# u the input fields (e.g. current guess)
def form_residual(u):
    u_l <- u # global to ghosted
    for each element in mesh:
        u_e <- u_l[element] # gather through element map
        for each qp in element:
            basis_fns <- eval_basis_funs(qp)
            J <- compute_geometry(element, qp)
            f_qp <- user_evaluation(qp, basis_fns, u_e)
            # insert into element residual
            f_e <- transform_to_physical(f_qp, J)
        f_l <- f_e # scatter through element map
    f <- f_l # ghosted to global
\end{minted}
\end{block}
\begin{block}<2| only@2>{Residual evaluation}
\begin{minted}[fontsize=\scriptsize]{python}
            f_qp <- user_evaluation(qp, basis_fns, u_e)
\end{minted}
\end{block}
\begin{itemize}[<2 | only@2>]
\item Variability in innermost loop
\item but relatively simple
\end{itemize}
\end{frame}

\begin{frame}[t]
  \frametitle{Who controls what?}
  \begin{block}<1-2 | only@1-2>{Type I: traditional FE libraries (e.g. deal.II, DUNE)}
    \begin{itemize}
    \item Library provides iteration constructs
    \item user provides \emph{full element loop}
    \end{itemize}
  \end{block}
  \begin{itemize}[<1-2 | only@2>]
  \item Full flexibility, can fuse across elements, cancel geometric
    transformations, etc...
  \item Can apply vectorization (and compiler may do good job,
    everything in one compilation unit)
  \item but, code commits to an implementation
  \end{itemize}
  \begin{block}<only@3-4>{Type II: FEniCS-like}
    \begin{itemize}
    \item Library iterates over mesh entities
    \item user provides \emph{element kernel}
    \end{itemize}
  \end{block}
  \begin{itemize}[<only@4>]
  \item code doesn't commit to an inter-element implementation
  \item can vectorize inside elements, difficult across elements
  \item but, (if using code-gen) ``nastiness'' in real codes difficult to
    handle
  \end{itemize}
  \begin{block}<only@5-6>{Type III: MOOSE-like}
    \begin{itemize}
    \item Library iterates over elements and quad points
    \item user provides \emph{pointwise function}
    \end{itemize}
  \end{block}
  \begin{itemize}[<only@6>]
  \item code doesn't commit to iteration at all
  \item but, can't exploit wider structure without provision of
    additional info
  \item unlikely to vectorize across quad points
  \end{itemize}
\end{frame}

\begin{frame}
  \frametitle{Issues}
  \begin{itemize}
  \item Optimisation probably won’t cross library/user-code divide
  \item Low order, inlining and fusing/vectorising probably important
  \item Suggests JIT-compilation of combination of library iteration
    and user code
  \end{itemize}
\end{frame}

\begin{frame}
  \frametitle{Questions/Thoughts}
  \begin{itemize}
  \item I think library, not user, should own (at least) the element loop
    (and probably quad loop)
  \item but then how do we exploit structure in the problem to get
    low-complexity algorithms?
  \item Is JIT-compilation the way to go? Can it be made easy?
  \item Vectorisation is still too hard, what can we do about it?
  \end{itemize}
\end{frame}
\end{document}
