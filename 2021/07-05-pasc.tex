\title{Flexible computational abstractions in the design and
  implementation of multiphysics preconditioners}


\begin{abstract}
  Small block overlapping, and non-overlapping, Schwarz methods are
  theoretically highly attractive as multigrid smoothers for a wide
  variety of problems that are not amenable to point relaxation
  methods. Examples include monolithic Vanka smoothers for Stokes,
  overlapping vertex-patch decompositions for $H(\text{div})$ and
  $H(\text{curl})$ problems, and line smoothers for elliptic problems
  in thin domains.

  While it is possible to manually program these different schemes,
  their use in general purpose libraries has been held back by a lack
  of generic, composable interfaces. I present a new approach to the
  specification and development such additive Schwarz methods in PETSc
  that cleanly separates the topological space decomposition from the
  discretisation and assembly of the equations. This separation
  enables both linear and nonlinear smoothers to be supported through
  the same callback interface.

  The preconditioner is flexible enough to support overlapping and
  non-overlapping additive Schwarz methods, and can be used to
  formulate line, and plane smoothers, Vanka iterations, and others. I
  will illustrate the effectiveness with examples using the Firedrake
  finite element library.
\end{abstract}

% Motivation:
% - Block preconditioning, what to do on blocks, how to deliver
%   auxiliary operators? Or what if the preconditioner is not just the
%   inverse of a single operator?
% - Multigrid with "fancy" smoothers: motivation augmented Lagrangian
%   methods for problems with constraints
%
%   Examples:
%   - Stokes: "Easy"
%   - Stokes: BFBT?; N-S: PCD
%   - Stationary problems: N-S (schur complement + AL)
%   - MHD (probably too much)
%   - Implicit timestepping (Colin's stuff for shallow-water)
%
% - Software
% - PETSc composition philosophy
% - callbacks to discretisation library for all the operators.
% - PCPatch (for the smoothers)

% Things that don't work
% DD (like ffddm)
% PCPatch doesn't work for high order (designed around doing dense
% inverse on assembly patch operators).
% Batching/small solvers
% Interface is necessarily (?) not the same as the full Jacobian
% construction. Design guided by performance?

