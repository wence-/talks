% -*- TeX-engine: luatex -*-
\documentclass[presentation,aspectratio=43, 10pt]{beamer}
\usepackage{booktabs}
\titlegraphic{\hfill\includegraphics[height=1.25cm]{durham-logo}}
\def\xunderbrace#1_#2{{\underbrace{#1}_{\mathclap{#2}}}}
\def\xoverbrace#1^#2{{\overbrace{#1}^{\mathclap{#2}}}}
\def\xunderarrow#1_#2{{\underset{\overset{\uparrow}{\mathclap{#2}}}{#1}}}
\def\xoverarrow#1^#2{{\overset{\underset{\downarrow}{\mathclap{#2}}}{#1}}}

\usepackage{appendixnumberbeamer}
\usepackage{amsmath}
\usepackage{amssymb}
\usepackage{mathtools}
\usepackage{hyperref}
\usepackage{xspace}
\newcommand{\arxivlink}[2]{{\texttt{arXiv:\,\href{https://arxiv.org/abs/#1}{#1\,[#2]}}}}

\newcommand{\eps}[1]{\ensuremath{\varepsilon{(#1)}}}
\newcommand{\honev}{\ensuremath{{H}^1(\Omega; \mathbb{R}^d)}\xspace}
\newcommand{\ltwov}{\ensuremath{{L}^2(\Omega; \mathbb{R}^d)}\xspace}
\newcommand{\ltwo}{\ensuremath{{L}^2(\Omega)}\xspace}
\newcommand{\inner}[1]{\left\langle #1 \right \rangle}
\newcommand{\dx}{\,\text{d}x}
\newcommand{\ds}{\,\text{d}s}


\usepackage{minted}
\usepackage[url=false,
doi=true,
isbn=false,
style=authoryear,
maxnames=5,
giveninits=true,
uniquename=init,
backend=biber]{biblatex}
\renewcommand{\bibfont}{\fontsize{7}{7}\selectfont}
\addbibresource{references.bib}

\setlength{\bibitemsep}{1ex}
\setlength{\fboxsep}{1pt}

\renewbibmacro{in:}{}
\DeclareFieldFormat[article]{volume}{\textbf{#1}}
\DeclareFieldFormat{doi}{%
  doi\addcolon%
  {\scriptsize\ifhyperref{\href{http://dx.doi.org/#1}{\nolinkurl{#1}}}
    {\nolinkurl{#1}}}}
\AtEveryBibitem{%
\clearfield{pages}%
\clearfield{issue}%
\clearfield{number}%
}

\DeclareMathOperator{\grad}{grad}
\let\div\relax
\DeclareMathOperator{\div}{div}
\DeclareMathOperator{\curl}{curl}
\DeclareMathOperator{\range}{range}
\DeclareMathOperator{\sym}{sym}
\newcommand{\advect}[2]{\ensuremath{(#2 \cdot \nabla) #1}}
\newcommand{\kerdiv}{\ker\div}
\newcommand{\kercurl}{\ker\curl}
\let\Re\relax
\DeclareMathOperator{\Re}{Re}

\usetheme{metropolis}
\setbeamertemplate{title graphic}{
  \vbox to 0pt {
    \vspace*{1em}
    \inserttitlegraphic%
  }%
  \nointerlineskip%
}
\metroset{background=light,progressbar=frametitle,numbering=counter,block=fill}

% https://www.dur.ac.uk/marketingandcommunications/marketing/branding/colourpalette/
% Most of these are indistinguishable to those suffering colour blindness
\definecolor{purple}{HTML}{68246D}
\definecolor{blue}{HTML}{002A41}
\definecolor{red}{HTML}{BE1E2D}
\definecolor{cyan}{HTML}{00AEEF}
\definecolor{yellow}{HTML}{FFD53A}

\newenvironment{variableblock}[3]
{\setbeamercolor{block body}{#2}
\setbeamercolor{block title}{#3}
\begin{block}{#1}}%
{\end{block}}
  
\newenvironment{challenge}[1]%
{\begin{variableblock}{#1}{bg=red!20,fg=black}{bg=red,fg=white}}%
{\end{variableblock}}

\newenvironment{answer}[1]%
{\begin{variableblock}{#1}{bg=cyan!20,fg=black}{bg=cyan,fg=white}}%
{\end{variableblock}}

\renewenvironment{exampleblock}[1]%
{\begin{variableblock}{#1}{bg=yellow!20,fg=black}{bg=yellow,fg=white}}%
{\end{variableblock}}

\setbeamercolor{normal text}{
  fg=black,
  bg=white
}
\setbeamercolor{alerted text}{
  fg=red
}
\setbeamercolor{example text}{
  fg=blue
}

\setbeamercolor{palette primary}{%
  use=normal text,
  fg=normal text.bg,
  bg=purple,
}

\usetheme{metropolis}

\author{Lawrence Mitchell\inst{1,*}}
\institute{
  \inst{1}Department of Computer Science, Durham University\\
  \inst{*}\texttt{lawrence.mitchell@durham.ac.uk}}

\title{Augmented Lagrangian preconditioning for fluids: theory and practice}

\usepackage{tikz}
\usetikzlibrary{trees,calc,positioning}
\usetikzlibrary{shapes, shapes.geometric}
\usetikzlibrary{arrows,chains,positioning,fit,backgrounds,calc,shapes,
  shadows,scopes,decorations.markings,plotmarks}

\newcommand*{\tettextsize}{\footnotesize}
\tikzstyle{line} = [draw, -, thick]
\tikzstyle{nodraw} = [draw, fill, circle, minimum width=0pt, inner sep=0pt]
\tikzstyle{sieve} = [line, circle, font=\tettextsize, inner sep=0pt,
  minimum size=12pt]

\tikzstyle{cell} = [sieve, fill=blue!60]
\tikzstyle{facet} = [sieve, fill=green!35]
\tikzstyle{edge} = [sieve, fill=red!35]
\tikzstyle{vertex} = [sieve, fill=blue!35]

% https://tex.stackexchange.com/questions/27171/padded-boundary-of-convex-hull
\newcommand{\convexpath}[2]{
  [
  create hullcoords/.code={
    \global\edef\namelist{#1}
    \foreach [count=\counter] \nodename in \namelist {
      \global\edef\numberofnodes{\counter}
      \coordinate (hullcoord\counter) at (\nodename);
    }
    \coordinate (hullcoord0) at (hullcoord\numberofnodes);
    \pgfmathtruncatemacro\lastnumber{\numberofnodes+1}
    \coordinate (hullcoord\lastnumber) at (hullcoord1);
  },
  create hullcoords
  ]
  ($(hullcoord1)!#2!-90:(hullcoord0)$)
  \foreach [
  evaluate=\currentnode as \previousnode using \currentnode-1,
  evaluate=\currentnode as \nextnode using \currentnode+1
  ] \currentnode in {1,...,\numberofnodes} {
    let \p1 = ($(hullcoord\currentnode) - (hullcoord\previousnode)$),
    \n1 = {atan2(\y1,\x1) + 90},
    \p2 = ($(hullcoord\nextnode) - (hullcoord\currentnode)$),
    \n2 = {atan2(\y2,\x2) + 90},
    \n{delta} = {Mod(\n2-\n1,360) - 360}
    in
    {arc [start angle=\n1, delta angle=\n{delta}, radius=#2]}
    -- ($(hullcoord\nextnode)!#2!-90:(hullcoord\currentnode)$)
  }
}

\graphicspath{{./\jobname.figures/}{../pictures/}}

\usepackage{emoji}

\begin{document}
\maketitle

\begin{abstract}
  Augmented Lagrangian preconditioning for fluids problems was
  introduced by Benzi and Olshanskii in 2006. The approach offers excellent,
  parameter-robust, control of the Schur complement approximation. The
  drawback is that the preconditioning scheme for the top-left block
  is significantly more complicated, and at the time, an extension to
  three dimensions was not known.

  In recent years, there has been significant progress in this area,
  guided by a deeper understanding of how to construct appropriate
  preconditioners, and software advances that ease implementation.

  The core idea in the design of effective preconditioners for the top
  left block is characterising a basis for the kernel of the augmented
  Lagrangian term. Structure preserving discretisations offer a
  systematic way to attack this problem when the augmentation is a
  differential operator. The resulting robust multigrid methods
  require small block overlapping additive Schwarz smoothers.

  In this talk I will discuss the general augmented Lagrangian
  approach, discuss a flexible preconditioning package that provides
  fast implementation of optimal methods, and illustrate with some
  examples covering stationary Navier--Stokes and MHD, along with
  time-dependent problems.
\end{abstract}


\begin{frame}
  \frametitle{A first problem}
  \begin{block}{Stationary, Newtonian, incompressible Navier--Stokes}
    Find $(u, p) \in \honev \times \ltwo$ such that
    \begin{alignat*}{2}
      -  \nu \nabla^2 u + \advect{u}{u} + \nabla p &= f \quad && \text{ in } \Omega, \\
      \nabla \cdot u &= 0 \quad && \text{ in } \Omega,
    \end{alignat*}
    with suitable boundary conditions, and $\nu$ the kinematic viscosity.
  \end{block}
  \begin{exampleblock}{Multiple solutions}<2->
    For $\nu \to 0$, these equations admit multiple solutions
  \end{exampleblock}
  \begin{challenge}{Motivating question}<3->
    What are the solution(s) as $\nu$ varies?
  \end{challenge}
\end{frame}

\begin{frame}
  \frametitle{Block preconditioning}
  \begin{exampleblock}{Newton linearisation}
    \begin{equation*}
      P := \begin{pmatrix}
        A & B^T \\
        B & 0
      \end{pmatrix}
      \begin{pmatrix}
        \delta u \\ \delta p
      \end{pmatrix}
      =
      \begin{pmatrix}
        b \\ 0
      \end{pmatrix}.
    \end{equation*}
  \end{exampleblock}
  \pause
  \begin{answer}{Block factorisations}
    \begin{equation*}
      P^{-1} =
      \begin{pmatrix}
        I   & -A^{-1} B^T \\
        0 & I \\
      \end{pmatrix}
      \begin{pmatrix}
        A^{-1}  & 0 \\
        0 & S^{-1} \\
      \end{pmatrix}
      \begin{pmatrix}
        I   & 0 \\
        -BA^{-1} & I \\
      \end{pmatrix}
    \end{equation*}
  \end{answer}
  \begin{challenge}{PDE-specific challenge}
    Find good approximations $\tilde{A}^{-1}$ for $A^{-1}$ and
    $\tilde{S}^{-1}$ for $S^{-1}$.
  \end{challenge}
\end{frame}

\begin{frame}
  \frametitle{Controlling the Schur complement with augmented
    Lagrangian idea}
  \begin{exampleblock}{Continuous augmentation}
    Add $\gamma \grad\div u$ term to the momentum equation.

    Doesn't change solution since $\div u = 0$
  \end{exampleblock}
  \begin{theorem}[Hestenes, Fortin, Glowinski, Olshanksii, \dots]
    As $\gamma \to \infty$, the Schur complement is well
    approximated by $\tilde{S}^{-1} = -(\nu + \gamma)M_p^{-1}$.
  \end{theorem}
  \pause
  \begin{exampleblock}{Discrete augmentation}
    \begin{equation*}
      \begin{pmatrix}
        A + \gamma B^T M_p^{-1} B & B^T \\
        B & 0
      \end{pmatrix}
      \begin{pmatrix}
        u \\ p
      \end{pmatrix}
      =
      \begin{pmatrix}
        b \\ 0
      \end{pmatrix}
    \end{equation*}

    Doesn't change solution since $B u = 0$.

    Again, as $\gamma \to \infty$, $S^{-1}$ is well approximated by
    $\tilde{S}^{-1} = -(\nu + \gamma)M_p^{-1}$.
  \end{exampleblock}
\end{frame}
\begin{frame}
  \frametitle{Why does this work?}
  \begin{answer}{New Schur complement}
    \begin{equation*}
    \begin{aligned}
      S^{-1} &= -(B(A+\gamma B^T M_p^{-1} B)^{-1} B^T)^{-1} \\
      &= -(B(A^{-1} - \gamma A^{-1} B^T(M_p + B A^{-1} B^T)^{-1}BA^{-1})B^T)^{-1}\\
      &= -(\underbrace{BA^{-1}B^T}_{\Lambda} - \gamma BA^{-1}B^T (M_p + BA^{-1} B^T)^{-1}BA^{-1}B^T)^{-1}\\
      &= -(\Lambda^{-1} + \gamma \Lambda^{-1}\Lambda(M_p + \Lambda - \Lambda \Lambda^{-1}\Lambda)^{-1}\Lambda \Lambda^{-1}) \\
      &= -(BA^{-1}B^T)^{-1} - \gamma M_p^{-1}.\\
    \end{aligned}
  \end{equation*}
   approximate commutator argument (ignoring advection):
   \begin{equation*}
     BA^{-1}B^T \approx \nu BB^TA^{-1} \approx \nu M_p^{-1}
   \end{equation*}
  \end{answer}
\end{frame}

\begin{frame}
  \frametitle{Conservation of misery}
  \begin{exampleblock}{Good news}
    \emoji{thumbsup} The Schur complement becomes easy to approximate as $\gamma \to \infty$
  \end{exampleblock}
  \begin{challenge}{Bad news}
    \emoji{thumbsdown} The scalable solver we had for $\tilde{A}^{-1}$ doesn't work for
    $A_\gamma := A + \gamma B^T M_p^{-1} B$
  \end{challenge}
  \pause
  \begin{center}
    \begin{tabular}{l| c |c}
      \toprule
      &  LU for $\tilde{A}_\gamma^{-1}$ & AMG for $\tilde{A}_\gamma^{-1}$\\
      \midrule
      $\gamma=10^{-1}$ & 15 &18\\
      $\gamma=1$ & 6 &40\\
      $\gamma=10^{1}$ & 3 &107\\
      \bottomrule
    \end{tabular}

    Outer Krylov iterations for different choices of $\tilde{A}^{-1}_\gamma$.
  \end{center}
\end{frame}

\begin{frame}
  \frametitle{$\gamma$-robust multigrid I}
  \begin{itemize}
  \item Ignorning advection, then the top-left block corresponds to
    discretisation of
    \begin{equation*}
      a_{\gamma}(u, v) = \xunderbrace{\int_\Omega 2\nu \eps{u}  :
        \eps{v} \dx}_{\text{sym.~pos.~def.}} \quad +  \quad
      \textcolor{red}{\xunderbrace{\int_\Omega \gamma\div u\div v \dx}_{\text{sym.~pos.~semi-def.}}}
    \end{equation*}
    \pause
  \item The semi-definite term is singular on all solenoidal fields
    $\Rightarrow$ the system becomes \emph{nearly singular} as $\gamma
    \to \infty$
    \pause
  \item To build a $\gamma$-robust scheme we need
    \parencite{Schoeberl:1999}
    \begin{itemize}
    \item a $\gamma$-robust smoother;
    \item a prolongation operator with $\gamma$-independent continuity
      constant;
    \item[$\Rightarrow$] overlapping Schwarz smoother with
      decomposition that decomposes the kernel. ``The right blocks for
      block Jacobi''.
    \end{itemize}
  \end{itemize}
\end{frame}

\begin{frame}
  \frametitle{$\gamma$-robust multigrid II}
  Consider the problem: for
  $\gamma \in \mathbb{R}_+$, find $u \in V$ such that
  \begin{equation*}
    a(u, v) + \gamma b(u, v) = (f, v) \quad \forall v \in V,
  \end{equation*}
  where $a$ is SPD, and $b$ is symmetric positive semidefinite.

  This matches our problem (ignoring advection) with $a = \nu (\eps{u},
  \eps{v})$ and $b = (\div u, \div v)$.
  \pause
  \begin{theorem}[Sch\"oberl (1999); Lee, Wu, Xu, Zikatanov (2007)]
    {\small
    Let the kernel be
    \begin{equation*}
      \mathcal{N} := \{ u \in V : b(u, v) = 0 \,\, \forall v \in V \}.
    \end{equation*}
    A Schwarz smoother whose decomposition $V = \sum_i
    V_i$ satisfies
    \begin{equation*}
      \mathcal{N} = \sum_i \mathcal{N} \cap V_i,
    \end{equation*}
    is robust wrt $\gamma$.
    \nocite{Schoeberl:1999,Lee:2007}
    }
  \end{theorem}
\end{frame}

\begin{frame}
  \frametitle{Systematic construction of $V_i$}
  Pick a discrete subcomplex of some complex with exact sequence
  properties.

  Gives easy characterisation of kernel for differential operators.

  \begin{equation*}
    H^2 \xrightarrow{\grad} H^1(\curl) \xrightarrow{\curl} H^1 \xrightarrow{\div} L^2.
  \end{equation*}
  \vspace{-\baselineskip}
  \begin{exampleblock}{Divergence-free}
    \begin{itemize}
    \item Scott--Vogelius pair: $P_k^d-P_{k-1}^\text{disc}$ on barycentrically refined
      meshes ($k \ge d$).
    \end{itemize}
  \begin{figure}
    \begin{center}
      \includegraphics[width=5cm]{macrostar}
    \end{center}
  \end{figure}
  \end{exampleblock}

\end{frame}

\appendix
\begin{frame}[allowframebreaks]
  \frametitle{References}
  \printbibliography[heading=none]
\end{frame}

\end{document}
