\documentclass[presentation,aspectratio=43]{beamer}

\usetheme{metropolis}

\author{Lawrence Mitchell\inst{1,*} \\
  \and {\scriptsize
    P.~E.~Farrell (Oxford)
    \and
    R.~C.~Kirby (Baylor)
    \and
    M.~G.~Knepley (Buffalo)}}
\institute{
  \inst{1}Department of Computer Science, Durham University\\
  \inst{*}\texttt{lawrence.mitchell@durham.ac.uk}}

\title{Robust multigrid solvers for the incompressible Navier-Stokes equations}

\begin{document}

\maketitle

\begin{abstract}
  At the core of the numerical simulation of physical systems is the
  task of inverting discrete representations (matrices) of the
  continuous equations.

  A key goal is to find methods to invert these operators that
  are both fast (at worst O(n log n) in the number of degrees of
  freedom), and robust to parameters in the equations.

  In most cases, we achieve the former by exploiting sparsity in the
  discretisation to define a hierarchy of scales: the multigrid method
  is a prototypical example of this.

  Addressing the second point is more delicate, and a lot depends on
  the form of the equations in question. For nearly singular symmetric
  operators, a robust approach is developed by Lee, Wu, Xu, and
  Zikatanov (2007). To make use of this theory in practical
  applications requires one to characterise the basis of the singular
  part of the operator. I will discuss some recent breakthroughs in
  this characterisation and finish with some numerically motivated
  conjectures regarding finite element bases on tetrahedral meshes.

  This is joint work with Patrick Farrell, L. Ridgway Scott, and
  Florian Wechsung. Some parts are presented in
  https://arxiv.org/abs/1810.03315.
\end{abstract}
\end{document}
