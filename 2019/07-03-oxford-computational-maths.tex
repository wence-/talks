\documentclass[presentation,aspectratio=43]{beamer}

\usetheme{metropolis}

\author{Lawrence Mitchell\inst{1,*} \\
  \and {\scriptsize
    P.~E.~Farrell (Oxford)
    \and
    R.~C.~Kirby (Baylor)
    \and
    M.~G.~Knepley (Buffalo)
    \and
    F.~Wechsung (Oxford)}}
\institute{
  \inst{1}Department of Computer Science, Durham University\\
  \inst{*}\texttt{lawrence.mitchell@durham.ac.uk}}

\title{Flexible computational abstractions for complex
  preconditioners}

\begin{document}

\maketitle

\begin{abstract}
  Small block overlapping, and non-overlapping, Schwarz methods are
  theoretically highly attractive as multilevel smoothers for a wide
  variety of problems that are not amenable to point relaxation
  methods.  Examples include monolithic Vanka smoothers for Stokes,
  overlapping vertex-patch decompositions for $H(\text{div})$ and
  $H(\text{curl})$ problems, along with nearly incompressible
  elasticity, and augmented Lagrangian schemes.

  While it is possible to manually program these different schemes,
  their use in general purpose libraries has been held back by a lack
  of generic, composable interfaces. We present a new approach to the
  specification and development such additive Schwarz methods in PETSc
  that cleanly separates the topological space decomposition from the
  discretisation and assembly of the equations. Our preconditioner is
  flexible enough to support overlapping and non-overlapping additive
  Schwarz methods, and can be used to formulate line, and plane
  smoothers, Vanka iterations, amongst others. I will illustrate these
  new features with some examples utilising the Firedrake finite
  element library, in particular how the design of an approriate
  computational interface enables these schemes to be used as building
  blocks inside block preconditioners.

  This is joint work with Patrick Farrell and Florian Wechsung
  (Oxford), and Matt Knepley (Buffalo).
\end{abstract}
\end{document}
