\documentclass[presentation,aspectratio=43]{beamer}
\usepackage{tikz}
\usetikzlibrary{positioning,calc}
\usetikzlibrary{shapes.geometric}
\usetikzlibrary{backgrounds}% only to show the bounding box
\usetikzlibrary{shapes,arrows}
\usepackage{appendixnumberbeamer}
\usepackage{amsmath}
\newif\ifwidescreen
\widescreenfalse
\date{6th June 2018}
\usetheme{metropolis}

\metroset{background=light,progressbar=frametitle,numbering=counter}

\renewcommand{\vec}[1]{\ensuremath{\boldsymbol{#1}}}
\newcommand{\ddt}[1]{\frac{\partial #1}{\partial t}}
\newcommand{\zhat}{\hat{\vec{z}}}
\newcommand{\W}{\ensuremath{\mathbb{W}}}

\newcommand{\inner}[1]{\left\langle #1 \right \rangle}

\newcommand{\KSP}[2]{\ensuremath{\mathcal{K}\left(#1, \mathbb{#2}\right)}}
\newcommand{\ksp}[1]{\KSP{#1}{#1}}

\newcommand{\colourfiredrake}[1]{\colorbox{red!20}{#1}}
\newcommand{\colourpetsc}[1]{\colorbox{blue!20}{#1}}

\author{Lawrence Mitchell\inst{1,*} \and Rob Kirby\inst{2,\dag}
  \and Patrick Farrell\inst{3,\ddag}}
\institute{
\inst{1}Departments of Computing and Mathematics, Imperial College
London

\inst{*}\texttt{lawrence.mitchell@imperial.ac.uk}
\and
\inst{2}Department of Mathematics, Baylor University

\inst{\dag}\texttt{robert\_kirby@baylor.edu}
\and
\inst{3}Mathematical Institute, University of Oxford

\inst{\ddag}\texttt{patrick.farrell@maths.ox.ac.uk}
}

\graphicspath{{./\jobname.figures/}{../pictures/}}
\usepackage[url=false,
            doi=true,
            isbn=false,
            style=authoryear,
            firstinits=true,
            uniquename=init,
            backend=biber]{biblatex}

\setbeamertemplate{bibliography item}{}
\renewcommand{\bibfont}{\fontsize{7}{7}\selectfont}
\addbibresource{references.bib}
\newcommand{\arxivlink}[2]{%
  \href{http://www.arxiv.org/abs/#1}%
  {{\small\texttt{arXiv:\,#1\,[#2]}}}%
}

\setlength{\bibitemsep}{1ex}
\setlength{\fboxsep}{1pt}

\renewbibmacro{in:}{}
\DeclareFieldFormat[article]{volume}{\textbf{#1}}
\DeclareFieldFormat{doi}{%
  doi\addcolon%
  {\ifhyperref{\href{http://dx.doi.org/#1}{\nolinkurl{#1}}}
    {\nolinkurl{#1}}}}
\AtEveryBibitem{%
\clearfield{pages}%
\clearfield{issue}%
\clearfield{number}%
}

\usepackage{minted}
\title{PDEs should be the solvers problem}

\titlegraphic{\hfill\includegraphics[height=1.1cm]{firedrake-word.pdf}}
\begin{document}

% \begin{abstract}
%   Many optimal solvers for PDEs require access to auxiliary
%   operators, or compositions thereof, over and above what is easily
%   offered by PETSc's Amat, Pmat interface.  Although possible,
%   setting things up ``by hand'' is tricky, error prone, and requires
%   changing to compare the performance of different solver options.
%   This is especially the case when we wish to provide
%   problem-specific data deep in some nested solver.

%   In this talk, I will describe how we address some of these problems in
%   Firedrake, by augmenting operators (and hence preconditioners) with
%   the ability to provide auxiliary operators as needed.  By more tightly
%   coupling the PDE library with the linear algebra, we can make solvers
%   problem- and discretisation-aware.

%   Recently, we have taken this approach to develop a very flexible
%   framework for domain-decomposition preconditioning, utilising DMPlex
%   to define topological patches, and the auxiliary information to
%   provide operator assembly.  Most of this is not specific to Firedrake,
%   and so the question naturally arises as to how to develop
%   discretisation- and problem-aware preconditioning infrastructure that
%   can live in PETSc, yet be usable by the plethora of PDE libraries in
%   the wider community.  I do not have the answer to this question, but
%   am hopeful of useful discussion.
% \end{abstract}

\bgroup
\setbeamertemplate{title graphic}{
  \vbox to 0pt {
    \vspace*{0.25em}
    \makebox[1.025\textwidth]{\inserttitlegraphic}%
  }%
  \nointerlineskip%
}
\setbeamertemplate{background}{%
  \raisebox{-\paperheight}[0pt][0pt]{%
    \makebox[\paperwidth][c]{%
      \includegraphics[width=\paperwidth]{bottom-swoosh}%
    }%
  }%
}
\begin{frame}
  \maketitle{}
\end{frame}
\egroup

\end{document}
