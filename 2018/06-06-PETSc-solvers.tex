\documentclass[presentation,aspectratio=43]{beamer}
\usepackage{tikz}
\usetikzlibrary{positioning,calc}
\usetikzlibrary{shapes.geometric}
\usetikzlibrary{backgrounds}% only to show the bounding box
\usetikzlibrary{shapes,arrows}
\usepackage{appendixnumberbeamer}
\usepackage{amsmath}
\newif\ifwidescreen
\widescreenfalse
\date{6th June 2018}
\usetheme{metropolis}

\metroset{background=light,progressbar=frametitle,numbering=counter}

\renewcommand{\vec}[1]{\ensuremath{\boldsymbol{#1}}}
\newcommand{\ddt}[1]{\frac{\partial #1}{\partial t}}
\newcommand{\zhat}{\hat{\vec{z}}}
\newcommand{\W}{\ensuremath{\mathbb{W}}}

\newcommand{\inner}[1]{\left\langle #1 \right \rangle}

\newcommand{\KSP}[2]{\ensuremath{\mathcal{K}\left(#1, \mathbb{#2}\right)}}
\newcommand{\ksp}[1]{\KSP{#1}{#1}}

\newcommand{\colourfiredrake}[1]{\colorbox{red!20}{#1}}
\newcommand{\colourpetsc}[1]{\colorbox{blue!20}{#1}}

\author{Lawrence Mitchell\inst{1,*} \and Rob Kirby\inst{2,\dag}
  \and Patrick Farrell\inst{3,\ddag}}
\institute{
\inst{1}Departments of Computing and Mathematics, Imperial College
London

\inst{*}\texttt{lawrence.mitchell@imperial.ac.uk}
\and
\inst{2}Department of Mathematics, Baylor University

\inst{\dag}\texttt{robert\_kirby@baylor.edu}
\and
\inst{3}Mathematical Institute, University of Oxford

\inst{\ddag}\texttt{patrick.farrell@maths.ox.ac.uk}
}

\graphicspath{{./\jobname.figures/}{../pictures/}}
\usepackage[url=false,
            doi=true,
            isbn=false,
            style=authoryear,
            firstinits=true,
            uniquename=init,
            backend=biber]{biblatex}

\setbeamertemplate{bibliography item}{}
\renewcommand{\bibfont}{\fontsize{7}{7}\selectfont}
\addbibresource{references.bib}
\newcommand{\arxivlink}[2]{%
  \href{http://www.arxiv.org/abs/#1}%
  {{\small\texttt{arXiv:\,#1\,[#2]}}}%
}

\setlength{\bibitemsep}{1ex}
\setlength{\fboxsep}{1pt}

\renewbibmacro{in:}{}
\DeclareFieldFormat[article]{volume}{\textbf{#1}}
\DeclareFieldFormat{doi}{%
  doi\addcolon%
  {\ifhyperref{\href{http://dx.doi.org/#1}{\nolinkurl{#1}}}
    {\nolinkurl{#1}}}}
\AtEveryBibitem{%
\clearfield{pages}%
\clearfield{issue}%
\clearfield{number}%
}

\usepackage{minted}
\title{PDEs should be the solver's problem}

\titlegraphic{\hfill\includegraphics[height=0.8cm]{firedrake-word.pdf}}
\begin{document}

% \begin{abstract}
%   Many optimal solvers for PDEs require access to auxiliary
%   operators, or compositions thereof, over and above what is easily
%   offered by PETSc's Amat, Pmat interface.  Although possible,
%   setting things up ``by hand'' is tricky, error prone, and requires
%   changing to compare the performance of different solver options.
%   This is especially the case when we wish to provide
%   problem-specific data deep in some nested solver.

%   In this talk, I will describe how we address some of these problems in
%   Firedrake, by augmenting operators (and hence preconditioners) with
%   the ability to provide auxiliary operators as needed.  By more tightly
%   coupling the PDE library with the linear algebra, we can make solvers
%   problem- and discretisation-aware.

%   Recently, we have taken this approach to develop a very flexible
%   framework for domain-decomposition preconditioning, utilising DMPlex
%   to define topological patches, and the auxiliary information to
%   provide operator assembly.  Most of this is not specific to Firedrake,
%   and so the question naturally arises as to how to develop
%   discretisation- and problem-aware preconditioning infrastructure that
%   can live in PETSc, yet be usable by the plethora of PDE libraries in
%   the wider community.  I do not have the answer to this question, but
%   am hopeful of useful discussion.
% \end{abstract}

\bgroup
\setbeamertemplate{title graphic}{
  \vbox to 0pt {
    \vspace*{0.25em}
    \makebox[1.075\textwidth]{\inserttitlegraphic}%
  }%
  \nointerlineskip%
}
\setbeamertemplate{background}{%
  \raisebox{-\paperheight}[0pt][0pt]{%
    \makebox[\paperwidth][c]{%
      \includegraphics[width=\paperwidth]{bottom-swoosh}%
    }%
  }%
}
\begin{frame}
  \maketitle{}
\end{frame}
\egroup

\section{Block preconditioning}
\begin{frame}
  \frametitle{A motivating problem,}
  \begin{block}{Stationary Rayleigh-B\'enard convection}
    \begin{equation*}
      \begin{split}
        -\Delta u + u\cdot\nabla u + \nabla p +
        \frac{\text{Ra}}{\text{Pr}} \hat{g}T &= 0 \\
        \nabla \cdot u &= 0 \\
        - \frac{1}{\text{Pr}} \Delta T + u\cdot \nabla T &= 0
      \end{split}
    \end{equation*}
    Newton linearisation
    \begin{equation*}
      \begin{bmatrix}
        F   & B^T & M_1 \\
        C   & 0   & 0   \\
        M_2 & 0 & K
      \end{bmatrix}
      \begin{bmatrix}
        \delta u \\
        \delta p \\
        \delta T
      \end{bmatrix} =
      \begin{bmatrix}
        f_1 \\
        f_2 \\
        f_3
      \end{bmatrix}
    \end{equation*}
  \end{block}
\end{frame}
\begin{frame}
  \frametitle{and a preconditioner,}
  {\small
  For each Newton step, solve
  \begin{equation*}
    \KSP{\begin{bmatrix}
        F & B^T & M_1\\
        C & 0 & 0 \\
        M_2 & 0 & K
      \end{bmatrix}}{J}
  \end{equation*}
  using a preconditioner from \textcite{Howle:2012}:
  \begin{equation*}
    \mathbb{J} =
    \begin{bmatrix}
      \KSP{\begin{bmatrix}
          F & B^T\\
          C & 0
        \end{bmatrix}}{N} & 0\\
      0 & I
    \end{bmatrix}
    \begin{bmatrix}
      I & 0 & -M_1\\
      0 & I & 0 \\
      0 & 0 & I
    \end{bmatrix}
    \begin{bmatrix}
      I & 0 & 0\\
      0 & I & 0\\
      0 & 0 &\ksp{K}
    \end{bmatrix}
  \end{equation*}
  with
  \begin{equation*}
    \mathbb{N} = \begin{bmatrix}
      F & 0 \\
      0 & \mathcal{K}(S_p, \KSP{M_p}{M}(\mathbb{I} + F_p \, \KSP{L_p}{L})
    \end{bmatrix}
    \begin{bmatrix}
      I & 0\\
      -C & I
    \end{bmatrix}
    \begin{bmatrix}
      \ksp{F} & 0 \\
      0 & I
    \end{bmatrix}
  \end{equation*}
  and
  \begin{equation*}
    S_p = -C \ksp{F} B^T.
  \end{equation*}
}
\end{frame}
\begin{frame}[fragile,t]
  \frametitle{and some solver options}
  \begin{columns}[t]
    \begin{column}{0.49\textwidth}
\begin{minted}[fontsize=\tiny,escapeinside=||]{python3}
-ksp_type fgmres
-pc_type fieldsplit
-pc_fieldsplit_type multiplicative
-pc_fieldsplit_0_fields 0,1
-pc_fieldsplit_1_fields 2
   -fieldsplit_0_
      -ksp_type gmres
      -pc_type fieldsplit
      -pc_fieldsplit_type schur
      -pc_fieldsplit_schur_fact_type lower
      -fieldsplit_0_ksp_type preonly
      -fieldsplit_0_pc_type gamg
      -fieldsplit_1_ksp_type preonly
      -fieldsplit_1_pc_type |\alert{\textbf{XXX}}|
   -fieldsplit_1_
      -ksp_type gmres
      -pc_type hypre
\end{minted}
    \end{column}
    \hspace{0.02\textwidth}
    \begin{column}{0.49\textwidth}
      {\tiny
      \begin{equation*}
        \begin{bmatrix}
          \begin{bmatrix}
              F & B^T\\
              C & 0
            \end{bmatrix}^{-1} & 0\\
          0 & I
        \end{bmatrix}
        \begin{bmatrix}
          I & 0 & -M_1\\
          0 & I & 0 \\
          0 & 0 & I
        \end{bmatrix}
        \begin{bmatrix}
          I & 0 & 0\\
          0 & I & 0\\
          0 & 0 & K^{-1}
        \end{bmatrix}
      \end{equation*}
      \\[\baselineskip]
      \begin{equation*}
        \begin{bmatrix}
          F & 0 \\
          0 & S_p^{-1}
        \end{bmatrix}
        \begin{bmatrix}
          I & 0\\
          -C & I
        \end{bmatrix}
        \begin{bmatrix}
          F^{-1} & 0 \\
          0 & I
        \end{bmatrix}
      \end{equation*}
      \\[0.5\baselineskip]
      \begin{equation*}
        F^{-1} \approx \text{\texttt{gamg}}(F)
      \end{equation*}
      \begin{equation*}
        S_p^{-1} \approx M_p^{-1}(\mathbb{I} + F_p L_p^{-1})\quad\text{PCD approximation}
      \end{equation*}
      \\[0.5\baselineskip]
      \begin{equation*}
        K^{-1} \approx \text{\texttt{hypre}}(K)
      \end{equation*}
      }
    \end{column}
  \end{columns}

  \begin{alertblock}<2->{Problem}
    How do I get $L_p^{-1}$, $M_p^{-1}$, and $F_p$ into the solver?
  \end{alertblock}
\end{frame}

\begin{frame}
  \frametitle{Idea}
  \begin{itemize}
  \item Endow discretised operators with PDE-level information:
    \begin{itemize}
    \item what equation/function space?
    \item boundary conditions, etc\ldots
    \end{itemize}
  \item Enable standard fieldsplits on these operators.
  \item Write custom preconditioners that utilise this information appropriately.
  \end{itemize}
  \begin{block}<2->{Extend PETSc with Firedrake-level PCs}
  \begin{itemize}
  \item PETSc already provides \emph{algebraic} composition of solvers. \nocite{Brown:2012}

  \item Firedrake can provide auxiliary operators

  \item We just need to combine these appropriately.
  \end{itemize}
  \end{block}
\end{frame}

\begin{frame}[fragile,t]
  \frametitle{Implementation: two parts}
  \begin{block}{A new matrix type}
    A shell matrix that implements matrix-free actions, and contains
    the symbolic information about the bilinear form.
    \begin{columns}
      \begin{column}{0.5\textwidth}
      \begin{equation*}
        y \leftarrow A x
      \end{equation*}
    \end{column}
    \hspace{-0.1\textwidth}
    \begin{column}{0.7\textwidth}
\begin{minted}[fontsize=\scriptsize]{python}
A = assemble(a, mat_type="matfree")
\end{minted}
    \end{column}
    \end{columns}
    \begin{flushright}
      \footnotesize Could do this all with assembled matrices if desired.
    \end{flushright}
  \end{block}
  \begin{block}{Custom preconditioners}
    These matrices do not have entries, we create preconditioners
    that inspect the UFL and do the appropriate thing.
    \begin{columns}
      \begin{column}{0.5\textwidth}
    \begin{equation*}
      y \leftarrow \tilde{A}^{-1} x
    \end{equation*}
  \end{column}
  \hspace{-0.1\textwidth}
  \begin{column}{0.7\textwidth}
\begin{minted}[fontsize=\scriptsize]{python}
solve(a == L, x,
      {"mat_type": "matfree",
       "pc_type": "python",
       "pc_python_type": "AssembledPC"})
\end{minted}
  \end{column}
  \end{columns}
  \end{block}
\end{frame}
\begin{frame}[fragile]
  \frametitle{This sounds like hard work}
  Fortunately, \texttt{petsc4py} makes it easy to write these PCs.
\begin{minted}[fontsize=\tiny,escapeinside=||,mathescape=true]{python3}
class MyPC(object):
    def setUp(self, pc):
        A, P = pc.getOperators()
        # A and P are shell matrices, carrying the symbolic
        # discretisation information.
        # So I have access to the mesh, function spaces, etc...
        # Can inspect options dictionary here
        # do whatever
    def apply(self, pc, r, e):
        # Compute approximation to error given current residual
        # $e \gets A^{-1} r$

solve(..., solver_parameters={"pc_type": "python",
                              "pc_python_type": "MyPC"})
\end{minted}
  PETSc manages all the splitting and nesting already.  So this does
  the right thing \emph{inside} multigrid, etc\ldots
\end{frame}

\begin{frame}[fragile,t]
  \frametitle{Back to Rayleigh-B\'enard}
  \begin{onlyenv}<1-2>
    \begin{alertblock}{Problem}
      How do I get $L_p^{-1}$, $M_p^{-1}$, and $F_p$ into the solver?
    \end{alertblock}
    \begin{alertblock}<2>{Solution}
      Write a custom PC for that makes them, calling back to the PDE library.
    \end{alertblock}
  \end{onlyenv}
  \begin{onlyenv}<3>
    \begin{columns}[T]
      \begin{column}{0.49\textwidth}
\begin{minted}[fontsize=\tiny,mathescape=true]{python3}
class PCDPC(PCBase):
    def initialize(self, pc):
        _, P = pc.getOperators()
        prefix = pc.getOptionsPrefix()
        ctx = P.getPythonContext()
        p, q = ctx.a.arguments()
        ...
        # convection operator
        fp = Re*dot(grad(p), u0)*q*dx
        self.Fp = assemble(fp, options_prefix=prefix + "fp_")
        # pressure laplacian
        laplace = inner(grad(p), grad(q))*dx
        Lp = assemble(laplace, bcs=bcs, options_prefix=prefix + "lp_")
        self.Lksp = PETSc.KSP().create(comm=pc.comm)
        self.Lksp.incrementTabLevel(1, parent=pc)
        self.Lksp.setOptionsPrefix(prefix + "lp_")
        self.Lksp.setOperators(Kp.petscmat)
        self.Lksp.setFromOptions()
        # pressure mass matrix
        mass = Re*p*q*dx
        Mp = assemble(mass, options_prefix=prefix + "mp_")
        self.Mksp = PETSc.KSP().create(comm=pc.comm)
        self.Mksp.incrementTabLevel(1, parent=pc)
        self.Mksp.setOptionsPrefix(prefix + "mp_")
        self.Mksp.setOperators(Mp.petscmat)
        self.Mksp.setFromOptions()
\end{minted}
      \end{column}
      \begin{column}{0.49\textwidth}
\begin{minted}[fontsize=\tiny,mathescape=true]{python3}
    def apply(self, pc, x, y):
        # $y \gets M^{-1}(\mathbb{I} + F_p L_p^{-1})x$
        z = self.work
        x.copy(z)
        self.bcs.apply(z)
        self.Lksp.solve(z, y)
        self.Fp.petscmat.mult(y, z)
        z.axpy(1.0, x)
        self.Mksp.solve(z, y)
\end{minted}
      \end{column}
    \end{columns}
  \end{onlyenv}
\end{frame}
\begin{frame}[fragile,t]
  \frametitle{and some more solver options}
  \begin{columns}[t]
    \begin{column}{0.49\textwidth}
\begin{minted}[fontsize=\tiny,escapeinside=||]{python3}
-ksp_type fgmres
-pc_type fieldsplit
-pc_fieldsplit_type multiplicative
-pc_fieldsplit_0_fields 0,1
-pc_fieldsplit_1_fields 2
   -fieldsplit_0_
      -ksp_type gmres
      -pc_type fieldsplit
      -pc_fieldsplit_type schur
      -pc_fieldsplit_schur_fact_type lower
      -fieldsplit_0_ksp_type preonly
      -fieldsplit_0_pc_type gamg
      -fieldsplit_1_
         -ksp_type preonly
         -pc_type python
         -pc_python_type PCDPC
         -lp_ksp_type preonly
         -lp_pc_type hypre
         -mp_ksp_type preonly
         -mp_pc_type sor
   -fieldsplit_1_
      -ksp_type gmres
      -pc_type hypre
\end{minted}
    \end{column}
    \hspace{0.02\textwidth}
    \begin{column}{0.49\textwidth}
      {\tiny
        \begin{equation*}
          \begin{bmatrix}
            \begin{bmatrix}
              F & B^T\\
              C & 0
            \end{bmatrix}^{-1} & 0\\
            0 & I
          \end{bmatrix}
          \begin{bmatrix}
            I & 0 & -M_1\\
            0 & I & 0 \\
            0 & 0 & I
          \end{bmatrix}
          \begin{bmatrix}
            I & 0 & 0\\
            0 & I & 0\\
            0 & 0 & K^{-1}
          \end{bmatrix}
        \end{equation*}
        \\[\baselineskip]
        \begin{equation*}
          \begin{bmatrix}
            F & 0 \\
            0 & S_p^{-1}
          \end{bmatrix}
          \begin{bmatrix}
            I & 0\\
            -C & I
          \end{bmatrix}
          \begin{bmatrix}
            F^{-1} & 0 \\
            0 & I
          \end{bmatrix}
        \end{equation*}
        \\[0.5\baselineskip]
        \begin{equation*}
          F^{-1} \approx \text{\texttt{gamg}}(F)
        \end{equation*}
        \\[1.25\baselineskip]
        \begin{equation*}
          S_p^{-1} \approx M_p^{-1}(\mathbb{I} + F_p L_p^{-1})\quad\text{PCD approximation}
        \end{equation*}
        \vspace{-1\baselineskip}
        \begin{equation*}
          L_p^{-1} \approx \text{\texttt{hypre}}(L_p)
        \end{equation*}
        \vspace{-1\baselineskip}
        \begin{equation*}
          M_p^{-1} \approx \text{\texttt{sor}}(M_p)
        \end{equation*}
        \\[0.4\baselineskip]
        \begin{equation*}
          K^{-1} \approx \text{\texttt{hypre}}(K)
        \end{equation*}
      }
    \end{column}
  \end{columns}
  \vspace{-\baselineskip}
  {\scriptsize \textcite[\S B.4]{Kirby:2018} shows a complete solver configuration.}
\end{frame}

\section{Schwarz smoothers}

\begin{frame}
  \frametitle{Now that I have a hammer\ldots}
  \ldots can I find some nails?

  \begin{block}<2->{Schwarz building blocks}
    \begin{enumerate}
    \item Subspace decomposition
    \item Operators on subspaces
    \item Solvers on subspaces
    \item Coarse spaces (not yet)
    \end{enumerate}
  \end{block}

  \begin{block}<3->{\texttt{-pc\_type patch}}
    \begin{itemize}
    \item \texttt{DMPlex} + \texttt{PetscSection} for subspace decomposition
    \item Callback interface (to Firedrake for now) to build operators
    \item \texttt{KSP} on each patch for solves
    \end{itemize}
  \end{block}
\end{frame}

\begin{frame}[fragile]
  \frametitle{Subspace definition}
  Each patch defined by set of mesh points on which dofs are free.
  \begin{block}{Builtin}
    Specify patches by selecting:
    \begin{enumerate}
    \item Mesh points $\{p_i\}$ to iterate over (e.g.~vertices, cells)
    \item Adjacency relation that gathers points in patch
      \begin{itemize}
      \item[\texttt{star}] points in $\text{star}(p_i)$
      \item[\texttt{vanka}] points in $\text{closure}(\text{star}(p_i))$
      \end{itemize}
    \end{enumerate}
  \end{block}
  \begin{block}{User-defined}
    Callback provides \texttt{IS}es for each patch, plus iteration order.
\begin{minted}[fontsize=\tiny]{c}
PetscErrorCode UserPatches(PC, PetscInt*, IS**, IS*, void*);
\end{minted}
  \end{block}
\end{frame}

\begin{frame}[t]
  \frametitle{Patch construction: \texttt{star}}
  \begin{columns}[T]
    \begin{column}{0.49\textwidth}
      \begin{tikzpicture}[scale=4]
        \tikzstyle{v}=[circle, fill, minimum size=0pt, inner sep=0pt]
        \tikzstyle{c}=[diamond, fill, minimum size=4pt, inner sep=0pt]
        \tikzstyle{e}=[rectangle, fill, minimum size=3pt, inner sep=0pt]
        \tikzstyle{select}=[orange, minimum size=5pt]
        \tikzstyle{star}=[magenta]
        \tikzstyle{complete}=[brown]
        \node[v] (v1) at (0, 0) {};
        \node[v] (v2) at (0.25, 0) {};
        \node[v] (v3) at (0.5, 0) {};
        \node[v] (v4) at (0.75, 0) {};
        \node[v] (v5) at (1, 0) {};
        \node[v] (v6) at (0, 0.25) {};
        \node[v] (v7) at (0, 0.5) {};
        \node[v] (v8) at (0, 0.75) {};
        \node[v] (v9) at (0, 1) {};
        \node[v] (v10) at (0.25, 0.25) {};
        \node[v] (v11) at (0.25, 0.5) {};
        \node[v] (v12) at (0.25, 0.75) {};
        \node[v] (v13) at (0.25, 1) {};
        \node[v] (v14) at (0.5, 0.25) {};
        \node[v] (v15) at (0.5, 0.5) {};
        \node[v] (v16) at (0.5, 0.75) {};
        \node[v] (v17) at (0.5, 1) {};
        \node[v] (v18) at (0.75, 0.25) {};
        \node[v] (v19) at (0.75, 0.5) {};
        \node[v] (v20) at (0.75, 0.75) {};
        \node[v] (v21) at (0.75, 1) {};
        \node[v] (v22) at (1, 0.25) {};
        \node[v] (v23) at (1, 0.5) {};
        \node[v] (v24) at (1, 0.75) {};
        \node[v] (v25) at (1, 1) {};

        \draw (v1) -- (v2) -- (v6) -- (v1);
        \draw (v2) -- (v10) -- (v6);
        \draw (v2) -- (v3) -- (v10);
        \draw (v3) -- (v14) -- (v10);
        \draw (v3) -- (v4) -- (v14);
        \draw (v4) -- (v18) -- (v14);
        \draw (v4) -- (v5) -- (v18);
        \draw (v5) -- (v22) -- (v18);

        \draw (v6) -- (v7) -- (v10);
        \draw (v7) -- (v11) -- (v10);
        \draw (v11) -- (v14) -- (v15) -- (v11);
        \draw (v15) -- (v18) -- (v19) -- (v15);
        \draw (v19) -- (v22) -- (v23) -- (v19);

        \draw (v7) -- (v8) -- (v11);
        \draw (v8) -- (v12) -- (v11);
        \draw (v12) -- (v15) -- (v16) -- (v12);
        \draw (v16) -- (v19) -- (v20) -- (v16);
        \draw (v20) -- (v23) -- (v24) -- (v20);

        \draw (v8) -- (v9) -- (v12);
        \draw (v9) -- (v13) -- (v12);
        \draw (v13) -- (v16) -- (v17) -- (v13);
        \draw (v17) -- (v20) -- (v21) -- (v17);
        \draw (v21) -- (v24) -- (v25) -- (v21);

        \uncover<2->{
          \node[v, select] at (v15) {};
        }
        \uncover<3->{
          \node[c, star] at (barycentric cs:v11=1,v14=1,v15=1) {};
          \node[c, star] at (barycentric cs:v14=1,v15=1,v18=1) {};
          \node[c, star] at (barycentric cs:v15=1,v18=1,v19=1) {};
          \node[c, star] at (barycentric cs:v11=1,v12=1,v15=1) {};
          \node[c, star] at (barycentric cs:v12=1,v15=1,v16=1) {};
          \node[c, star] at (barycentric cs:v15=1,v16=1,v19=1) {};
          \node[e, star] at (barycentric cs:v11=1,v15=1) {};
          \node[e, star] at (barycentric cs:v15=1,v19=1) {};
          \node[e, star] at (barycentric cs:v14=1,v15=1) {};
          \node[e, star] at (barycentric cs:v15=1,v16=1) {};
          \node[e, star] at (barycentric cs:v18=1,v15=1) {};
          \node[e, star] at (barycentric cs:v15=1,v12=1) {};
        }
        \uncover<4->{
          \node[v, complete, minimum size=4pt] at (v14) {};
          \node[v, complete, minimum size=4pt] at (v18) {};
          \node[v, complete, minimum size=4pt] at (v11) {};
          \node[v, complete, minimum size=4pt] at (v19) {};
          \node[v, complete, minimum size=4pt] at (v12) {};
          \node[v, complete, minimum size=4pt] at (v16) {};
          \node[e, complete] at (barycentric cs:v14=1,v18=1) {};
          \node[e, complete] at (barycentric cs:v12=1,v16=1) {};
          \node[e, complete] at (barycentric cs:v11=1,v12=1) {};
          \node[e, complete] at (barycentric cs:v18=1,v19=1) {};
          \node[e, complete] at (barycentric cs:v14=1,v11=1) {};
          \node[e, complete] at (barycentric cs:v19=1,v16=1) {};
        }
      \end{tikzpicture}
    \end{column}
    \hspace{-2em}
    \begin{column}{0.6\textwidth}
      {\small
        \begin{block}{Looping over vertices}
          \begin{itemize}
          \item<2-> Select mesh point
          \item<3-> Add points in star
          \item<4-> Complete with FEM adjacency
          \end{itemize}
        \end{block}
      }
    \end{column}
  \end{columns}
\end{frame}
\begin{frame}[t]
  \frametitle{Patch construction: \texttt{vanka}}
  \begin{columns}[T]
    \begin{column}{0.49\textwidth}
      \begin{tikzpicture}[scale=4]
        \tikzstyle{v}=[circle, fill, minimum size=0pt, inner sep=0pt]
        \tikzstyle{c}=[diamond, fill, minimum size=4pt, inner sep=0pt]
        \tikzstyle{e}=[rectangle, fill, minimum size=3pt, inner sep=0pt]
        \tikzstyle{select}=[orange, minimum size=5pt]
        \tikzstyle{star}=[magenta]
        \tikzstyle{closure}=[blue]
        \tikzstyle{complete}=[brown]
        \node[v] (v1) at (0, 0) {};
        \node[v] (v2) at (0.25, 0) {};
        \node[v] (v3) at (0.5, 0) {};
        \node[v] (v4) at (0.75, 0) {};
        \node[v] (v5) at (1, 0) {};
        \node[v] (v6) at (0, 0.25) {};
        \node[v] (v7) at (0, 0.5) {};
        \node[v] (v8) at (0, 0.75) {};
        \node[v] (v9) at (0, 1) {};
        \node[v] (v10) at (0.25, 0.25) {};
        \node[v] (v11) at (0.25, 0.5) {};
        \node[v] (v12) at (0.25, 0.75) {};
        \node[v] (v13) at (0.25, 1) {};
        \node[v] (v14) at (0.5, 0.25) {};
        \node[v] (v15) at (0.5, 0.5) {};
        \node[v] (v16) at (0.5, 0.75) {};
        \node[v] (v17) at (0.5, 1) {};
        \node[v] (v18) at (0.75, 0.25) {};
        \node[v] (v19) at (0.75, 0.5) {};
        \node[v] (v20) at (0.75, 0.75) {};
        \node[v] (v21) at (0.75, 1) {};
        \node[v] (v22) at (1, 0.25) {};
        \node[v] (v23) at (1, 0.5) {};
        \node[v] (v24) at (1, 0.75) {};
        \node[v] (v25) at (1, 1) {};

        \draw (v1) -- (v2) -- (v6) -- (v1);
        \draw (v2) -- (v10) -- (v6);
        \draw (v2) -- (v3) -- (v10);
        \draw (v3) -- (v14) -- (v10);
        \draw (v3) -- (v4) -- (v14);
        \draw (v4) -- (v18) -- (v14);
        \draw (v4) -- (v5) -- (v18);
        \draw (v5) -- (v22) -- (v18);

        \draw (v6) -- (v7) -- (v10);
        \draw (v7) -- (v11) -- (v10);
        \draw (v11) -- (v14) -- (v15) -- (v11);
        \draw (v15) -- (v18) -- (v19) -- (v15);
        \draw (v19) -- (v22) -- (v23) -- (v19);

        \draw (v7) -- (v8) -- (v11);
        \draw (v8) -- (v12) -- (v11);
        \draw (v12) -- (v15) -- (v16) -- (v12);
        \draw (v16) -- (v19) -- (v20) -- (v16);
        \draw (v20) -- (v23) -- (v24) -- (v20);

        \draw (v8) -- (v9) -- (v12);
        \draw (v9) -- (v13) -- (v12);
        \draw (v13) -- (v16) -- (v17) -- (v13);
        \draw (v17) -- (v20) -- (v21) -- (v17);
        \draw (v21) -- (v24) -- (v25) -- (v21);

        \uncover<2->{
          \node[v, select] at (v15) {};
        }
        \uncover<3->{
          \node[c, star] at (barycentric cs:v11=1,v14=1,v15=1) {};
          \node[c, star] at (barycentric cs:v14=1,v15=1,v18=1) {};
          \node[c, star] at (barycentric cs:v15=1,v18=1,v19=1) {};
          \node[c, star] at (barycentric cs:v11=1,v12=1,v15=1) {};
          \node[c, star] at (barycentric cs:v12=1,v15=1,v16=1) {};
          \node[c, star] at (barycentric cs:v15=1,v16=1,v19=1) {};
          \node[e, star] at (barycentric cs:v11=1,v15=1) {};
          \node[e, star] at (barycentric cs:v15=1,v19=1) {};
          \node[e, star] at (barycentric cs:v14=1,v15=1) {};
          \node[e, star] at (barycentric cs:v15=1,v16=1) {};
          \node[e, star] at (barycentric cs:v18=1,v15=1) {};
          \node[e, star] at (barycentric cs:v15=1,v12=1) {};
        }

        \uncover<4->{
          \node[v, closure, minimum size=4pt] at (v14) {};
          \node[v, closure, minimum size=4pt] at (v18) {};
          \node[v, closure, minimum size=4pt] at (v11) {};
          \node[v, closure, minimum size=4pt] at (v19) {};
          \node[v, closure, minimum size=4pt] at (v12) {};
          \node[v, closure, minimum size=4pt] at (v16) {};
          \node[e, closure] at (barycentric cs:v14=1,v18=1) {};
          \node[e, closure] at (barycentric cs:v12=1,v16=1) {};
          \node[e, closure] at (barycentric cs:v11=1,v12=1) {};
          \node[e, closure] at (barycentric cs:v18=1,v19=1) {};
          \node[e, closure] at (barycentric cs:v14=1,v11=1) {};
          \node[e, closure] at (barycentric cs:v19=1,v16=1) {};
        }

        \uncover<5->{
          \node[v, complete, minimum size=4pt] at (v3) {};
          \node[v, complete, minimum size=4pt] at (v4) {};
          \node[v, complete, minimum size=4pt] at (v5) {};
          \node[v, complete, minimum size=4pt] at (v10) {};
          \node[v, complete, minimum size=4pt] at (v13) {};
          \node[v, complete, minimum size=4pt] at (v7) {};
          \node[v, complete, minimum size=4pt] at (v23) {};
          \node[v, complete, minimum size=4pt] at (v8) {};
          \node[v, complete, minimum size=4pt] at (v20) {};
          \node[v, complete, minimum size=4pt] at (v9) {};
          \node[v, complete, minimum size=4pt] at (v13) {};
          \node[v, complete, minimum size=4pt] at (v17) {};
          \node[c, complete] at (barycentric cs:v3=1,v14=1,v10=1) {};
          \node[c, complete] at (barycentric cs:v3=1,v4=1,v14=1) {};
          \node[c, complete] at (barycentric cs:v4=1,v18=1,v14=1) {};
          \node[c, complete] at (barycentric cs:v4=1,v5=1,v18=1) {};
          \node[c, complete] at (barycentric cs:v5=1,v22=1,v18=1) {};
          \node[c, complete] at (barycentric cs:v7=1,v11=1,v10=1) {};
          \node[c, complete] at (barycentric cs:v10=1,v11=1,v14=1) {};
          \node[c, complete] at (barycentric cs:v18=1,v19=1,v22=1) {};
          \node[c, complete] at (barycentric cs:v19=1,v22=1,v23=1) {};
          \node[c, complete] at (barycentric cs:v7=1,v8=1,v11=1) {};
          \node[c, complete] at (barycentric cs:v8=1,v12=1,v11=1) {};
          \node[c, complete] at (barycentric cs:v16=1,v19=1,v20=1) {};
          \node[c, complete] at (barycentric cs:v19=1,v20=1,v23=1) {};
          \node[c, complete] at (barycentric cs:v8=1,v9=1,v12=1) {};
          \node[c, complete] at (barycentric cs:v9=1,v13=1,v12=1) {};
          \node[c, complete] at (barycentric cs:v12=1,v13=1,v16=1) {};
          \node[c, complete] at (barycentric cs:v13=1,v16=1,v17=1) {};
          \node[c, complete] at (barycentric cs:v16=1,v17=1,v20=1) {};
          \node[e, complete] at (barycentric cs:v3=1,v4=1) {};
          \node[e, complete] at (barycentric cs:v4=1,v5=1) {};
          \node[e, complete] at (barycentric cs:v10=1,v14=1) {};
          \node[e, complete] at (barycentric cs:v18=1,v22=1) {};
          \node[e, complete] at (barycentric cs:v7=1,v11=1) {};
          \node[e, complete] at (barycentric cs:v19=1,v23=1) {};
          \node[e, complete] at (barycentric cs:v8=1,v12=1) {};
          \node[e, complete] at (barycentric cs:v16=1,v20=1) {};
          \node[e, complete] at (barycentric cs:v9=1,v13=1) {};
          \node[e, complete] at (barycentric cs:v13=1,v17=1) {};
          \node[e, complete] at (barycentric cs:v7=1,v8=1) {};
          \node[e, complete] at (barycentric cs:v8=1,v9=1) {};
          \node[e, complete] at (barycentric cs:v10=1,v11=1) {};
          \node[e, complete] at (barycentric cs:v12=1,v13=1) {};
          \node[e, complete] at (barycentric cs:v3=1,v14=1) {};
          \node[e, complete] at (barycentric cs:v16=1,v17=1) {};
          \node[e, complete] at (barycentric cs:v4=1,v18=1) {};
          \node[e, complete] at (barycentric cs:v19=1,v20=1) {};
          \node[e, complete] at (barycentric cs:v5=1,v22=1) {};
          \node[e, complete] at (barycentric cs:v22=1,v23=1) {};
          \node[e, complete] at (barycentric cs:v3=1,v10=1) {};
          \node[e, complete] at (barycentric cs:v4=1,v14=1) {};
          \node[e, complete] at (barycentric cs:v5=1,v18=1) {};
          \node[e, complete] at (barycentric cs:v10=1,v7=1) {};
          \node[e, complete] at (barycentric cs:v22=1,v19=1) {};
          \node[e, complete] at (barycentric cs:v11=1,v8=1) {};
          \node[e, complete] at (barycentric cs:v23=1,v20=1) {};
          \node[e, complete] at (barycentric cs:v12=1,v9=1) {};
          \node[e, complete] at (barycentric cs:v16=1,v13=1) {};
          \node[e, complete] at (barycentric cs:v20=1,v17=1) {};
        }
      \end{tikzpicture}
    \end{column}
    \hspace{-2em}
    \begin{column}{0.6\textwidth}
      {\small
        \begin{block}{Looping over vertices}
          \begin{itemize}
          \item<2-> Select mesh point
          \item<3-> Add points in star
          \item<4-> Add points in closure
          \item<5-> Complete with FEM adjacency
          \end{itemize}
        \end{block}
      }
    \end{column}
  \end{columns}
  \begin{block}<6->{Discretisation-independent}
    \begin{itemize}
    \item With points selected, \texttt{PetscSection} gives dofs
    \item Operators ``just do assembly'' on the patch
    \end{itemize}
  \end{block}
\end{frame}

\begin{frame}
  \frametitle{Another tool in the box}
  \begin{itemize}
  \item Requires slightly more setup than purely algebraic PCs
  \item Need to feed in operator callback, and some discretisation information
  \item I do this with the same Python interface as for the block PCs
  \item Opens up ability for ``monolithic'' multigrid in PETSc
  \end{itemize}

  Code available at \url{github.com/wence-/ssc}, hopefully in PETSc RSN.
\end{frame}

\begin{frame}[fragile]
  \frametitle{Example: P2-P1 Stokes}
  \begin{onlyenv}<1-2>
    \begin{onlyenv}<1>
      \begin{center}
        \includegraphics[height=0.8\textheight]{stokes-viscosity}
      \end{center}
    \end{onlyenv}
    \begin{onlyenv}<2>
    Monolithic multigrid with Vanka smoother on each level.
    \begin{columns}
      \begin{column}{0.6\textwidth}
\begin{minted}[fontsize=\tiny]{python3}
monolithic_solver_parameters = {
    "mat_type": "matfree",
    "snes_type": "ksponly",
    "ksp_rtol": 1e-8,
    "ksp_type": "fgmres",
    "pc_type": "mg",
    "mg_levels": {"ksp_type": "gmres",
        "ksp_max_it": 5,
        "pc_type": "python",
        "pc_python_type": "ssc.PatchPC",
        "patch_pc_patch_save_operators": True,
        "patch_pc_patch_construction_type": "vanka",
        "patch_pc_patch_partition_of_unity": True,
        "patch_pc_patch_vanka_dim": 0,
        "patch_pc_patch_construction_dim": 0,
        "patch_pc_patch_exclude_subspace": 1,
        "patch_pc_patch_sub_mat_type": "seqaij",
        "patch_sub_ksp_type": "preonly",
        "patch_sub_pc_type": "lu",
        "patch_sub_pc_factor_shift_type": "nonzero"},
    "mg_coarse": {"ksp_type": "preonly",
        "pc_type": "python",
        "pc_python_type": "firedrake.AssembledPC",
        "assembled_pc_type": "svd",}
}
\end{minted}
      \end{column}
      \hspace{-1em}
      \begin{column}{0.4\textwidth}
\begin{minted}[fontsize=\tiny]{md}
  0 SNES Function norm 51.1133
    Residual norms for  solve.
    0 KSP Residual norm 51.1133
    1 KSP Residual norm 4.24056
    2 KSP Residual norm 1.29486
    3 KSP Residual norm 0.207982
    4 KSP Residual norm 0.1554
    5 KSP Residual norm 0.0422543
    6 KSP Residual norm 0.0278166
    7 KSP Residual norm 0.00664682
    8 KSP Residual norm 0.00307886
    9 KSP Residual norm 0.000731788
   10 KSP Residual norm 0.000238784
   11 KSP Residual norm 3.09127e-05
   12 KSP Residual norm 4.9848e-06
   13 KSP Residual norm 1.07483e-06
   14 KSP Residual norm 1.00638e-07
  1 SNES Function norm 1.00638e-07
\end{minted}
      \end{column}
    \end{columns}
    \end{onlyenv}
  \end{onlyenv}
\end{frame}
\begin{frame}
  \frametitle{Conclusions}

  \begin{itemize}
  \item Composable solvers, using PDE library to easily
    develop complex block preconditioners.

  \item Model formulation decoupled from solver configuration.

  \item Automatically takes advantage of any improvements in both
    PETSc and Firedrake.

  \item Same approach works for Schwarz-like methods.
  \end{itemize}
  \begin{center}
    \url{www.firedrakeproject.org}
  \end{center}

  \begin{flushright}
    {\footnotesize
    \textcite{Kirby:2018} \arxivlink{1706.01346}{cs.MS}}
  \end{flushright}
  \begin{tikzpicture}[remember picture,overlay]
    \node[at=(current page.south west), anchor=south west] {\includegraphics[width=2.5cm]{epsrc-logo}};
  \end{tikzpicture}
\end{frame}
\appendix
\begin{frame}
  \frametitle{References}
  \printbibliography[heading=none]
\end{frame}
\end{document}
