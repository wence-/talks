\documentclass[presentation]{beamer}

\usepackage{tikz}
\usetikzlibrary{positioning,calc,matrix}
\usetikzlibrary{shapes.geometric}
\usetikzlibrary{arrows}
\usetikzlibrary{shapes}
\usetikzlibrary{backgrounds}% only to show the bounding box
\usepackage{pgf}
\usepackage{pgfplots}
\pgfplotsset{compat=1.13}
\usepackage{pgfplotstable}
\usepackage{appendixnumberbeamer}
\usepackage{amsmath}
\usepackage{pifont}
\newcommand{\cmark}{\ding{51}}
\newcommand{\xmark}{\ding{55}}
\DeclareMathOperator{\tr}{tr}
\DeclareMathOperator{\grad}{grad}
\let\div\relax
\DeclareMathOperator{\div}{div}
\DeclareMathOperator{\curl}{curl}
\date{25th January 2018}
\usetheme{metropolis}
\metroset{progressbar=frametitle}

\usepackage{algpseudocode}
\renewcommand{\vec}[1]{\ensuremath{\boldsymbol{#1}}}
\newcommand{\ddt}[1]{\frac{\partial #1}{\partial t}}
\newcommand{\zhat}{\hat{\vec{z}}}
\newcommand{\W}{\ensuremath{\mathbb{W}}}

\newcommand{\inner}[1]{\left\langle #1 \right \rangle}

\newcommand{\KSP}[2]{\ensuremath{\mathcal{K}\left(#1, \mathbb{#2}\right)}}
\newcommand{\ksp}[1]{\KSP{#1}{#1}}

\newcommand{\highlight}[1]{\colorbox{red!20}{\color{black} #1}}
\newcommand{\arxivlink}[2]{%
  \href{http://www.arxiv.org/abs/#1}%
  {\texttt{arXiv:\,#1\,[#2]}}%
}
\newcommand{\doilink}[1]{%
  \href{http://dx.doi.org/#1}%
  {\texttt{doi:\,#1}{}}%
}

\author{Lawrence Mitchell\inst{1,*} \\ {\scriptsize C.J.~Cotter,
    P.E.~Farrell, D.A.~Ham, M.~Homolya, P.H.J.~Kelly, R.C.~Kirby, F.~Wechsung \ldots}}
\institute{
\inst{1}Department of Computer Science, Durham University}
London

\inst{*}\texttt{lawrence.mitchell@durham.ac.uk}
}

\graphicspath{{./\jobname.figures/}{../pictures/}}

\usepackage[url=false,
            doi=true,
            isbn=false,
            style=authoryear,
            giveninits=true,
            uniquename=init,
            backend=biber]{biblatex}

\setbeamertemplate{bibliography item}{}

\renewcommand{\bibfont}{\fontsize{7}{7}\selectfont}
\addbibresource{references.bib}

\setlength{\bibitemsep}{1ex}
\setlength{\fboxsep}{1pt}

\renewbibmacro{in:}{}
\DeclareFieldFormat[article]{volume}{\textbf{#1}}
\DeclareFieldFormat{doi}{%
  doi\addcolon%
  {\scriptsize\ifhyperref{\href{http://dx.doi.org/#1}{\nolinkurl{#1}}}
    {\nolinkurl{#1}}}}
\AtEveryBibitem{%
\clearfield{pages}%
\clearfield{issue}%
\clearfield{number}%
}

\usepackage{minted}
\RecustomVerbatimEnvironment{Verbatim}{BVerbatim}{}


\title{From symbolic mathematics to fast solvers for finite element problems}

\begin{document}

\begin{abstract}
  One of the great and enduring successes of numerical computing is
  the capturing of mathematical abstractions in software.  This allows
  the programmer to express the intent of their code, without
  specifying its low-level implementation.  We then automate the
  synthesis of efficient code by using (or writing) compilers.

  In the context of solving numerical PDEs, the finite element method
  is particularly amenable to this approach.  For many problems, the
  choice of discretisation completely specifies the mathematical
  intent.  A symbolic description of the PDE can then be manipulated
  by a domain-specific compiler to produce a high-performance
  implementation.

  In this talk, I will present a concrete realisation of these ideas,
  embodied in the finite element software Firedrake
  (www.firedrakeproject.org).  I will discuss how capturing and
  exploiting symbolic structure in numerical software greatly
  simplifies model development, while simultaneously permitting the
  synthesis of a high performance implementation.

  I will then discuss recent work extending these same ideas to the
  development of preconditioners, illustrating with some recent work
  on scalable solvers for the three-dimensional stationary
  Navier-Stokes equations.
\end{abstract}

% People:
% Jerome Droniou
% Michael Page
% Hans de Sterck
% Philip Hall
% Steve Siems
% Yann Bernard
% Simon Clarke
% Julie Clutterbuck
% Anja Slim
% Kengo Deguchi
% Christian Thomas

\begin{frame}
  \frametitle{Outline}

  25 + 5 minutes

  - Finite elements
  - Code that matches maths
  - Compilers for this code
  - Cover sum factorisation
  - Hybridisation methods

  - Navier-Stokes
  - Augmented Lagrangian
  - Multigrid requires characterisation of kernel of div
  - New: extension of scheme to 3D
  - Full Newton
  - Scalability results
  - Future direction H(div)/Scott-Vogelius
\end{frame}
\end{document}
