\documentclass[presentation]{beamer}

\usepackage{tikz}
\usetikzlibrary{trees,calc}
\usepackage{booktabs}
\usepackage{pgf}
\usepackage{pgfplots}
\pgfplotsset{compat=1.13}
\usepackage{pgfplotstable}
\usepackage{appendixnumberbeamer}
\usepackage{amsmath}
\usepackage{pifont}
\newcommand{\cmark}{\ding{51}}
\newcommand{\xmark}{\ding{55}}
\DeclareMathOperator{\grad}{grad}
\DeclareMathOperator{\tr}{tr}
\let\div\relax
\DeclareMathOperator{\div}{div}
\DeclareMathOperator{\curl}{curl}
\date{25th January 2018}
\usetheme{metropolis}
\metroset{progressbar=frametitle}

\usepackage{algpseudocode}
\renewcommand{\vec}[1]{\ensuremath{\boldsymbol{#1}}}
\newcommand{\ddt}[1]{\frac{\partial #1}{\partial t}}
\newcommand{\zhat}{\hat{\vec{z}}}
\newcommand{\W}{\ensuremath{\mathbb{W}}}

\newcommand{\inner}[1]{\left\langle #1 \right \rangle}

\newcommand{\KSP}[2]{\ensuremath{\mathcal{K}\left(#1, \mathbb{#2}\right)}}
\newcommand{\ksp}[1]{\KSP{#1}{#1}}

\newcommand{\highlight}[1]{\colorbox{red!20}{\color{black} #1}}
\newcommand{\arxivlink}[2]{%
  \href{http://www.arxiv.org/abs/#1}%
  {\texttt{arXiv:\,#1\,[#2]}}%
}
\newcommand{\doilink}[1]{%
  \href{http://dx.doi.org/#1}%
  {\texttt{doi:\,#1}{}}%
}

\author{Lawrence Mitchell\inst{1,*} \\ {\scriptsize C.J.~Cotter,
    P.E.~Farrell, D.A.~Ham, M.~Homolya, P.H.J.~Kelly, R.C.~Kirby, F.~Wechsung \ldots}}
\institute{
\inst{1}Department of Computer Science, Durham University

\inst{*}\texttt{lawrence.mitchell@durham.ac.uk}
}

\graphicspath{{./\jobname.figures/}{../pictures/}}

\usepackage[url=false,
doi=true,
isbn=false,
style=authoryear,
giveninits=true,
uniquename=init,
backend=biber]{biblatex}

\usepackage{xspace}
\renewcommand{\Re}{\ensuremath{\mathrm{Re}}\xspace}
\newcommand{\honev}{\ensuremath{{H}^1(\Omega; \mathbb{R}^d)}\xspace}
\newcommand{\ltwov}{\ensuremath{{L}^2(\Omega; \mathbb{R}^d)}\xspace}
\newcommand{\ltwo}{\ensuremath{{L}^2(\Omega)}\xspace}
\newcommand{\laplace}{\ensuremath{\Delta}\,}
\newcommand{\hess}{\ensuremath{\mathbf{hessian}}\,}
\newcommand{\rt}{\ensuremath{\mathrm{RT}_0}\xspace}
\newcommand{\nd}{\ensuremath{\mathrm{ND}_0}\xspace}
\newcommand{\kerdiv}{\ensuremath{\mathrm{ker}(\mathbf{div})}\xspace}
\newcommand{\kercurl}{\ensuremath{\mathrm{ker}(\mathbf{curl})}\xspace}
\newcommand{\eker}{\ensuremath{e^{\mathrm{ker}}}\xspace}
\newcommand{\ds}{\ \mathrm{d}s}
\newcommand{\dx}{\ \mathrm{d}x}
\newcommand{\Pq}{\ensuremath{\mathrm{P}_{Q_h}}}
\newcommand{\PqK}{\ensuremath{\mathrm{P}_{Q_h(K)}}}
\newcommand{\Ptwo}{\ensuremath{\mathbb{P}_2}\xspace}
\newcommand{\Pthree}{\ensuremath{\mathbb{P}_3}\xspace}
\newcommand{\PtwoPzero}{\ensuremath{[\mathbb{P}_2]^2\mathrm{-}\mathbb{P}_0}\xspace}
\newcommand{\PthreePzero}{\ensuremath{[\mathbb{P}_3]^3\mathrm{-}\mathbb{P}_0}\xspace}
\newcommand{\Pzero}{\ensuremath{\mathbb{P}_0}\xspace}
\newcommand{\Pv}{\ensuremath{\mathbb{P}_v}\xspace}
\newcommand{\BR}{\ensuremath{\left(\mathbb{P}_1 \oplus B^F_3\right)}\xspace}
\newcommand{\PoneFB}{\ensuremath{\mathbb{P}_1 \oplus B^F_3}\xspace}
\newcommand{\PtwoFB}{\ensuremath{\mathbb{P}_2 \oplus B^F_3}\xspace}
\newcommand{\BRzero}{\ensuremath{\BR^3\mathrm{-}\mathbb{P}_0}\xspace}
\newcommand{\fmw}{\ensuremath{\left(\mathbb{P}_2 \oplus B^F_3\right)}\xspace}
\newcommand{\fmwzero}{\ensuremath{\fmw^3\mathrm{-}\mathbb{P}_0}\xspace}
%\newcommand{\advect}[2]{\ensuremath{(\nabla #1) \cdot #2}}
\newcommand{\advect}[2]{\ensuremath{(#2 \cdot \nabla) #1}}
\newcommand{\mesh}{\ensuremath{\mathcal{M}}\xspace}
\newcommand{\Ac}{\ensuremath{\mathcal{A}}}
\newcommand{\Bc}{\ensuremath{\mathcal{B}}}
\setbeamertemplate{bibliography item}{}

\renewcommand{\bibfont}{\fontsize{7}{7}\selectfont}
\addbibresource{references.bib}

\setlength{\bibitemsep}{1ex}
\setlength{\fboxsep}{1pt}

\renewbibmacro{in:}{}
\DeclareFieldFormat[article]{volume}{\textbf{#1}}
\DeclareFieldFormat{doi}{%
  doi\addcolon%
  {\scriptsize\ifhyperref{\href{http://dx.doi.org/#1}{\nolinkurl{#1}}}
    {\nolinkurl{#1}}}}
\AtEveryBibitem{%
\clearfield{pages}%
\clearfield{issue}%
\clearfield{number}%
}

\usepackage{minted}
\RecustomVerbatimEnvironment{Verbatim}{BVerbatim}{}


\title{From symbolic mathematics to fast solvers for finite element problems}

\begin{document}

\begin{abstract}
  One of the great and enduring successes of numerical computing is
  the capturing of mathematical abstractions in software.  This allows
  the programmer to express the intent of their code, without
  specifying its low-level implementation.  We then automate the
  synthesis of efficient code by using (or writing) compilers.

  In the context of solving numerical PDEs, the finite element method
  is particularly amenable to this approach.  For many problems, the
  choice of discretisation completely specifies the mathematical
  intent.  A symbolic description of the PDE can then be manipulated
  by a domain-specific compiler to produce a high-performance
  implementation.

  In this talk, I will present a concrete realisation of these ideas,
  embodied in the finite element software Firedrake
  (www.firedrakeproject.org).  I will discuss how capturing and
  exploiting symbolic structure in numerical software greatly
  simplifies model development, while simultaneously permitting the
  synthesis of a high performance implementation.

  I will then discuss recent work extending these same ideas to the
  development of preconditioners, illustrating with some recent work
  on scalable solvers for the three-dimensional stationary
  Navier-Stokes equations.
\end{abstract}

% People:
% Jerome Droniou
% Michael Page
% Hans de Sterck
% Philip Hall
% Steve Siems
% Yann Bernard
% Simon Clarke
% Julie Clutterbuck
% Anja Slim
% Kengo Deguchi
% Christian Thomas

\begin{frame}
  \frametitle{Outline}

  25 + 5 minutes

  - Finite elements
  - Code that matches maths
  - Compilers for this code
  - Cover sum factorisation
  - Hybridisation methods

  - Navier-Stokes
  - Augmented Lagrangian
  - Multigrid requires characterisation of kernel of div
  - New: extension of scheme to 3D
  - Full Newton
  - Scalability results
  - Future direction H(div)/Scott-Vogelius
\end{frame}

\begin{frame}
  \frametitle{Navier-Stokes}
Find $(u, p) \in \honev \times \ltwo$ such that
\begin{alignat*}{2}
  -  \nu \nabla^2 u + \advect{u}{u} + \nabla p &= f \quad && \text{ in } \Omega, \label{eqn:momentum} \\
  \nabla \cdot u &= 0 \quad && \text{ in } \Omega, \\
  u &= g \quad && \text{ on } \Gamma_D, \\
  \nu \nabla u \cdot n &= pn \quad && \text{ on } \Gamma_N,
\end{alignat*}
\end{frame}

\begin{frame}
  \frametitle{Linear saddle point system}
  After Newton linearization and choosing a spatial discretisation
\begin{equation} \label{eqn:sp}
\begin{pmatrix}
A & B^T \\
B & 0
\end{pmatrix}
\begin{pmatrix}
\delta u \\ \delta p
\end{pmatrix}
=
\begin{pmatrix}
b \\ 0
\end{pmatrix},
\end{equation}

Challenge: computing inverse.

Augmented Lagrangian scheme

\begin{equation} \label{eqn:spal}
\begin{pmatrix}
A + \gamma B^T M_p^{-1} B & B^T \\
B & 0
\end{pmatrix}
\begin{pmatrix}
\delta u \\ \delta p
\end{pmatrix}
=
\begin{pmatrix}
b \\ 0
\end{pmatrix},
\end{equation}

\begin{equation} \label{eqn:schur}
S^{-1} \approx (\nu + \gamma) M_p^{-1},
\end{equation}
Challenge: $\nu$ and $\gamma$ independent PC for $A + \gamma B^T
M_p^{-1} B$.  We take $\gamma$ large.
\end{frame}

\begin{frame}
  \frametitle{Multilevel solver}

  \resizebox{\textwidth}{!}{
  \begin{tikzpicture}[
  every node/.style={draw=black, thick, anchor=west},
  grow via three points={one child at (0.0,-0.7) and
  two children at (0.0,-0.7) and (0.0,-1.4)},
  edge from parent path={(\tikzparentnode.210) |- (\tikzchildnode.west)}]
  \node {Continuation}
    child { node {Newton solver with line search}
      child { node {Krylov solver (FGMRES)}
        child { node {Block preconditioner}
          child { node {Approximate Schur complement inverse}}
          child { node {F-cycle on augmented momentum block}
              child { node {Coarse grid solver}
                child { node {LU factorization on assembled matrix}}
              }
              child [missing] {}
              child { node {Prolongation operator}
                child { node {Local solves over coarse cells}}
              }
              child [missing] {}
              child { node {Relaxation}
                child { node {GMRES}
                  child { node {Matrix-free additive star iteration}}
                }
              }
          }
        }
      }
    };
  \end{tikzpicture}
  }
\end{frame}

\begin{frame}
  \frametitle{$A_\gamma$ preconditioner}
  Pieces:

  Smoother that is robust wrt $\nu$ and $\gamma$, requires subspace
  decomposition
  \begin{equation*}
    \mathcal{N} = \sum_i \left(V_i \cap \mathcal{N} \right)
  \end{equation*}
  \begin{equation*}
    \mathcal{N} := \{u \in V : (\Pq \nabla \cdot u, \nabla \cdot v) =
    0 \ \forall\ v \in V\}
  \end{equation*}
  \begin{equation*}
    V = \sum_i V_i
  \end{equation*}

  Kernel is spanned by basis functions with local support around each
  vertex.

  Relaxation scheme: \emph{star iteration} (Also used for robust
  multigrid in $H(\text{div}/\text{curl})$ (AFW 2000).
\end{frame}

\begin{frame}
  \frametitle{Prolongation}
  Need to preserve kernel of Pdiv in prolongation.

  Natural embedding does not do this.

  Solve local Stokes problem on each coarse cell to fix things up.

  P3-P0 expensive.

  P1+FB-P0: inf-sup (via BR)

  Now prolongation is not nested: preserve flux by scaling.
\end{frame}

\begin{frame}
  \frametitle{Software abstractions}
  Schwarz building blocks:

  Subspace decomp
  Operators on subspaces
  Solvers on subspaces
  Coarse spaces (not yet)

  Separate topological decomposition with callback interface for operators
\end{frame}
\begin{frame}
  \frametitle{Flexible selection of patches}
  Star patches animation

  Coarse cell patch animation.

  Operators just by doing finite element assembly
\end{frame}
\begin{frame}
  \frametitle{Results: 3D lid-driven cavity}
  \begin{onlyenv}<1>
    Reynolds number robust convergence.
    \begin{center}
      \begin{tabular}{cc|ccccc}
        \toprule
        \# refinements & \# dofs & \multicolumn{5}{c}{Reynolds number} \\
                       && 10 & 100 & 1000 & 2500 & 5000 \\
        \midrule
        1 & $2.1 \times 10^6$ & 7.50 & 7.33 & 7.50 & 7.00 & 6.50 \\
        2 & $1.7 \times 10^7$ & 8.50 & 7.00 & 7.50 & 6.50 & 5.50 \\
        3 & $1.3 \times 10^8$ & 7.00 & 7.00 & 6.50 & 5.00 & 6.50 \\
        4 & $1.1 \times 10^9$ & 7.00 & 7.33 & 5.50 & 4.00 & 9.00 \\
        \bottomrule
      \end{tabular}
    \end{center}
  \end{onlyenv}
  \begin{onlyenv}<2>
    
    \begin{center}
      \pgfplotstableread[col sep=comma, row sep=\\]{%
        Cores,Time,Dofs\\
        48,1.91e2,2134839\\
        384,2.52e2,16936779\\
        3072,2.3e2,134930451\\
        24576,2.55e2,1077196323\\
      }\datatable
      \begin{tikzpicture}[scale=0.8]
        \begin{semilogxaxis}[
          log basis x=2,
          ymin=0,
          xtick=data,
          xticklabels from table={\datatable}{Cores},
          extra x ticks={48, 384, 3072, 24576},
          extra x tick labels={$[2.13]$, $[16.9]$,$[135]$,$[1077]$},
          extra x tick style={tick label style={yshift=-2ex}},
          xlabel={Cores\\{}[DoFs $\times 10^6$]},
          xlabel style={align=center},
          ylabel near ticks,
          ylabel style={align=center, text width=4cm},
          ylabel={Time to solution over all continuation steps [min]},
          title style={align=center, text width=7cm},
          ]
          \addplot+ table[x=Cores,y=Time] {\datatable};
        \end{semilogxaxis}
      \end{tikzpicture}
    \end{center}
  \end{onlyenv}
\end{frame}

\begin{frame}
  \frametitle{Future directions}

  This problem:
  Better discretisation

  Options: Scott-Vogelius

  H(div)-L2 with penalty scheme for diffusion operator (potential for
  non-nested mesh hierarchies)

  Technology:
  Nonlinear schwarz; FAS smoothers.
\end{frame}
\end{document}
