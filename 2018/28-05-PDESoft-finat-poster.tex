\documentclass{article}
\usepackage[utf8]{inputenc}
\usepackage{amsmath,amsfonts,amssymb}
\usepackage{graphicx}
\usepackage{times}
\parindent0pt
\parskip 1.5ex plus 1ex minus .5ex
%%%%%%%%%%%%%%%%%%%%%%%%%%%%%%%%%%%%%%%%%%%%%%%%%%%%%%%%%%%%%%%%%%%%%%
% Do not change above this line! Do not add packages! You may comment
% out packages missing in your installation.
%
% Everything before "\begin{document}" will be ignored! Do not define
%   macros!
%%%%%%%%%%%%%%%%%%%%%%%%%%%%%%%%%%%%%%%%%%%%%%%%%%%%%%%%%%%%%%%%%%%%%%
\begin{document}
\title{FInAT: exploiting structure in code generation for high-order finite elements}
\author{%
  Mikl\'os Homolya\thanks{Department of Computing, Imperial College
    London \texttt{m.homolya14@imperial.ac.uk}}
  \and
  Robert C.~Kirby\thanks{Department of Mathematics, Baylor University,
    \texttt{robert\_kirby@baylor.edu}}
  \and
  David A.~Ham\thanks{Department of Mathematics, Imperial College
    London \texttt{david.ham@imperial.ac.uk}}
  \and
  Lawrence Mitchell\thanks{Department of Computing and Department of
    Mathematics, Imperial College London
    \texttt{lawrence.mitchell@imperial.ac.uk}}
}
\maketitle

{\Large This is a submission for poster presentation only (rather than
  a talk).}

Code generation is an increasingly popular technique for the
development of complicated finite element discretisations of partial
differential equations.  In this poster, we show new developments in
the Firedrake finite element framework that allow the code generation
pipeline to exploit structure inherent to some finite elements.  This
includes, in particular, sum factorisation on tensor product cells for
the full family of $Q^-$ spaces.

This is achieved by providing a more expressive interface between the
form compiler and the library providing the implementation of finite
element tabulation.  This is encapsulated in a new finite element
library, FInAT, that explicitly communicates the structure of
elements.  New form compiler algorithms are introduced that exploit
this structure.


\end{document}

%%% Local Variables: 
%%% mode: latex
%%% TeX-master: t
%%% End: 
