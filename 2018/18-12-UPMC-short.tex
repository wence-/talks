\documentclass[presentation, 10pt]{beamer}

\usepackage{booktabs}
\usepackage{tikz}
\usetikzlibrary{trees,calc,positioning}
\usetikzlibrary{shapes, shapes.geometric}
% https://tex.stackexchange.com/questions/27171/padded-boundary-of-convex-hull
\newcommand{\convexpath}[2]{
  [
  create hullcoords/.code={
    \global\edef\namelist{#1}
    \foreach [count=\counter] \nodename in \namelist {
      \global\edef\numberofnodes{\counter}
      \coordinate (hullcoord\counter) at (\nodename);
    }
    \coordinate (hullcoord0) at (hullcoord\numberofnodes);
    \pgfmathtruncatemacro\lastnumber{\numberofnodes+1}
    \coordinate (hullcoord\lastnumber) at (hullcoord1);
  },
  create hullcoords
  ]
  ($(hullcoord1)!#2!-90:(hullcoord0)$)
  \foreach [
  evaluate=\currentnode as \previousnode using \currentnode-1,
  evaluate=\currentnode as \nextnode using \currentnode+1
  ] \currentnode in {1,...,\numberofnodes} {
    let \p1 = ($(hullcoord\currentnode) - (hullcoord\previousnode)$),
    \n1 = {atan2(\y1,\x1) + 90},
    \p2 = ($(hullcoord\nextnode) - (hullcoord\currentnode)$),
    \n2 = {atan2(\y2,\x2) + 90},
    \n{delta} = {Mod(\n2-\n1,360) - 360}
    in
    {arc [start angle=\n1, delta angle=\n{delta}, radius=#2]}
    -- ($(hullcoord\nextnode)!#2!-90:(hullcoord\currentnode)$)
  }
}
\usepackage{booktabs}
\usepackage{pgf}
\usepackage{pgfplots}
\pgfplotsset{compat=1.16}
\usepackage{pgfplotstable}
\usepackage{appendixnumberbeamer}
\usepackage{amsmath}
\usepackage{pifont}
\newcommand{\cmark}{\ding{51}}
\newcommand{\xmark}{\ding{55}}
\DeclareMathOperator{\grad}{grad}
\DeclareMathOperator{\tr}{tr}
\let\div\relax
\DeclareMathOperator{\div}{div}
\DeclareMathOperator{\curl}{curl}
\DeclareMathOperator*{\argmin}{arg\,min}
\date{December 18, 2018}
\usetheme{metropolis}
\metroset{progressbar=frametitle}

\usepackage{algpseudocode}
\renewcommand{\vec}[1]{\ensuremath{\boldsymbol{#1}}}
\newcommand{\ddt}[1]{\frac{\partial #1}{\partial t}}
\newcommand{\zhat}{\hat{\vec{z}}}
\newcommand{\W}{\ensuremath{\mathbb{W}}}

\newcommand{\inner}[1]{\left\langle #1 \right \rangle}

\newcommand{\KSP}[2]{\ensuremath{\mathcal{K}\left(#1, \mathbb{#2}\right)}}
\newcommand{\ksp}[1]{\KSP{#1}{#1}}

\newcommand{\highlight}[1]{\colorbox{red!20}{\color{black} #1}}
\newcommand{\arxivlink}[2]{%
  \href{https://www.arxiv.org/abs/#1}%
  {\texttt{arXiv:\,#1\,[#2]}}%
}
\newcommand{\doilink}[1]{%
  \href{https://dx.doi.org/#1}%
  {\texttt{doi:\,#1}{}}%
}

\author{Lawrence Mitchell\inst{1,*} \and Patrick E.~Farrell\inst{2}
  \and Florian Wechsung\inst{2}}
\institute{
\inst{1}Department of Computer Science, Durham University

\inst{*}\texttt{lawrence.mitchell@durham.ac.uk}

\inst{2}Mathematical Institute, University of Oxford
}

\graphicspath{{./\jobname.figures/}{../pictures/}}

\usepackage[url=false,
doi=true,
isbn=false,
style=authoryear,
maxnames=5,
giveninits=true,
uniquename=init,
backend=biber]{biblatex}

\usepackage{xspace}
\let\Re\relax
\DeclareMathOperator{\Re}{Re}
\newcommand{\honev}{\ensuremath{{H}^1(\Omega; \mathbb{R}^d)}\xspace}
\newcommand{\ltwov}{\ensuremath{{L}^2(\Omega; \mathbb{R}^d)}\xspace}
\newcommand{\ltwo}{\ensuremath{{L}^2(\Omega)}\xspace}
\newcommand{\laplace}{\ensuremath{\Delta}\,}
\newcommand{\rt}{\ensuremath{\mathrm{RT}_0}\xspace}
\newcommand{\nd}{\ensuremath{\mathrm{ND}_0}\xspace}
% \let\ker\relax
% \DeclareMathOperator{\ker}{ker}
\newcommand{\kerdiv}{\ker\div}
\newcommand{\kercurl}{\ker\curl}
\newcommand{\eker}{\ensuremath{e^{\ker}}\xspace}
\newcommand{\ds}{\ \text{d}s}
\newcommand{\dx}{\ \text{d}x}
\newcommand{\Pq}{\ensuremath{\mathrm{P}_{Q_h}}}
\newcommand{\PqK}{\ensuremath{\mathrm{P}_{Q_h(K)}}}
\newcommand{\Ptwo}{\ensuremath{\mathbb{P}_2}\xspace}
\newcommand{\Pthree}{\ensuremath{\mathbb{P}_3}\xspace}
\newcommand{\PtwoPzero}{\ensuremath{[\mathbb{P}_2]^2\mathrm{-}\mathbb{P}_0}\xspace}
\newcommand{\PtwothreePzero}{\ensuremath{[\mathbb{P}_2]^3\mathrm{-}\mathbb{P}_0}\xspace}
\newcommand{\PthreePzero}{\ensuremath{[\mathbb{P}_3]^3\mathrm{-}\mathbb{P}_0}\xspace}
\newcommand{\Pzero}{\ensuremath{\mathbb{P}_0}\xspace}
\newcommand{\Pv}{\ensuremath{\mathbb{P}_v}\xspace}
\newcommand{\BR}{\ensuremath{\left[\mathbb{P}_1 \oplus B^F_3\right]}\xspace}
\newcommand{\PoneFB}{\ensuremath{\mathbb{P}_1 \oplus B^F_3}\xspace}
\newcommand{\PtwoFB}{\ensuremath{\mathbb{P}_2 \oplus B^F_3}\xspace}
\newcommand{\BRzero}{\ensuremath{\BR^3\mathrm{-}\mathbb{P}_0}\xspace}
\newcommand{\fmw}{\ensuremath{\left(\mathbb{P}_2 \oplus B^F_3\right)}\xspace}
\newcommand{\fmwzero}{\ensuremath{\fmw^3\mathrm{-}\mathbb{P}_0}\xspace}
%\newcommand{\advect}[2]{\ensuremath{(\nabla #1) \cdot #2}}
\newcommand{\advect}[2]{\ensuremath{(#2 \cdot \nabla) #1}}
\newcommand{\mesh}{\ensuremath{\mathcal{M}}\xspace}
\newcommand{\Ac}{\ensuremath{\mathcal{A}}}
\newcommand{\Bc}{\ensuremath{\mathcal{B}}}
\setbeamertemplate{bibliography item}{}

\titlegraphic{\hfill\includegraphics[height=0.8cm]{durham-logo}}

\renewcommand{\bibfont}{\fontsize{7}{7}\selectfont}
\addbibresource{references.bib}

\setlength{\bibitemsep}{1ex}
\setlength{\fboxsep}{1pt}

\renewbibmacro{in:}{}
\DeclareFieldFormat[article]{volume}{\textbf{#1}}
\DeclareFieldFormat{doi}{%
  doi\addcolon%
  {\scriptsize\ifhyperref{\href{http://dx.doi.org/#1}{\nolinkurl{#1}}}
    {\nolinkurl{#1}}}}
\AtEveryBibitem{%
\clearfield{pages}%
\clearfield{issue}%
\clearfield{number}%
}

\usepackage{minted}
\RecustomVerbatimEnvironment{Verbatim}{BVerbatim}{}


\title{Reynolds-robust preconditioning for Navier--Stokes}

\begin{document}

\maketitle

\section{Firedrake overview}

\begin{frame}[fragile]
  \frametitle{Code that looks like maths}
  \begin{columns}
    \begin{column}{0.47\framewidth}
      {\footnotesize
        Find $(u, p, T) \in V\times W\times Q$ s.t.
        \begin{align*}
          \int\!\nabla u \cdot \nabla v + (u \cdot \nabla u) \cdot v \\
          - p\nabla\cdot v + \frac{\text{Ra}}{\text{Pr}} Tg \hat{z} \cdot v\,\text{d}x &= 0 \\
          \int\!\nabla\cdot u q\,\text{d}x&= 0\\
          \int\! (u\cdot \nabla T) S + \text{Pr}^{-1} \nabla T \cdot \nabla
          S\,\text{d}x &= 0\\
          \quad \forall\, (v,q,T) \in V\times W \times Q
        \end{align*}
        }
    \end{column}
      \begin{column}{0.52\framewidth}
\begin{minted}[fontsize=\tiny]{python}
from firedrake import *
mesh = Mesh(...)
V = VectorFunctionSpace(mesh, "CG", 2)
W = FunctionSpace(mesh, "CG", 1)
Q = FunctionSpace(mesh, "CG", 1)
Z = V * W * Q
Ra = Constant(...)
Pr = Constant(...)
upT = Function(Z)
u, p, T = split(upT)
v, q, S = TestFunctions(Z)
bcs = [...]

F = (inner(grad(u), grad(v))
     + inner(dot(grad(u), u), v)
     - inner(p, div(v))
     + (Ra/Pr)*inner(T*g, v)
     + inner(div(u), q)
     + inner(dot(grad(T), u), S)
     + (1/Pr) * inner(grad(T), grad(S)))*dx

solve(F == 0, upT, bcs=bcs)
\end{minted}
      \end{column}
  \end{columns}
\end{frame}

\begin{frame}[t]
  \frametitle{Code that looks like maths}
  \begin{itemize}
  \item Mathematics just says ``here is the integral to compute on each
    element, do that everywhere''
  \item Computer implementation chooses how to do so
  \end{itemize}
  \begin{block}{Assertion(s)}
    \vspace{0.25\baselineskip}
    Having chosen a discretisation, writing the element integral is ``mechanical''.

    With an element integral in hand, integrating over a mesh is
    ``mechanical''.
  \end{block}

  \begin{corollary}
    \vspace{0.25\baselineskip}
    Computers are good at mechanical things, why don't we get the
    computer to write them for us?
  \end{corollary}
\end{frame}

\begin{frame}[t]
  \frametitle{Firedrake \url{www.firedrakeproject.org}}

  \begin{overlayarea}{\textwidth}{\textheight}
    \begin{onlyenv}<1>
  \begin{columns}
    \begin{column}{0.8\textwidth}
      \begin{quote}
        {\normalfont [\ldots]} an automated system for the solution of
        partial differential equations using the finite element
        method.
      \end{quote}
    \end{column}
    \begin{column}{0.2\textwidth}
      \includegraphics[width=0.8\textwidth]{firedrake-small}
    \end{column}
  \end{columns}
      \begin{itemize}
      \item Written in Python.
      \item Finite element problems specified with \emph{embedded}
        domain specific language, UFL \parencite{Alnaes:2014} from the
        FEniCS project.
      \item \emph{Runtime} compilation to optimised, low-level (C)
        code.
      \item PETSc for meshes and (algebraic) solvers.
      \end{itemize}

    \begin{flushright}
      {\scriptsize F.~Rathgeber, D.A.~Ham, \textbf{LM}, M.~Lange,
        F.~Luporini, A.T.T.~McRae, G.-T.~Bercea, G.R.~Markall,
        P.H.J.~Kelly. ACM Transactions on Mathematical Software,
        2016. \arxivlink{1501.01809}{cs.MS}\nocite{Rathgeber:2016}}
    \end{flushright}
  \end{onlyenv}
  \begin{onlyenv}<2>
    \begin{block}{Particular strengths}
      \begin{itemize}
      \item Geophysical fluid dynamics (one structured dimension)
        
      \item Extensible interface for multigrid and block
        preconditioning
      \item All\textsuperscript{(*)} the finite elements
      \end{itemize}

      \textsuperscript{(*)} Not actually all of them
    \end{block}
    \begin{block}{User groups}
      \begin{itemize}
      \item Imperial, Oxford, Bath, Leeds, Durham, Kiel, Rice,
        Houston, Exeter, Buffalo, Waterloo, Minnesota, Baylor, Texas
        A\&M, \dots
      \item Annual user \& developer meeting this year had 45
        attendees.
      \item Next meeting (2019) probably in Durham in September.
      \end{itemize}
    \end{block}
  \end{onlyenv}
\end{overlayarea}
\end{frame}


\begin{frame}[fragile]
  \frametitle{Automation makes your code short}
  \begin{columns}
    \begin{column}{0.45\textwidth}
      \begin{equation*}
        a(u, v) = \int_\Omega \nabla u \cdot \nabla v\,\text{d}x \quad \forall v \in V
      \end{equation*}
\begin{minted}[fontsize=\scriptsize]{python}
V = FiniteElement("Lagrange",
                  triangle, 1)
u = TrialFunction(V)
v = TestFunction(V)
F = dot(grad(u), grad(v))*dx
\end{minted}
    \end{column}
    \hspace{0.05\textwidth}
    \begin{column}{0.45\textwidth}
      \begin{tikzpicture}
        \node[draw,rectangle, thick, text width=1in, align=center] (A)
        {\small Pen and paper};

        \node[draw, below=of A, rectangle, thick, text width=1in,
        align=center] (B) {\small High-level code};

        \node[draw, below=of B, rectangle, thick, text width=1in,
        align=center] (C) {\small Low-level code};

        \node[draw, below=of C, rectangle, thick, text width=1in,
        align=center] (D) {\small Machine code};

        \draw[-stealth] (A) -> (B) node [midway,right,text
        width=1in,align=center] {\small Manual};

        \draw[-stealth] (B) -> (C) node [midway, right, text
        width=1in, align=center]
        {\small Automated};

        \draw[-stealth] (C) -> (D) node [midway, right, text
        width=1.5in, align=center] {\small Automated (gcc)};
      \end{tikzpicture}
    \end{column}
  \end{columns}
\end{frame}

\begin{frame}[fragile, t]
  \frametitle{Automation enables optimisations}
  Compile UFL (symbolics) into low-level code (implementation).
  \vspace{-\baselineskip}
  \begin{flushright}
    {\scriptsize M.~Homolya, \textbf{LM}, F.~Luporini, D.A.~Ham. SIAM
      SISC (2018). \arxivlink{1705.03667}{cs.MS}\nocite{Homolya:2018}}
  \end{flushright}
  \vspace{-0.5\baselineskip}
  \begin{itemize}
  \item Element integral
    \begin{columns}
      \begin{column}{0.4\textwidth}
        \begin{equation*}
          \int_e \nabla u \cdot \nabla v\,\text{d}x
        \end{equation*}
      \end{column}
      \hspace{-3em}
      \begin{column}{0.6\textwidth}
\begin{minted}[fontsize=\scriptsize]{python}
V = FiniteElement("Lagrange", triangle, 1)
u = TrialFunction(V)
v = TestFunction(V)
a = dot(grad(u), grad(v))*dx
\end{minted}
      \end{column}
    \end{columns}
    \hspace{0.5\baselineskip}
  \item Is transformed to a tensor algebra expression
    {\small \begin{equation*}
        \sum_q w_q \left| d \right| \sum_{i_5} \left( \sum_{i_3}
          K_{i_3,i_5} \begin{bmatrix}
            E^{(1)}_{q,k} & E^{(2)}_{q,k}
          \end{bmatrix}_{i_3} \right)
        \left( \sum_{i_4} K_{i_4,i_5} \begin{bmatrix}
            E^{(1)}_{q,j} & E^{(2)}_{q,j}
          \end{bmatrix}_{i_4} \right)
      \end{equation*}}
  \item Multiple optimisation passes aim to minimise FLOPs required to
    evaluate this expression.
  \end{itemize}
\end{frame}

\begin{frame}[t]
  \frametitle{Optimisation passes}
  \begin{block}{Sum factorisation}
    \vspace{0.25\baselineskip}
    Spectral elements typically use \emph{tensor product} basis functions
    \begin{equation*}
      \phi_{i,q} := \phi_{(j,k),(p,r)} = \varphi_{j,p}\varphi_{k,r}
    \end{equation*}
    Tensor algebra maintains this structure, compiler exploits it for
    low-complexity evaluation of integrals.
    \vspace{-\baselineskip}
    \begin{flushright}
      {\scriptsize M.~Homolya, R.C.~Kirby, D.A.~Ham, \arxivlink{1711.02473}{cs.MS}\nocite{Homolya:2017a}}
    \end{flushright}
  \end{block}
  \vspace{-0.5\baselineskip}
  \begin{block}{Loop transformations}
    \vspace{0.25\baselineskip}
    Solve ILP problem to reorder loops, and decide how to split
    complicated expressions.\nocite{Luporini:2017}
  \end{block}
  \begin{block}{Vectorisation}
    \vspace{0.25\baselineskip}
    Low-level rearrangement to ease the job of C compilers in
    generating efficient code.\nocite{Luporini:2015}
  \end{block}

\end{frame}

\section{Preconditioning}
\begin{frame}[t]
  \frametitle{The problem}
  \begin{block}{Stationary incompressible Navier--Stokes}
    \vspace{0.25\baselineskip}
    Find $(u, p) \in \honev \times \ltwo$ such that
    \begin{alignat*}{2}
      -  \Re^{-1} \nabla^2 u + \advect{u}{u} + \nabla p &= f \quad && \text{ in } \Omega, \\
      \nabla \cdot u &= 0 \quad && \text{ in } \Omega, \\
      u &= g \quad && \text{ on } \Gamma_D, \\
      \Re^{-1} \nabla u \cdot n &= pn \quad && \text{ on } \Gamma_N,
    \end{alignat*}

    For $\Re \gg 1$ Navier--Stokes admits multiple solutions. Can we find them?
  \end{block}
  
\end{frame}

\begin{frame}[t]
  \frametitle{Solving for the Newton step}
  \begin{block}{Newton linearisation}
    \begin{alignat*}{2}
      - \Re^{-1} \nabla^2 u + (u \cdot \nabla) w + (w \cdot \nabla) u) +
      \nabla p &= f \quad && \text{ in } \Omega, \\
      \nabla \cdot u &= \quad && \text { in } \Omega.\\
    \end{alignat*}
  \end{block}
  \vspace{-2\baselineskip}
  \begin{columns}[t]
    \begin{column}{0.5\textwidth}
      \begin{block}{LU factorisation}
        \begin{itemize}
        \item[\cmark] Scales well with $\Re \to \infty$
        \item[\xmark] Scales poorly with dof count
        \end{itemize}
      \end{block}
    \begin{alertblock}{This work}
      \begin{itemize}
      \item[\cmark] Scales well with $\Re \to \infty$
      \item[\cmark] Scales well with dof count
      \end{itemize}
    \end{alertblock}
    \end{column}
    \begin{column}{0.5\textwidth}
      \begin{block}{Existing preconditioners}
        \begin{itemize}
        \item[\xmark] Scale badly with $\Re \to \infty$
        \item[\cmark] Scale well with dof count
        \end{itemize}
      \end{block}
    \end{column}
  \end{columns}
\end{frame}

\begin{frame}[t]
  \frametitle{Backing off to Stokes}

  \begin{block}{Stokes' equations}
    \vspace{0.25\baselineskip}
    Find $(u, p) \in \honev \times \ltwo$ such that
    \begin{alignat*}{2}
      -  \Re^{-1} \nabla^2 u + \nabla p &= f \quad && \text{ in } \Omega, \\
      \nabla \cdot u &= 0 \quad && \text{ in } \Omega, \\
    \end{alignat*}
  \end{block}
  \begin{block}{Discretisation}
    \vspace{0.25\baselineskip}
    Discretising with inf-sup stable element pair results in:
    \begin{equation*}
      Jx := \begin{pmatrix}
        A & B^T \\
        B & 0
      \end{pmatrix}
      \begin{pmatrix}
        u \\ p
      \end{pmatrix}
      =
      \begin{pmatrix}
        b \\ 0
      \end{pmatrix}.
    \end{equation*}
  \end{block}
\end{frame}

\begin{frame}[t]
  \frametitle{Block preconditioning}
  \begin{block}{Block factorisations [Murphy, Golub, Wathen (2000)]}
    \vspace{0.25\baselineskip}
    Build preconditioners based on
    \begin{equation*}
      J^{-1} =
      \begin{pmatrix}
        I   & -A^{-1} B^T \\
        0 & I \\
      \end{pmatrix}
      \begin{pmatrix}
        A^{-1}  & 0 \\
        0 & S^{-1} \\
      \end{pmatrix}
      \begin{pmatrix}
        I   & 0 \\
        -BA^{-1} & I \\
      \end{pmatrix}
    \end{equation*}
    Where $S$ is the (dense) Schur complement
    \begin{equation*}
      S = - B A^{-1} B^T
    \end{equation*}
  \end{block}
  \begin{block}{PDE-specific challenges}
    \vspace{0.25\baselineskip}
    Find fast approximations $\tilde{A}^{-1}$ and $\tilde{S}^{-1}$ to
    $A^{-1}$ and $S^{-1}$.
  \end{block}
\end{frame}

\begin{frame}[t]
  \frametitle{Choosing $\tilde{A}^{-1}$ and $\tilde{S}^{-1}$}

  \begin{block}{Stokes [Silvester \& Wathen (1994)]}
    \vspace{0.25\baselineskip}
    Multigrid for $\tilde{A}^{-1}$ and choose $\tilde{S}^{-1} = -Q^{-1}$ ($Q$ the pressure mass matrix).
  \end{block}
  \begin{alertblock}{Bad news}
    \vspace{0.25\baselineskip}
    For Navier--Stokes, choosing $\tilde{S}^{-1} = -Q^{-1}$ degrades like $\mathcal{O}(\Re^2)$
  \end{alertblock}
  \begin{block}{PCD for Navier--Stokes}
    \vspace{0.25\baselineskip}
    Approximate $S^{-1}$ with convection-diffusion solves on the
    pressure space. Performs better than $-Q^{-1}$, but \dots
  \end{block}
\end{frame}

\begin{frame}[t]
  \frametitle{Performance of PCD approximation}

  Setup: v3.5 of IFISS, regularised 2D lid-driven cavity, $Q_1-P_0$
  element pair.

  \begin{table}[htbp]
    \centering
    \begin{tabular}{cc|c@{\hspace{9pt}}c@{\hspace{9pt}}c@{\hspace{9pt}}c@{\hspace{9pt}}c}
      \toprule
      $1/h$ & \# dofs            & \multicolumn{5}{c}{Reynolds number}                 \\
            &                    & 10   & 100  & 1000  & 5000          & 10000         \\
      \midrule
      $2^4$ & $8.34 \times 10^2$ & 22.0 & 40.4 & 103.3 & \texttt{NaNF} & -             \\
      $2^5$ & $3.20 \times 10^3$ & 23.0 & 41.3 & 137.7 & \texttt{NaNF} & -             \\
      $2^6$ & $1.25 \times 10^4$ & 24.5 & 42.0 & 157.0 & 386.0         & \texttt{NaNF} \\
      $2^7$ & $4.97 \times 10^4$ & 25.5 & 42.7 & 149.0 & 482.0         & 747.5         \\
      $2^8$ & $1.98 \times 10^5$ & 26.0 & 44.0 & 137.0 & 492.5         & 852.5         \\
      \bottomrule
    \end{tabular}
    \caption{Number of outer Krylov iterations per Newton step with PCD}
  \end{table}
\end{frame}

\begin{frame}[t]
  \frametitle{Augmented Lagrangian preconditioners}
  \vspace{2\baselineskip}
  \begin{quote}
    We describe an effective solver for the discrete Oseen problem
    [\dots] this approach results in fast convergence, independent of
    the mesh size and largely insensitive to the viscosity.

    \begin{flushright}
      {\normalfont [Benzi \& Olshanksii (2006)]}
    \end{flushright}
  \end{quote}

  \vspace{2\baselineskip}
  \begin{block}{Observation}
    \begin{itemize}
    \item[\cmark] Reynolds-robust preconditioning
    \item[\xmark] No-one else appears to have implemented the full scheme
      (2006--2018)
    \end{itemize}
  \end{block}
\end{frame}

\begin{frame}[t]
  \frametitle{Objectives}
  \begin{itemize}
  \item Can we make the first general implementation of the method? \cmark
  \item Can we extend the solver and discretisation to three
    dimensions? \cmark
  \item Can we extend the solver to a discretisation that enforces
    \begin{equation*}
      \nabla \cdot u = 0
    \end{equation*}
    pointwise? Crucial for velocity error estimates at high $\Re$. \textcolor{gray}{\cmark}
  \end{itemize}
  \uncover<2>{
    \begin{block}{Three ideas}
      \begin{enumerate}
      \item Control Schur complement with an augmented Lagrangian term
      \item Kernel-capturing multigrid relaxation
      \item Robust multigrid prolongation
      \end{enumerate}
    \end{block}
    \vspace{-\baselineskip}
    \begin{flushright}
      {\scriptsize P.E.~Farrell, \textbf{LM}, F.~Wechsung \arxivlink{1810.03315}{math.NA}\nocite{Farrell:2018}}
    \end{flushright}
  }
\end{frame}

\begin{frame}[t]
  \frametitle{Idea 1: controlling the Schur complement}

  Add an \emph{augmented Lagrangian} term.

  \begin{block}{Stokes again}
    \begin{equation*}
      u = \argmin_{v \in \honev} \frac{1}{2} \int \nabla : \nabla \dx
      - \int f \cdot v \dx \uncover<2->{+ \frac{\gamma}{2} \int (\nabla \cdot v)^2 \dx}
    \end{equation*}
    Subject to $\nabla \cdot v = 0$
  \end{block}

  \pause Doesn't change solution, since $\nabla \cdot v = 0$
  \pause
  \begin{theorem}[Hestenes, Fortin, Glowinski, Olshanksii, \dots]
    \vspace{0.25\baselineskip}
    As $\gamma \to \infty$, the Schur complement is well
    approximated by $S^{-1} \sim -(1 + \gamma)Q^{-1}$.
  \end{theorem}
\end{frame}

\begin{frame}[t]
  \frametitle{\dots or discretely}
  \begin{block}{Discrete augmented Lagrangian stabilisation}
    \vspace{0.25\baselineskip}
    \begin{equation*}
      \begin{pmatrix}
        A + \gamma B^T Q^{-1} B & B^T \\
        B & 0
      \end{pmatrix}
      \begin{pmatrix}
        u \\ p
      \end{pmatrix}
      =
      \begin{pmatrix}
        b \\ 0
      \end{pmatrix}
    \end{equation*}
  \end{block}

  Doesn't change solution, since $B u = 0$
  \begin{block}{Same result}
    \vspace{0.25\baselineskip}
    As $\gamma \to \infty$, the Schur complement is well approximated
    by $S^{-1} \sim -(1 + \gamma)Q^{-1}$.
  \end{block}

  \pause

  \begin{alertblock}{Not just for Stokes}
    \vspace{0.25\baselineskip}
    Crucially, this property \emph{still} holds for Navier--Stokes.
  \end{alertblock}
\end{frame}

\begin{frame}[t]
  \frametitle{Conservation of misery}
  Augmented Lagrangian scheme moves difficulty from $\tilde{S}^{-1}$
  to $\tilde{A}^{-1}$.

  We now must solve either: find $u \in V \subseteq \honev$ such that
  \begin{block}{Continuous stabilisation}
    \begin{equation*}
      \Re^{-1} (\nabla u, \nabla v) + ((w \cdot \nabla)u, v) + ((u \cdot
      \nabla)w, v) + \gamma(\nabla \cdot u, \nabla \cdot v) = (f, v)
    \end{equation*}
  \end{block}
  \vspace{-0.5\baselineskip}
  or 
  \begin{block}{Discrete stabilisation}
    \begin{equation*}
    \Re^{-1} (\nabla u, \nabla v) + ((w \cdot \nabla)u, v) + ((u \cdot \nabla)w, v) + \gamma(\overline{\nabla \cdot u}, \overline{\nabla \cdot v}) = (f, v)
  \end{equation*}
  \end{block}
  \vspace{-0.5\baselineskip}
  for all $v \in V$, where $\overline{x}$ is the $L^2$ projection onto the discrete pressure space.
\end{frame}

\begin{frame}[t]
  \frametitle{Idea 2: kernel-capturing multigrid relaxation}
  Consider the problem: for $\alpha, \beta \in \mathbb{R}$, find $u \in V$ such that
  \begin{equation*}
    \alpha a(u, v) + \beta b(u, v) = (f, v) \quad \forall v \in V,
  \end{equation*}
  where $a$ is SPD, and $b$ is symmetric positive semidefinite.

  \begin{block}{Relaxation method}
    \vspace{0.25\baselineskip}
    Choose a subspace decomposition
    \begin{equation*}
      V = \sum_i V_i,
    \end{equation*}
    solve the problem on each subspace and combine the updates.
  \end{block}
  \begin{exampleblock}{Example}
    \vspace{0.25\baselineskip}
    If each $V_i$ is the span of a single basis function, this defines a
    Jacobi (Gauss-Seidel) iteration.
  \end{exampleblock}
\end{frame}

\begin{frame}[t]
  \frametitle{Idea 2: kernel-capturing multigrid relaxation}
  \begin{theorem}[Sch\"oberl (1999); Lee, Wu, Xu, Zikatanov (2007)]
    \vspace{0.25\baselineskip}
    Let the kernel be
    \begin{equation*}
      \mathcal{N} := \{ u \in V : b(u, v) = 0 \,\, \forall v \in V \}.
    \end{equation*}
    If the subspace decomposition \emph{captures the kernel}
    \begin{equation*}
      \mathcal{N} = \sum_i \mathcal{N} \cap V_i,
    \end{equation*}
    then convergence of the relaxation defined by this decomposition
    will be \emph{independent} of $\alpha$ and $\beta$.
    \nocite{Schoeberl:1999,Lee:2007}
  \end{theorem}

  \begin{corollary}
    \vspace{0.25\baselineskip}
    Capturing the kernel of the augmented Lagrangian term results in a robust multigrid method.
  \end{corollary}
\end{frame}

\begin{frame}[t]
  \frametitle{The kernel of $\overline{\nabla \cdot}$}
  \begin{theorem}[Sch\"oberl (1999)]
    \vspace{0.25\baselineskip}
    Choose \emph{discrete} stabilisation, and \emph{piecewise constant} pressures. The \emph{star} decomposition $V = \sum_i V_i$ given by:
    \begin{center}
      \begin{tikzpicture}[scale=4]
        \tikzstyle{v}=[circle, fill, minimum size=0pt, inner sep=0pt]
        \tikzstyle{c}=[diamond, fill, minimum size=4pt, inner sep=0pt]
        \tikzstyle{e}=[rectangle, fill, minimum size=3pt, inner sep=0pt]
        \tikzstyle{select}=[minimum size=5pt]
        \node[v] (v1) at (0, 0) {};
        \node[v] (v2) at (0.25, 0) {};
        \node[v] (v3) at (0.5, 0) {};
        \node[v] (v4) at (0.75, 0) {};
        \node[v] (v5) at (1, 0) {};
        \node[v] (v6) at (0, 0.25) {};
        \node[v] (v7) at (0, 0.5) {};
        \node[v] (v8) at (0, 0.75) {};
        \node[v] (v9) at (0, 1) {};
        \node[v] (v10) at (0.25, 0.25) {};
        \node[v] (v11) at (0.25, 0.5) {};
        \node[v] (v12) at (0.25, 0.75) {};
        \node[v] (v13) at (0.25, 1) {};
        \node[v] (v14) at (0.5, 0.25) {};
        \node[v] (v15) at (0.5, 0.5) {};
        \node[v] (v16) at (0.5, 0.75) {};
        \node[v] (v17) at (0.5, 1) {};
        \node[v] (v18) at (0.75, 0.25) {};
        \node[v] (v19) at (0.75, 0.5) {};
        \node[v] (v20) at (0.75, 0.75) {};
        \node[v] (v21) at (0.75, 1) {};
        \node[v] (v22) at (1, 0.25) {};
        \node[v] (v23) at (1, 0.5) {};
        \node[v] (v24) at (1, 0.75) {};
        \node[v] (v25) at (1, 1) {};

        \draw[thin,gray] (v1) -- (v2) -- (v6) -- (v1);
        \draw[thin,gray] (v2) -- (v10) -- (v6);
        \draw[thin,gray] (v2) -- (v3) -- (v10);
        \draw[thin,gray] (v3) -- (v14) -- (v10);
        \draw[thin,gray] (v3) -- (v4) -- (v14);
        \draw[thin,gray] (v4) -- (v18) -- (v14);
        \draw[thin,gray] (v4) -- (v5) -- (v18);
        \draw[thin,gray] (v5) -- (v22) -- (v18);
        \draw[thin,gray] (v6) -- (v7) -- (v10);
        \draw[thin,gray] (v7) -- (v11) -- (v10);
        \draw[thin,gray] (v11) -- (v14) -- (v15) -- (v11);
        \draw[thin,gray] (v15) -- (v18) -- (v19) -- (v15);
        \draw[thin,gray] (v19) -- (v22) -- (v23) -- (v19);
        \draw[thin,gray] (v7) -- (v8) -- (v11);
        \draw[thin,gray] (v8) -- (v12) -- (v11);
        \draw[thin,gray] (v12) -- (v15) -- (v16) -- (v12);
        \draw[thin,gray] (v16) -- (v19) -- (v20) -- (v16);
        \draw[thin,gray] (v20) -- (v23) -- (v24) -- (v20);
        \draw[thin,gray] (v8) -- (v9) -- (v12);
        \draw[thin,gray] (v9) -- (v13) -- (v12);
        \draw[thin,gray] (v13) -- (v16) -- (v17) -- (v13);
        \draw[thin,gray] (v17) -- (v20) -- (v21) -- (v17);
        \draw[thin,gray] (v21) -- (v24) -- (v25) -- (v21);

        \node[v, select, blue!75!green] at (v15) {};
        \node at (barycentric cs:v11=1,v14=1,v15=1) (s1) {};
        \node at (barycentric cs:v14=1,v15=1,v18=1) (s2) {};
        \node at (barycentric cs:v15=1,v18=1,v19=1) (s3) {};
        \node at (barycentric cs:v11=1,v12=1,v15=1) (s4) {};
        \node at (barycentric cs:v12=1,v15=1,v16=1) (s5) {};
        \node at (barycentric cs:v15=1,v16=1,v19=1) (s6) {};
        \draw[thick,blue!75!green] \convexpath{s1,s4,s5,s6,s3,s2}{1pt};

        \node[v, select, blue!75!green] at (v11) {};
        \node at (barycentric cs:v12=1,v11=1,v8=1) (ss1) {};
        \node at (barycentric cs:v7=1,v11=1,v8=1) (ss2){};
        \node at (barycentric cs:v7=1,v10=1,v11=1) (ss3) {};
        \node at (barycentric cs:v14=1,v10=1,v11=1) (ss4) {};
        \draw[thick, blue!75!green] \convexpath{ss4,ss3,ss2,ss1,s4,s1}{1pt};
      \end{tikzpicture}
    \end{center}
    captures the kernel of $\overline{\nabla \cdot }$.
  \end{theorem}
\end{frame}

\begin{frame}[t]
  \frametitle{Idea 3: robust prolongation for
    $\overline{\nabla \cdot}$}
  \begin{theorem}[Sch\"oberl (1999)]
    \vspace{0.25\baselineskip}
    Let the prolongation $\mathbb{P} : V_H \to V_h$. For the multigrid
    cycle to be robust we require:
    \begin{equation*}
      \Re^{-1} \| \nabla \mathbb{P} u_H \|^2_{L^2} + \gamma \|\overline{\nabla \cdot \mathbb{P} u_H}\|^2_{L^2} \le c \left(\Re^{-1} \| \nabla u_H \|^2_{L^2} + \gamma \|\overline{\nabla \cdot u_H}\|^2_{L^2}\right) \quad \forall u_H \in V_H,
    \end{equation*}
    with $c$ independent of $\Re$.
  \end{theorem}
  Can ensure this by solving a small Stokes problem in each coarse cell:
  \begin{center}
    \begin{tikzpicture}[scale=3]
      \tikzstyle{v}=[circle, fill, minimum size=0pt, inner sep=0pt]
      \tikzstyle{c}=[diamond, fill, minimum size=4pt, inner sep=0pt]
      \tikzstyle{e}=[rectangle, fill, minimum size=3pt, inner sep=0pt]
      \tikzstyle{select}=[minimum size=5pt]
      \node[v] (v1) at (0, 0) {};
      \node[v] (v2) at (0.25, 0) {};
      \node[v] (v3) at (0.5, 0) {};
      \node[v] (v4) at (0.75, 0) {};
      \node[v] (v5) at (1, 0) {};
      \node[v] (v6) at (0, 0.25) {};
      \node[v] (v7) at (0, 0.5) {};
      \node[v] (v8) at (0, 0.75) {};
      \node[v] (v9) at (0, 1) {};
      \node[v] (v10) at (0.25, 0.25) {};
      \node[v] (v11) at (0.25, 0.5) {};
      \node[v] (v12) at (0.25, 0.75) {};
      \node[v] (v13) at (0.25, 1) {};
      \node[v] (v14) at (0.5, 0.25) {};
      \node[v] (v15) at (0.5, 0.5) {};
      \node[v] (v16) at (0.5, 0.75) {};
      \node[v] (v17) at (0.5, 1) {};
      \node[v] (v18) at (0.75, 0.25) {};
      \node[v] (v19) at (0.75, 0.5) {};
      \node[v] (v20) at (0.75, 0.75) {};
      \node[v] (v21) at (0.75, 1) {};
      \node[v] (v22) at (1, 0.25) {};
      \node[v] (v23) at (1, 0.5) {};
      \node[v] (v24) at (1, 0.75) {};
      \node[v] (v25) at (1, 1) {};

      \draw[thin,gray] (v1) -- (v2) -- (v6) -- (v1);
      \draw[thin,gray] (v2) -- (v10) -- (v6);
      \draw[thin,gray] (v2) -- (v3) -- (v10);
      \draw[thin,gray] (v3) -- (v14) -- (v10);
      \draw[thin,gray] (v3) -- (v4) -- (v14);
      \draw[thin,gray] (v4) -- (v18) -- (v14);
      \draw[thin,gray] (v4) -- (v5) -- (v18);
      \draw[thin,gray] (v5) -- (v22) -- (v18);
      \draw[thin,gray] (v6) -- (v7) -- (v10);
      \draw[thin,gray] (v7) -- (v11) -- (v10);
      \draw[thin,gray] (v11) -- (v14) -- (v15) -- (v11);
      \draw[thin,gray] (v15) -- (v18) -- (v19) -- (v15);
      \draw[thin,gray] (v19) -- (v22) -- (v23) -- (v19);
      \draw[thin,gray] (v7) -- (v8) -- (v11);
      \draw[thin,gray] (v8) -- (v12) -- (v11);
      \draw[thin,gray] (v12) -- (v15) -- (v16) -- (v12);
      \draw[thin,gray] (v16) -- (v19) -- (v20) -- (v16);
      \draw[thin,gray] (v20) -- (v23) -- (v24) -- (v20);
      \draw[thin,gray] (v8) -- (v9) -- (v12);
      \draw[thin,gray] (v9) -- (v13) -- (v12);
      \draw[thin,gray] (v13) -- (v16) -- (v17) -- (v13);
      \draw[thin,gray] (v17) -- (v20) -- (v21) -- (v17);
      \draw[thin,gray] (v21) -- (v24) -- (v25) -- (v21);

      \draw[very thick, black] (v1) -- (v5);
      \draw[very thick, black] (v1) -- (v9);
      \draw[very thick, black] (v3) -- (v17);
      \draw[very thick, black] (v5) -- (v25);
      \draw[very thick, black] (v9) -- (v25);
      \draw[very thick, black] (v7) -- (v23);
      \draw[very thick, black] (v3) -- (v7);
      \draw[very thick, black] (v9) -- (v5);
      \draw[very thick, black] (v17) -- (v23);

      \node[v, select, blue!75!green] at (barycentric cs:v7=1,v3=1,v15=1) (cc1) {};
      \node at (barycentric cs:v3=1,v10=1,v14=1) (fc1) {};
      \node at (barycentric cs:v14=1,v15=1,v11=1) (fc2) {};
      \node at (barycentric cs:v7=1,v10=1,v11=1) (fc3) {};
      \draw[thick, blue!75!green] \convexpath{fc3,fc2,fc1}{1pt};

      \node[v, select, blue!75!green] at (barycentric cs:v7=1,v9=1,v15=1) (cc2) {};
      \node at (barycentric cs:v7=1,v8=1,v11=1) (fc4) {};
      \node at (barycentric cs:v8=1,v9=1,v12=1) (fc5) {};
      \node at (barycentric cs:v11=1,v12=1,v15=1) (fc6) {};
      \draw[thick, blue!75!green] \convexpath{fc4,fc5,fc6}{1pt};
    \end{tikzpicture}
  \end{center}
\end{frame}


\begin{frame}[fragile]
  \frametitle{Full solver}
  \resizebox{\textwidth}{!}{
    \begin{tikzpicture}[
      every node/.style={draw=black, thick, anchor=west},
      grow via three points={one child at (0.0,-0.7) and
        two children at (0.0,-0.7) and (0.0,-1.4)},
      edge from parent path={(\tikzparentnode.210) |- (\tikzchildnode.west)}]
      \node {Continuation}
      child { node {Newton solver with line search}
        child { node {Krylov solver (FGMRES)}
          child { node {Block preconditioner}
            child { node {Approximate Schur complement inverse}}
            child { node {F-cycle on augmented momentum block}
              child { node {Coarse grid solver}
                child { node {LU factorization}}
              }
              child [missing] {}
              child { node {Prolongation operator}
                child { node {Local solves over coarse cells}}
              }
              child [missing] {}
              child { node {Relaxation}
                child { node {GMRES}
                  child { node {Additive star iteration}}
                }
              }
            }
          }
        }
      };
    \end{tikzpicture}
  }
\end{frame}


\begin{frame}
  \frametitle{2D regularised lid-driven cavity}
  \begin{table}[htbp]
    \centering
    \begin{tabular}{cc|ccccc}
      \toprule
      \# refinements & \# dofs & \multicolumn{5}{c}{Reynolds number} \\
                     && 10 & 100 & 1000 & 5000 & 10000 \\
      \midrule
      1 & $1.0 \times 10^4$ & 3.00 &  4.00 &  5.33 & 8.50 & 11.0 \\
      2 & $4.1 \times 10^4$ & 2.50 &  3.67 &  6.00 & 8.00 & 9.50 \\
      3 & $1.6 \times 10^5$ & 2.50 &  3.00 &  5.67 & 7.50 & 9.00 \\
      4 & $6.6 \times 10^5$ & 2.50 &  3.00 &  5.00 & 7.00 & 8.00 \\
      \bottomrule
    \end{tabular}
    \caption{Average outer Krylov iterations per Newton step for the
      2D regularised lid-driven cavity.}
    \label{tab:ourldc}
  \end{table}
\end{frame}

\begin{frame}
  \frametitle{3D regularised lid-driven cavity}
  \begin{table}
    \centering
    \begin{tabular}{cc|ccccc}
      \toprule
      \# refinements & \# dofs & \multicolumn{5}{c}{Reynolds number} \\
                     && 10 & 100 & 1000 & 2500 & 5000 \\
      \midrule
      1 & $2.1 \times 10^6$ & 7.50 & 7.33 & 7.50 & 7.00 & 6.50 \\
      2 & $1.7 \times 10^7$ & 8.50 & 7.00 & 7.50 & 6.50 & 5.50 \\
      3 & $1.3 \times 10^8$ & 7.00 & 7.00 & 6.50 & 5.00 & 6.50 \\
      4 & $1.1 \times 10^9$ & 7.00 & 7.33 & 5.50 & 4.00 & 9.00 \\
      \bottomrule
    \end{tabular}
    \caption{Average outer Krylov iterations per Newton step for the 3D regularised lid-driven cavity.}
  \end{table}
\end{frame}

\begin{frame}[t]
  \frametitle{Conclusions \& outlook}
  \begin{block}{Main conclusion}
    \vspace{0.215\baselineskip}
    Solving the Navier--Stokes equations in a $\Re$-robust way \emph{is} possible.
  \end{block}
  \begin{block}{Ongoing work}
    \vspace{0.125\baselineskip}
    Better element pair, by choosing exact subcomplexes of
    \begin{equation*}
      \mathbb{R} \xrightarrow{\operatorname{id}} H^2 \xrightarrow{\grad} H^1(\curl)
      \xrightarrow{\curl} H^1 \xrightarrow{\div} L^2 \xrightarrow{\operatorname{null}} 0,
    \end{equation*}
    such that the subspace of $H^1(\curl)$ has a local basis.
  \end{block}
  \begin{block}{Other PDEs}
    \begin{columns}[t]
      \begin{column}{0.5\textwidth}
        \begin{itemize}
        \item $H(\curl)$ Riesz map \cmark
        \item $H(\div)$ Riesz map \cmark
        \item Biharmonic $I + \nabla^2$ \textcolor{gray}{\cmark}
        \item Vanka (monolithic multigrid) for Stokes \cmark
        \end{itemize}
      \end{column}
      \begin{column}{0.5\textwidth}
        \begin{itemize}
        \item non-Newtonian Navier--Stokes?
        \item nearly incompressible hyperelasticity?
        \item \dots
        \end{itemize}
      \end{column}
    \end{columns}
  \end{block}
\end{frame}

\begin{frame}[allowframebreaks, t]
  \frametitle{References}
  \printbibliography[heading=none]
\end{frame}
\end{document}
