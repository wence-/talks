\documentclass[presentation]{beamer}
\usetheme{metropolis}
\metroset{progressbar=frametitle}
\usepackage{tikz}
\date{13th March 2018}
\title{Compiling tensor algebra for finite elements}
\usepackage{censor}
\author{Lawrence Mitchell\inst{1,*}\\
  {\scriptsize Thomas Gibson, David A.~Ham, Mikl\'os Homolya, Paul H.~J.~Kelly, Fabio
    Luporini, TJ Sun}
}
\institute{
\inst{1}Department\alert{s} of Computing and Mathematics,
\censor{Imperial College London}

\inst{*}\texttt{lawrence.mitchell@imperial.ac.uk}
}
\usepackage{amsmath}
\usepackage{amssymb}
\usepackage{mathtools}

\DeclarePairedDelimiter\abs{\lvert}{\rvert}%
\DeclarePairedDelimiter\norm{\lVert}{\rVert}%
\usepackage{pifont}
\usepackage{shellesc}
\usepackage{minted}
\newcommand{\cmark}{\ding{51}}
\newcommand{\xmark}{\ding{55}}
\newcommand{\arxivlink}[2]{%
  \href{http://www.arxiv.org/abs/#1}%
  {\texttt{arXiv:\,#1\,[#2]}}%
}
\newcommand{\doilink}[1]{%
  \href{http://dx.doi.org/#1}%
  {\texttt{doi:\,#1}}%
}
\usepackage[url=false,
            doi=true,
            isbn=false,
            style=authoryear,
            maxnames=10,
            firstinits=true,
            uniquename=init,
            backend=biber]{biblatex}

\setbeamertemplate{bibliography item}{}
\renewcommand{\bibfont}{\fontsize{8}{8}\selectfont}
\addbibresource{references.bib}
\setlength{\bibitemsep}{1ex}
\setlength{\fboxsep}{1pt}

\renewbibmacro{in:}{}
\DeclareFieldFormat[article]{volume}{\textbf{#1}}
\DeclareFieldFormat{doi}{%
  doi\addcolon%
  \ifhyperref{\href{http://dx.doi.org/#1}{\nolinkurl{#1}}}
  {\nolinkurl{#1}}}
\AtEveryBibitem{%
\clearfield{pages}%
\clearfield{issue}%
\clearfield{number}%
}

\DeclareMathOperator{\tr}{tr}

\graphicspath{{./\jobname.figures/}{../pictures/}}

\begin{document}

\begin{frame}[plain,noframenumbering]
  \maketitle
  \begin{tikzpicture}[remember picture,overlay]
    \node[at=(current page.south west), anchor=south west] {\includegraphics[height=0.9cm]{epsrc-logo}};
    \node[at=(current page.south east), anchor=south east] {\includegraphics[height=0.9cm]{firedrake-small}};
  \end{tikzpicture}
\end{frame}

% \begin{abstract}
%   Compiling tensor algebra for finite element computations
%   ========================================================

%   At the core of a PDE any library that uses finite elements is a
%   large tensor contraction.  Providing a low flop count, highly
%   efficient, implementation of this contraction is either devolved to
%   the computational scientist (and then a general purpose compiler),
%   or else to a domain specific compiler (and thence to a general
%   purpose one).

%   I will talk about the domain specific compiler, and the optimisation
%   passes, that we use in the Firedrake project
%   (www.firedrakeproject.org), that deliver low algorithmic
%   complexity algorithms on a class of finite elements that exhibit
%   kronecker product structure.

%   I will then cover some open questions and future research
%   directions, in particular how to extend the code transformation
%   pipeline to incorporate operations on tensors that are not easily
%   expressible as scalar indexed expressions: of particular interest is
%   to widen the applicability to include tensor inverse and determinant
%   calculations.
% \end{abstract}


\begin{frame}
  \frametitle{Firedrake \url{www.firedrakeproject.org}}

  \begin{quote}
    {\normalfont [\ldots]} an automated system for the solution of partial
    differential equations using the finite element method.
  \end{quote}

  \begin{onlyenv}<1>
    \begin{itemize}
    \item Written in Python.
    \item Finite element problems specified with \emph{embedded}
      domain specific language, UFL \parencite{Alnaes:2014} from the
      FEniCS project.
    \item \emph{Runtime} compilation to low-level (C) code.
    \item Explicitly \emph{data parallel} API.
    \end{itemize}

    \begin{flushright}
      {\scriptsize F.~Rathgeber, D.A.~Ham, \textbf{LM}, M.~Lange,
        F.~Luporini, A.T.T.~McRae, G.-T.~Bercea, G.R.~Markall,
        P.H.J.~Kelly. TOMS,
        2016. \arxivlink{1501.01809}{cs.MS}\nocite{Rathgeber:2016}}
    \end{flushright}
  \end{onlyenv}
  \begin{onlyenv}<2>
    \begin{block}{User groups at}
      Imperial, Bath, Leeds, Kiel, Rice, Houston, Exeter, Buffalo,
      ETH, Waterloo, Minnesota, Baylor, Texas A\&M, Oxford, \dots
    \end{block}
  \end{onlyenv}
\end{frame}

\begin{frame}[fragile]
  \frametitle{DSLs for finite elements}
  \begin{columns}
    \begin{column}{0.47\framewidth}
      {\footnotesize
        Find $(u, p, T) \in V\times W\times Q$ s.t.
        \begin{align*}
          \int\!\nabla u \cdot \nabla v + (u \cdot \nabla u) \cdot v \\
          - p\nabla\cdot v + \frac{\text{Ra}}{\text{Pr}} Tg \hat{z} \cdot v\,\text{d}x &= 0 \\
          \int\!\nabla\cdot u q\,\text{d}x&= 0\\
          \int\! (u\cdot \nabla T) S + \text{Pr}^{-1} \nabla T \cdot \nabla
          S\,\text{d}x &= 0\\
          \quad \forall\, (v,q,T) \in V\times W \times Q
        \end{align*}
        }
    \end{column}
      \begin{column}{0.52\framewidth}
\begin{minted}[fontsize=\tiny]{python}
from firedrake import *
mesh = Mesh(...)
V = VectorFunctionSpace(mesh, "CG", 2)
W = FunctionSpace(mesh, "CG", 1)
Q = FunctionSpace(mesh, "CG", 1)
Z = V * W * Q
Ra = Constant(...)
Pr = Constant(...)
upT = Function(Z)
u, p, T = split(upT)
v, q, S = TestFunctions(Z)
bcs = [...]

F = (inner(grad(u), grad(v))
     + inner(dot(grad(u), u), v)
     - inner(p, div(v))
     + (Ra/Pr)*inner(T*g, v)
     + inner(div(u), q)
     + inner(dot(grad(T), u), S)
     + (1/Pr) * inner(grad(T), grad(S)))*dx

solve(F == 0, upT, bcs=bcs)
\end{minted}
      \end{column}
  \end{columns}
\end{frame}

\begin{frame}
  \frametitle{Just tensor algebra}
  \begin{itemize}
  \item Continuous
    \begin{equation*}
      \int \nabla u \cdot \nabla v \,\text{d}x = \int J^{-T}\hat{\nabla}
      \tilde u \cdot J^{-T}\hat{\nabla} \tilde v \abs{\det J}\,\text{d}\hat{x}
    \end{equation*}
  \item Discrete, with quadrature $\{(\xi_q, w_q)\}$
    \begin{equation*}
      \mathbf{A}_{i j} = \sum_{q\alpha \beta \gamma} w_q K_{q\beta \alpha}
      \frac{\partial{\Psi_i}}{\partial \hat{x}_\beta}(\xi_{q})
      K_{q\gamma \alpha} \frac{\partial{\Psi_j}}{\partial \hat{x}_\gamma}(\xi_{q})
      \abs{\det J_q}
    \end{equation*}

    with $J_{q\alpha\beta}$ the Jacobian of the coordinate transformation and
    $K_{q\alpha\beta}$ its inverse.
  \item \emph{Always} linear in $\Psi_i$ and $\Psi_j$, the ``arguments''.
  \item Sometimes, there is structure, $\Psi_i = \Psi_{(i_1,i_2,\dots,i_d)} =
    \psi_{i_1}\psi_{i_2}\dots\psi_{i_d}$ and so
    $\partial_{\hat{x}_\alpha}\Psi_i =
    \psi_{i_1}\dots\partial_{\hat{x}_\alpha}\psi_{i_\alpha}\dots\psi_{i_d}$
  \end{itemize}
\end{frame}

\begin{frame}
  \frametitle{Why a domain-specific compiler?}
  \begin{itemize}
  \item Want predictably good performance across:
    \begin{itemize}
    \item choice of PDE;
    \item discretisation order.
    \end{itemize}
  \item Want $<10\text{s}$ compilation time for anything domain
    scientist throws at it.
  \item Need to exploit domain knowledge to prune search spaces,
    build cost models, need to choose optimisations that are ``always
    good''.
  \end{itemize}
\end{frame}

\begin{frame}
  \frametitle{IR: GEM ``comes after FEM''}
  \begin{itemize}
  \item Replace finite element terminals by \emph{evaluation rules}
    \begin{flushright}
      {\footnotesize Homolya, \textbf{LM}, Luporini,
        Ham. \arxivlink{1705.03667}{cs.MS}}
    \end{flushright}
  \item Ensure that the evaluation rules express structure
    \begin{flushright}
      {\footnotesize Homolya, Kirby, Ham. \arxivlink{1711.02473}{cs.MS}}
    \end{flushright}
    
  \item Now we ``just'' have a tensor contraction to transform.
  \item Passes implemented as GEM $\to$ GEM transformers.
  \end{itemize}
\end{frame}

\begin{frame}
  \frametitle{Optimisation passes}
  \begin{itemize}
  \item Argument factorisation: transform to ``sum-of-products''
    canonical form.
  \item Delta cancellation over tensor contractions
  \item Delta cancellation across assignments
  \item Sum factorisation: optimal contraction order found by
    searching all permutations ($N!$, but $N\le 4$).
  \item For each contraction, apply ILP-driven factorisation algorithm of
    {\footnotesize Luporini, Ham, Kelly (2017) \arxivlink{1604.05872}{cs.MS}}
    (exploits distributivity).
  \end{itemize}
\end{frame}
\begin{frame}
  \frametitle{Open problems}
  \begin{itemize}
  \item Often want to build expressions like
    \begin{equation*}
      D_{ij} - C_{ik} (A^{-1})_{k l} B_{l j}
    \end{equation*}
  \item $|i| = |j| \in [1, 100]$.
  \item Impractical to expand out the expression for $A^{-1}$ for
    large $|i|$.  So we must deal with the matrix ``as a whole''.
  \item Right now to do this we build matrix-sized temporaries, and do
    the linear algebra with Eigen.
  \item \emph{Destroys structure}.  Sad.
  \end{itemize}
    \begin{flushright}
      {\footnotesize Gibson, Mitchell, Ham, Cotter. \arxivlink{1802.00303}{cs.MS}}
    \end{flushright}
\end{frame}

\begin{frame}
  \frametitle{With tensor products I can do better}
  \begin{itemize}
  \item   Let $\mathcal{A} = \bigotimes_i A_i$
    with $A_i$ $n\times n$.  Then
    \item
      $\mathcal{A}^{-1} = \bigotimes_i A_i^{-1};$
    \item
      $\mathcal{A}^T = \bigotimes_i A_i^T;$
    \item
      $\det \mathcal{A} = \prod_i (\det A_i)^n.$
    \item Suggests that I need representation in IR of inverse (and
      det), and transformation rules that deal with them.
    \item Any ideas?
    \end{itemize}
\end{frame}

\begin{frame}[standout]
  Thanks!

  {\small www.firedrakeproject.org}
\end{frame}
\end{document}
