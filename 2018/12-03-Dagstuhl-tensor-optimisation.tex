\documentclass[presentation]{beamer}
\usetheme{metropolis}
\metroset{progressbar=frametitle}
\usepackage{tikz}

\date{12th March 2018}
\title{Compiling tensor algebra for finite elements}
\author{Lawrence Mitchell\inst{1,*}}
\institute{
\inst{1}Department\textbf{s} of Computing and Mathematics, Imperial College
London

\inst{*}\texttt{lawrence.mitchell@imperial.ac.uk}
}
\usepackage{pifont}
\usepackage{minted}
\newcommand{\cmark}{\ding{51}}
\newcommand{\xmark}{\ding{55}}
\newcommand{\arxivlink}[2]{%
  \href{http://www.arxiv.org/abs/#1}%
  {\texttt{arXiv:\,#1\,[#2]}}%
}
\newcommand{\doilink}[1]{%
  \href{http://dx.doi.org/#1}%
  {\texttt{doi:\,#1}}%
}
\usepackage[url=false,
            doi=true,
            isbn=false,
            style=authoryear,
            maxnames=10,
            firstinits=true,
            uniquename=init,
            backend=biber]{biblatex}

\setbeamertemplate{bibliography item}{}
\renewcommand{\bibfont}{\fontsize{8}{8}\selectfont}
\addbibresource{references.bib}
\setlength{\bibitemsep}{1ex}
\setlength{\fboxsep}{1pt}

\renewbibmacro{in:}{}
\DeclareFieldFormat[article]{volume}{\textbf{#1}}
\DeclareFieldFormat{doi}{%
  doi\addcolon%
  \ifhyperref{\href{http://dx.doi.org/#1}{\nolinkurl{#1}}}
  {\nolinkurl{#1}}}
\AtEveryBibitem{%
\clearfield{pages}%
\clearfield{issue}%
\clearfield{number}%
}

\DeclareMathOperator{\tr}{tr}

\graphicspath{{./\jobname.figures/}{../pictures/}}

\begin{document}

\begin{frame}[plain,noframenumbering]
  \maketitle
  \begin{tikzpicture}[remember picture,overlay]
    \node[at=(current page.south west), anchor=south west] {\includegraphics[height=0.9cm]{epsrc-logo}};
    \node[at=(current page.south east), anchor=south east] {\includegraphics[height=0.9cm]{imperial-two-tone}};
  \end{tikzpicture}
\end{frame}

\begin{abstract}
  Compiling tensor algebra for finite element computations
  ========================================================

  At the core of a PDE any library that uses finite elements is a
  large tensor contraction.  Providing a low flop count, highly
  efficient, implementation of this contraction is either devolved to
  the computational scientist (and then a general purpose compiler),
  or else to a domain specific compiler (and thence to a general
  purpose one).

  I will talk about the domain specific compiler, and the optimisation
  passes, that we use in the Firedrake project
  (www.firedrakeproject.org), that deliver low algorithmic
  complexity algorithms on a class of finite elements that exhibit
  kronecker product structure.

  I will then cover some open questions and future research
  directions, in particular how to extend the code transformation
  pipeline to incorporate operations on tensors that are not easily
  expressible as scalar indexed expressions: of particular interest is
  to widen the applicability to include tensor inverse and determinant
  calculations.
\end{abstract}

