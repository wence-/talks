\documentclass{article}
\usepackage[utf8]{inputenc}
\usepackage{amsmath,amsfonts,amssymb}
\usepackage{graphicx}
\usepackage{times}
\parindent0pt
\parskip 1.5ex plus 1ex minus .5ex
%%%%%%%%%%%%%%%%%%%%%%%%%%%%%%%%%%%%%%%%%%%%%%%%%%%%%%%%%%%%%%%%%%%%%%
% Do not change above this line! Do not add packages! You may comment
% out packages missing in your installation.
%
% Everything before "\begin{document}" will be ignored! Do not define
%   macros!
%%%%%%%%%%%%%%%%%%%%%%%%%%%%%%%%%%%%%%%%%%%%%%%%%%%%%%%%%%%%%%%%%%%%%%
\begin{document}
\title{Solver composition across the PDE/linear algebra divide}
\author{%
  Lawrence Mitchell\thanks{Department of Computing and Department of
    Mathematics, Imperial College London
    \texttt{lawrence.mitchell@imperial.ac.uk}}
  \and
  Robert C.~Kirby\thanks{Department of Mathematics, Baylor University,
    \texttt{robert\_kirby@baylor.edu}}
}
\maketitle

The efficient solution of discretisations of coupled systems of
partial differential equations (PDEs) is at the core of much of
numerical simulation.  Significant effort has been expended on
scalable algorithms to precondition Krylov iterations for the linear
systems that arise.  With few exceptions, the reported numerical
implementation of such solution strategies is specific to a particular
model setup, and intimately ties the solver strategy to the
discretisation and PDE, especially when the preconditioner requires
auxiliary operators.

In this talk, I will present recent improvements in the Firedrake
finite element library that allow for
straightforward development of the building blocks of extensible,
composable preconditioners that decouple the solver from the model
formulation.

The implementation extends the algebraic composability of linear
solvers offered by the PETSc library by augmenting operators, and
hence preconditioners, with the ability to provide any necessary
auxiliary operators.  As such, the full flexibility of the PDE library
is available \emph{within} the solver: many complex preconditioners
for Schur complement factorisations can be coded in a few 10s of
lines.  Moreover, these preconditioners are reusable and nestable
within larger block solvers.


\end{document}

%%% Local Variables: 
%%% mode: latex
%%% TeX-master: t
%%% End: 
