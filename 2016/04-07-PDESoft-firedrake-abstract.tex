\documentclass{article}
\usepackage[utf8]{inputenc}
\usepackage{amsmath,amsfonts,amssymb}
\usepackage{graphicx}
\usepackage{times}
\parindent0pt
\parskip 1.5ex plus 1ex minus .5ex
%%%%%%%%%%%%%%%%%%%%%%%%%%%%%%%%%%%%%%%%%%%%%%%%%%%%%%%%%%%%%%%%%%%%%%
% Do not change above this line! Do not add packages! You may comment
% out packages missing in your installation.
%
% Everything before "\begin{document}" will be ignored! Do not define
%   macros!
%%%%%%%%%%%%%%%%%%%%%%%%%%%%%%%%%%%%%%%%%%%%%%%%%%%%%%%%%%%%%%%%%%%%%%
\begin{document}
\title{Firedrake: composable abstractions for finite element simulation}

\author{%
  Lawrence Mitchell\thanks{Department of Computing and Department of Mathematics, Imperial College London,
    \texttt{lawrence.mitchell@imperial.ac.uk}}
  \and
  David A.~Ham\thanks{Department of Mathematics, Imperial College London}
  \and
  Gheorghe-Teodor Bercea\thanks{Department of Computing, Imperial
    College London}, 
  Mikl\'os Homolya\footnotemark[3], Fabio Luporini\footnotemark[3], Paul
  H.~J.~Kelly\footnotemark[3]
  \and
  Florian Rathgeber\thanks{European Centre for Medium-range Weather
    Forecasts}
  \and
  Andrew T.~T.~McRae\thanks{Department of Mathematical Sciences, University of Bath}
}
\maketitle

The complexity inherent in the application of advanced numerics on
modern hardware to coupled physical systems presents a critical
barrier to simulation development.  To overcome this, we must create
simulation software which embodies the abstraction and composability
of the underlying mathematics.  In this way, a system is created in
which mathematicians, computer scientists, and application specialists
can each deploy their own expertise, benefiting from the expertise of
the others.  Critically, this approach minimises the extent to which
individuals must become polymaths to share in these advances.

In this talk I will present Firedrake and PyOP2, a composition of new
and existing abstractions which creates a particularly complete
separation of concerns.  A key observation is that by careful
abstraction design, we are able to teach computers how to reason about
the mathematical structure of the problem, rather than working out
everything on paper.  This enables the creation of high performance,
sophisticated finite element models from a very high level
mathematical specification and has enabled advances in computer
science and numerics, while also facilitating the creation of
simulation systems for a variety of applications.

\end{document}
