\documentclass{beamer}


\usepackage{tikz}
\usetikzlibrary{positioning,calc}
\usetikzlibrary{shapes.geometric}
\usetikzlibrary{backgrounds}% only to show the bounding box
\usetikzlibrary{shapes,arrows}
\usepackage{pgfplots}
\usepackage{pgfplotstable}
\usetikzlibrary{pgfplots.groupplots}
\pgfplotsset{compat=1.12}
\usepackage{appendixnumberbeamer}
\usepackage{amsmath}
\date{8th June 2016}
\usetheme{metropolis}

\definecolor{river_color}{RGB}{65, 105, 225}
\definecolor{ocean_color}{RGB}{220, 20, 60}
\definecolor{plume_color}{RGB}{218, 165, 32}

\title[]{Three dimensional baroclinic ocean model}
\author[Tuomas K\"{a}rn\"{a}]{
Tuomas K\"{a}rn\"{a}\inst{1},
Ant\'{o}nio M. Baptista\inst{1},
Stephan Kramer\inst{2},
\underline{Lawrence Mitchell}\inst{2},
David Ham\inst{2}
}


\institute[CMOP]{
\inst{1}
Oregon Health \& Science University\\
\inst{2}
Imperial College London
}

\date{13th July 2016}

%\titlegraphic{
%  \includegraphics[height=29pt]{figures/logos/nsflogo}
%  \includegraphics[height=30pt]{figures/logos/cmoplogo}
%  \includegraphics[height=30pt]{thetis_logo_white}
%   \includegraphics[width=180px]{logo_IMUM_mesh_04_800px}
%}

\begin{document}

\maketitle

% TODO get global ocean model figure, and example model
% TODO get ROMS example figure

\frame{
\frametitle{Why do we need a new coastal ocean model?}
\begin{itemize}
 \item Sophisticated structured grid ocean models exists for global and regional applications
 \item Coastal applications require unstructured grids
 \item Unstructured grid models tend to be \emph{expensive} and \emph{diffusive}
\end{itemize}
\vspace{10pt}
\begin{columns}
 \column{0.32\textwidth}
 \textbf{Global}
 \begin{center}
 \includegraphics[width=0.95\textwidth]{global_grid_nash_remo}
 \end{center}
 \column{0.32\textwidth}
 \textbf{Regional}
 \begin{center}
 \includegraphics[width=0.95\textwidth]{curvilinear_grid_cape_cod_bays_usgs}
 \end{center}
 \column{0.32\textwidth}
 \textbf{Coastal/estuary}
 \begin{center}
 \includegraphics[width=\textwidth]{mesh_db31_coast-to-estuary_plain}
 \end{center}
\end{columns}
}


\frame{
\frametitle{Columbia River estuary application}
\small
\begin{columns}
 \column{0.5\textwidth}
  \vspace{-10pt}
  \begin{itemize}
   \item Complex topography
   \item Energetic tides
   \item Strong river discharge
   \item Strong density gradients
   \item Baroclinic effects very important
%   \item Complex topography
%   \item Wide range scales
%   \begin{itemize}
%     \item Coast: 1 yr, 100 km
%     \item Estuary: 1 min, 100 m
%   \end{itemize}
%   \item Advection dominated
%   \item Energetic tides
%   \begin{itemize}
%     \item Currents ~3 m/s
%   \end{itemize}
%   \item Strong density gradients
%   \begin{itemize}
%     \item from 0 to 32 psu in
%     \item[] 100 m (horizontal)
%     \item[] 1 m (vertical)
%   \end{itemize}
%   \item Baroclinic effects very important
  \end{itemize}
  \vspace*{20pt}
  $\rightarrow\ $ Current model (SELFE) is too diffusive
 \column{0.6\textwidth}
%  \vspace*{-35pt}
 \begin{center}
 \hspace*{-20pt}
   \raisebox{5pt}{\includegraphics[width=0.25\textwidth]{map_CRE_small}}
   \hspace{2pt}
   \includegraphics[width=0.78\textwidth]{SELFE_CRE_salt_contours_0211}
  \\
 \vspace{10pt}
 \includegraphics[width=\textwidth]{auvcmp_run027_AUV48_salt_leg09_small_2}
 \end{center}
\end{columns}
}


\frame{
\vspace*{-25pt}
\begin{columns}
 \column{0.3\textwidth}
  \begin{center}
  \includegraphics[width=0.8\textwidth]{thetis_logo_white}
  \end{center}
 \column{0.7\textwidth}
 Unstructured grid coastal ocean model\\[5pt]
 {\small
 \url{https://github.com/thetisproject/thetis}
 }
\end{columns}
\vspace{10pt}
\small
\begin{columns}
\column{0.9\textwidth}
\begin{itemize}
 \item Solves 3D Navies-Stokes equations
 \item Hydrostatic and Boussinesq approximations
 \item Implemented on \raisebox{-7pt}{\includegraphics[height=15pt]{firedrake_logo}}
\end{itemize}
\end{columns}
}

\frame{
\frametitle{Idealized river plume test: Thetis is less diffusive}
\vspace*{-8pt}
\begin{center}
% Idealized river plume test case\\
\includegraphics[width=0.75\textwidth]{rhinerofi_thetis_salt_comp_frame_0209.png}\\
\includegraphics[width=0.75\textwidth]{rhinerofi_selfe_salt_comp_frame_0209.png}
\end{center}
}

\frame{
\frametitle{Why Firedrake?}
\begin{columns}
 \column{0.3\textwidth}
 \includegraphics[width=\textwidth]{mesh_3d_bath}
 \column{0.7\textwidth}
 \includegraphics[width=\textwidth]{frame_ale2_0040}
\end{columns}
\vspace{10pt}
\begin{itemize}
 \item Mesh is extruded in the vertical direction
 \item Large element aspect ratio
 \item Vertical/horizontal dynamics are treated separately
 \item[] $\Rightarrow$ Non-affine finite elements $\Rightarrow$ TSFC form compiler
%  \item Large element aspect ratio:
%  \begin{itemize}
%   \item horizontal / vertical length scale $ \approx 100 \ldots 1000$
%   % \item Depth 1 ... 100 m
%   % \item Horizontal scale 100 ... 10 000 m
%  \end{itemize}
%  \item Horizontal and vertical dynamics treated separately
%  \item Mesh extruded in the vertical direction
%  \item[] $\Rightarrow$ Non-affine finite elements $\Rightarrow$ TSFC form compiler
\end{itemize}
}

\frame{
\frametitle{Governing equations (I)}
\small
\textbf{Horizontal momentum equation} ($\bu$)
\begin{align}
\begin{split}
% \pD{\bu}{t}
\pd{\bu}{t} + \bnabla_h \cdot (\bu\bu)
+ \pd{\left(w\bu \right)}{z} +& \\
f\bm{e}_z\wedge\bu + g\bnabla_h \eta + g\bnabla_h r
 &=
\bnabla_h \cdot \left( \nu_h \bnabla_h \bu \right) + \pd{}{z}\left( \nu  \pd{\bu}{z}\right),
\label{eq:momentum3d}
\end{split}
% \text{with}\ r &= \frac{1}{\rho_0} \int_{z}^\eta \rho' d\zeta, \nonumber %\label{eq:baroclinicHead}
\end{align}

\visible<2>{
\textbf{Continuity equation} ($w$, diagnostic)
\begin{align}
\bnabla_h \cdot \bu  + \pd{w}{z} = 0,  \label{eq:continuity3d}
\end{align}
}
}

\frame{
\frametitle{Governing equations (II)}
\small
Separate 2D equations for splitting 2D/3D dynamics (mode-splitting)\\[20pt]
\textbf{2D Depth Averaged Shallow Water equations} ($\eta$, $\bbaru$)
\begin{align}
\pd{\eta}{t} + \bnabla_h \cdot \left( (\eta + h) \bbaru \right) &= 0,  \label{eq:freesurface2d} \\
 \pd{\bbaru}{t} + \bbaru \cdot \bnabla_h\bbaru + f\bm{e}_z\wedge \bbaru
+ g \bnabla_h \eta + g \frac{1}{H}\int_{-h}^\eta \bnabla_h r dz &=  \bar{\bm{A}}_H \label{eq:momentum2d}
\end{align}
}

\frame{
\frametitle{Governing equations (III)}
\small
\textbf{Tracer equation} ($T$, $S$, temperature, salinity)
\begin{align}
\begin{split}
\pd{T}{t} + \bnabla_h \cdot (\bu T) + \pd{\left(wT \right)}{z} =
 \bnabla_h \cdot \left( \mu_h \bnabla_h T \right) + \pd{}{z}\left( \mu  \pd{T}{z}\right),  \label{eq:T}
\end{split}
\end{align}
\textbf{Equation of state} ($\rho'$ , diagnostic)
\begin{align*}
\rho' = \rho'(T, S)
\end{align*}
\textbf{Internal pressure gradient}  ($\bnabla_h r$ , diagnostic)
\begin{align*}
\bnabla_h r &= \frac{1}{\rho_0} \bnabla_h  \int_{z}^\eta  \rho' d\zeta, \nonumber %\label{eq:baroclinicHead}
\end{align*}
}

\frame{
\frametitle{Governing equations (IV)}
\small
\textbf{Generic Length scale turbulence closure model} ($\nu, \mu$)
\begin{align}
  \pd{k}{t} + \bnabla_h \cdot (\bu k) + \pd{(w k)}{z} &= \pd{}{z}\left(\frac{\nu}{\sigma_k} \pd{k}{z}\right) + P + B - \varepsilon\\
  \pd{\Psi}{t} + \bnabla_h \cdot (\bu \Psi) + \pd{(w k)}{z} &= \pd{}{z}\left(\frac{\nu}{\sigma_\Psi} \pd{\Psi}{z}\right) + \frac{\Psi}{k}(c_1 P + c_3 B - c_2 \varepsilon) \\
  \nu &= c_\nu \sqrt{k} L, \nonumber \\
  \mu &= c_{\mu} \sqrt{k} L, \nonumber \\
  L &= \left(c_\mu^0\right)^{-p/n} k^{-m/n} \Psi^{1/n} \nonumber
\end{align}
}


\frame{
\frametitle{Discretization}
\textbf{Element families}
\begin{itemize}
 \item P1DG
 \begin{itemize}
  \item 2D element pair: P1DG-P1DG
  \item 3D velocity, tracers: P1DG$\times$P1DG
  % \item Tracers: P1DG$\times$P1DG
  \item Turbulent quantities: P0$\times$P0
 \end{itemize}
 \item Mimetic pair RT1-P1DG, ...
\end{itemize}

\visible<2>{
\textbf{Time integration}
\begin{itemize}
 \item Implicit terms
 \begin{itemize}
  \item Surface gravity waves: $\Theta$ scheme
  \item Vertical diffusion: Backward Euler
 \end{itemize}
 \item Explicit terms
 \begin{itemize}
  \item Strong Stability Preserving Runge-Kutta SSPRK(3,3)
 \end{itemize}
 \item Tracer advection
 \begin{itemize}
  \item SSPRK scheme + slope limiters = positivity preserving scheme
 \end{itemize}
\end{itemize}
}
}

% \frame{
% \frametitle{DG-DG element pair}
% \small
% \begin{columns}
% \column{0.45\textwidth}
%   \textbf{Elements for $n=1$}\\[5pt]
%   \textbf{2D}
%   \begin{itemize}
%     \item 2D velocity: P1DG
%     \item 2D elevation: P1DG
%   \end{itemize}
% \uncover<2>{
%   \textbf{3D}
%   \begin{itemize}
%     \item horizontal velocity: \mbox{P1DG x P1DG}
%     \item vertical velocity: \mbox{P1DG x P1DG}
%     \item tracers: P1DG x P1DG
%     \item turb. variables: P0 x P0
%   \end{itemize}
% }
% \column{0.47\textwidth}
%   \includegraphics[height=32pt]{triangle_p1} \raisebox{5pt}{$U, V$}\hspace{18pt}
%   \raisebox{0pt}{\includegraphics[height=32pt]{triangle_p1}} \raisebox{13pt}{$\eta$}\\[10pt]
%   \vspace{20pt}
% \visible<2>{
%   \includegraphics[height=50pt]{prism_p1} $u,v$\hspace{8pt}
%   \includegraphics[height=50pt]{prism_p1} $w$\\[5pt]
%   \hspace{3pt}\includegraphics[height=50pt]{prism_p1} $T,S$
%   \hspace{8pt}\includegraphics[height=50pt]{prism_p0} $k,\Psi$
% }
% \end{columns}
% }

% \frame{
% \frametitle{RT-DG element pair}
% \small
% \begin{columns}
% \column{0.45\textwidth}
%   \textbf{Elements for $n=1$}\\[5pt]
%   \textbf{2D}
%   \begin{itemize}
%     \item 2D velocity: RT1
%     \item 2D elevation: P1DG
%   \end{itemize}
% \uncover<2>{
%   \textbf{3D}
%   \begin{itemize}
%     \item horizontal velocity: \mbox{RT1 x P1DG $\in \mathcal{H}(\text{div})$}
%     \item vertical velocity: \mbox{P1DG x P2 $\in \mathcal{H}(\text{div})$}
%     \item tracers: P1DG x P1DG
%   \end{itemize}
% }
% \column{0.47\textwidth}
%   \includegraphics[height=40pt]{triangle_RT2} \raisebox{5pt}{$U, V$}\hspace{-8pt}
%   \raisebox{25pt}{$\xrightarrow{\text{div}(\cdot)}$}\hspace{5pt}
%   \raisebox{8pt}{\includegraphics[height=32pt]{triangle_p1}} \raisebox{13pt}{$\eta$}\\[10pt]
%   \vspace{20pt}
% \visible<2>{
%   \includegraphics[height=50pt]{mimetic_prism_uv_RT-DG} $u,v$\hspace{8pt}
%   \includegraphics[height=50pt]{mimetic_prism_w_DG-CG} $w$\\[5pt]
%   \hspace{20pt}$\downarrow\ \ ${\scriptsize$\text{div}(\cdot)$}\\[5pt]
%   \hspace{5pt}\includegraphics[height=50pt]{prism_p1} $T,S$
% }
% \end{columns}
% }

% \frame{
% \frametitle{Time integration}
% \begin{itemize}
%  \item Most restrictive CFL conditions
%  \begin{itemize}
%   \item \textcolor{red}{Free surface waves}: {\scriptsize $\Delta t < \Delta x / \sqrt{gH}$}
%   \item \textcolor{blue}{Vertical diffusion}: {\scriptsize $\Delta t < \Delta z^2 / \nu$}
%   \item Horizontal advection: {\scriptsize $\Delta t < \Delta x / U$}
%  \end{itemize}
%  \item<2-> 2D shallow water equations
%  \begin{itemize}
%   \item \textcolor{red}{Free surface waves} are treated with $\Theta$ scheme
%   \item Rest of the terms are explicit
%  \end{itemize}
%  \item<3-> 3D momentum and tracer equations
%  \begin{itemize}
%   \item \textcolor{blue}{Vertical diffusion} is treated implicitly
%   \item Rest of the terms are explicit
%  \end{itemize}
%  \item<4-> Tracer advection
%  \begin{itemize}
%   \item Accurate and positivity preserving advection scheme
%   \item 3rd order Strong-Stability Preserving Runge-Kutta scheme
%   \item Slope limiters
%   \item No overshoots
%  \end{itemize}
% \end{itemize}
% }

% \frame{
% \frametitle{Function spaces}
% \small
% \textbf{Horizontal space} (velocity-pressure)
% \begin{itemize}
%  \item FE element pairs for rotating shallow water flows
% %  \item<2-> P1DG-P1DG
% %  \item<2-> P1DG-P2 (Cotter et al., 2009)
% %  \item<3-> RT1-P1DG (\emph{mimetic elements}, McRae and Cotter, 2014)
% \end{itemize}
% \visible<2->{
% \textbf{P1DG-P1DG}
% \begin{itemize}
%   \item 2D elevation: P1DG
%   \item 2D velocity: P1DG
%   \item 3D horizontal velocity: P1DG$\times$P1DG
%   \item 3D vertical velocity: P1DG$\times$P1DG
%   \item 3D tracers: P1DG$\times$P1DG
% \end{itemize}
% }
% \visible<3->{
% \textbf{RT1-P1DG} \emph{Mimetic elements} (McRae and Cotter, 2014)
% \begin{itemize}
%   \item 2D elevation: P1DG
%   \item 2D velocity: RT1
%   \item 3D horizontal velocity: RT1$\times$P1DG
%   \item 3D vertical velocity: P1DG$\times$P1DG
%   \item 3D tracers: P1DG$\times$P1DG
% \end{itemize}
% }
% }

% \frame{
% \frametitle{Lock-exchange flow}
% \begin{itemize}
%  \item Simulation without turbulence
%  \item Lock exchange?
% \end{itemize}
% }
%
% \frame{
% \frametitle{Turbulence closure}
% \begin{itemize}
%  \item Equations
%  \item Implementation
% \end{itemize}
% }

\frame{
\frametitle{Idealized estuarine circulation}
\includegraphics[width=\textwidth]{warner_normal_frame_0175}
}

\frame{
\frametitle{Idealized river plume simulation}
\hspace*{-18pt}
\includegraphics[width=1.1\textwidth]{rhinerofi_coarse_frame_0221.png}
}

\frame{
\frametitle{What about computational efficiency?}
Lock-exchange test case
\begin{itemize}
 \item[--] 512 triangles, 20 layers, 10k elements
 \item[--] 61k DOFs (P1DG scalar)
 \item[--] CPU time for 900 s simulation on single core
\end{itemize}
\begin{center}
\small
\begin{tabular}{lr@{.}lr}
Firedrake ``bendy'' and FFC form compilers & $\approx$190&0 s & 100\% \\
With TSFC form compiler & 145&5 s & 76\% \\
With COFFEE optimizations & 93&1 s & 45\% \\
Tuned solver parameters and quadrature rules & 67&0 s & 35\% \\
\hline\\
\visible<2>{
Comparison: SLIM P1DG-P1GD model & $\approx$ 148&0 s
}
\end{tabular}
\end{center}
}

\frame{
\frametitle{}
\vspace{-40pt}
\begin{columns}
 \column{0.3\textwidth}
  \begin{center}
  \includegraphics[width=0.85\textwidth]{thetis_logo_white}
  \end{center}
 \column{0.7\textwidth}
 \vspace*{10pt}\\
 Unstructured grid coastal ocean model\\[3pt]
 {\small
 \url{https://github.com/thetisproject/thetis}
 }
\end{columns}
\vspace{5pt}
\begin{itemize}
 \item Realistic ocean model can be implemented in Firedrake
 \item Model is computationally efficient
\end{itemize}
\vspace{12pt}
\visible<2>{
Acknowledgments:\\
\hspace*{0pt}  \raisebox{0pt}{\includegraphics[width=0.15\textwidth]{logos/nsflogo}}
\hspace*{0pt} \raisebox{5pt}{\includegraphics[width=0.3\textwidth]{xsede_logo}}
\hspace*{20pt} \raisebox{15pt}{\includegraphics[width=0.25\textwidth]{TACC_logo}}\\
\vspace*{-20pt}
\hspace*{0pt}  \raisebox{5pt}{\includegraphics[width=0.4\textwidth]{firedrake_logo}}
\hspace*{5pt} \raisebox{0pt}{\includegraphics[width=0.3\textwidth]{fenics_logo_text_transparent}}
\hspace*{5pt} \raisebox{5pt}{\includegraphics[width=0.2\textwidth]{PETCS_logo}}\\
}
}

% \frame{
% \frametitle{Next steps}
% Thetis
% \begin{itemize}
%  \item Solvers: Improve computational efficiency
% \end{itemize}
%
% Firedrake
% \begin{itemize}
%  \item Better checkpointing
%  \item Support for Local Discontinuous Galerkin formulation
% \end{itemize}
% }



% \frame{
% \scriptsize
% \bibliographystyle{apalike}
% \bibliography{references}
% }


\end{document}
