\documentclass[presentation]{beamer}

\usepackage{tikz}
\usetikzlibrary{positioning,calc}
\usetikzlibrary{shapes.geometric}
\usetikzlibrary{shapes,arrows}
\usepackage{pgfplots}
\usepackage{pgfplotstable}
\usetikzlibrary{pgfplots.groupplots}
\usepackage{appendixnumberbeamer}
\usepackage{amsmath}
\usetheme{metropolis}

\renewcommand{\vec}[1]{\ensuremath{\boldsymbol{#1}}}
\newcommand{\ddt}[1]{\frac{\partial #1}{\partial t}}
\newcommand{\zhat}{\hat{\vec{z}}}
\newcommand{\W}{\ensuremath{\mathbb{W}}}

\DeclareMathOperator{\grad}{grad}
\let\div\relax
\DeclareMathOperator{\div}{div}
\DeclareMathOperator{\curl}{curl}

\author{Lawrence Mitchell\inst{1} \and Eike M\"uller\inst{2}}
\institute{\inst{1}Departments of Computing and Mathematics, Imperial College London 
  \and
  \inst{2}Department of Mathematical Sciences, University of Bath}
\title{Multigrid for numerical weather prediction}

\graphicspath{{./\jobname.figures/}}

\usepackage[url=false,
            doi=true,
            isbn=false,
            style=authoryear,
            firstinits=true,
            uniquename=init,
            backend=biber]{biblatex}

\setbeamertemplate{bibliography item}{}
\renewcommand{\bibfont}{\footnotesize}
\addbibresource{references.bib}

\setlength{\bibitemsep}{1ex}

\renewbibmacro{in:}{}
\DeclareFieldFormat[article]{volume}{\textbf{#1}}
\DeclareFieldFormat{doi}{%
  doi\addcolon%
  {\scriptsize\ifhyperref{\href{http://dx.doi.org/#1}{\nolinkurl{#1}}}
    {\nolinkurl{#1}}}}
\AtEveryBibitem{%
\clearfield{pages}%
\clearfield{issue}%
\clearfield{number}%
}

\usepackage{minted}

\begin{document}
\maketitle

\section{Introduction}

\begin{frame}
  \frametitle{Motivation}
  Weather forecasting centres moving from orthogonal
  lat-lon to non-orthogonal meshes (e.g.~cubed sphere, or icosahedral
  sphere).
  \begin{center}
    \begin{tikzpicture}
      \node (A) {\includegraphics[width=2cm]{sphere-latlon}}; \node
      (B) [right=of A] {\includegraphics[width=2cm]{sphere-cubed}};
      \draw[-latex, very thick] (A)--(B);
    \end{tikzpicture}
  \end{center}
  Can no longer use traditional staggered finite differences.
  
  \emph{compatible finite elements} \parencite{Cotter:2012a} a
  plausible soltion.
\end{frame}

\begin{frame}
  \frametitle{Compatible FE}
  Choose discrete spaces that match the properties of the
  continuous equations.  

  \begin{equation*}
    \W_0 \stackrel{\grad}{\longrightarrow}  \W_1  \stackrel{\curl}{\longrightarrow}  \W_2  \stackrel{\div}{\longrightarrow}  \W_3
  \end{equation*}

  We use tensor product spaces on wedges

  \begin{center}
    \begin{tikzpicture}[scale=0.25]
      \node[thick, isosceles triangle, isosceles triangle apex
      angle=60, draw] (A) {};
    
      \node (B) [right=0.05cm of A] {$\otimes$};
    
      \node (C) [right=0.05cm of B, thick, rectangle, minimum
      width=0pt, inner sep=0pt, draw, minimum height=0.5cm] {};

      \draw[-latex, very thick] (C)++(0.5cm,0) -- +(2cm, 0);
      \begin{scope}[shift={($(C) + (3cm, -2cm)$)}]
        \draw[thick] (0,0) rectangle (4,4); \draw[thick] (0,0) --
        (1,1); \draw[thick] (4,0) -- (1,1); \draw[thick] (0,4) --
        (1,5); \draw[thick] (4,4) -- (1,5); \draw[thick] (1,1) --
        (1,5);
      \end{scope}
    \end{tikzpicture}
  \end{center}
\end{frame}

\begin{frame}
  \frametitle{Computational challenges}
  Dynamical core (fluids solver) needs to handle fast acoustic waves
  implicitly (to avoid terrible CFL constraint).  Results in an
  elliptic problem for pressure correction.

  Need a solver that is \emph{scalable} and \emph{fast}

  Large aspect ratio of the domain is challenging for black box solvers.
\end{frame}

\section{Formulation}

\begin{frame}
  \frametitle{Linearised gravity wave}
  Linear system for velocity $\vec{u}$, pressure $p$ and buoyancy $b$.
  \begin{columns}
    \begin{column}{0.4\textwidth}
      \begin{align*}
        \label{eq:1}
        \ddt{\vec{u}} &= \nabla p + b \zhat, \\
        \ddt{p} &= -c^2 \nabla\cdot \vec{u}, \\
        \ddt{b} &= -N^2\vec{u}\cdot\zhat.\\
      \end{align*}
    \end{column}
    \begin{column}{0.6\textwidth}
      Domain $S^2(R) \times [0, H]$.\\
      Enforce $\vec{u}\cdot \hat{\vec{n}} = 0$ at top and bottom.\\
      $R\approx 6000\textsf{km}$, $H\approx 80\textsf{km}$.
    \end{column}
  \end{columns}
\end{frame}

\begin{frame}
  \frametitle{Discretised system}
  After discretising in space and time we obtain a linear operator for
  $u \subset H(\div)$, $p \subset L^2$, $b$ scalar, collocated with
  vertical part of $u$ space.
\begin{equation*}
\begin{pmatrix}
  M_v & 
    -\frac{\Delta t}{2}D^T & 
    -\frac{\Delta t}{2}Q\\[1ex]
  \frac{\Delta t}{2}c^2D & M_p & 0\\[1ex]
  \frac{\Delta t}{2}N^2Q^T & 0 & M_b
\end{pmatrix}
\end{equation*}
\end{frame}

\begin{frame}
  \frametitle{Preconditioning}
  Without mountains, buoyancy can be eliminated pointwise, with
  mountains the pointwise elimination is good in a preconditioner.  We
  focus on the velocity-pressure system
\begin{equation*}
\begin{pmatrix}
  M_v & -\frac{\Delta t}{2}D^T\\
  \frac{\Delta t}{2}c^2D & M_p\\
\end{pmatrix}
\end{equation*}

This is a discretisation of a sign-positive $H(\div)-L^2$ Helmholtz
equation.
\end{frame}

\begin{frame}[t]
  \frametitle{Options}
  \begin{columns}
    \begin{column}{0.52\textwidth}
      \begin{block}{Riesz map}
        Block diagonal preconditioner using the appropriate function
        space inner products
        \begin{equation*}
          \begin{pmatrix}
            (I - \grad \div)^{-1} & 0 \\
            0 & I\\
          \end{pmatrix}
        \end{equation*}
      \end{block}
      \begin{itemize}
      \item All operators sparse
      \item $H(\div)$ multigrid \parencite{Arnold:2000} is challenging
      \end{itemize}
    \end{column}
    \begin{column}{0.52\textwidth}
      \begin{block}{Schur complement}
        Block elimination and back substitution
        \begin{equation*}
          \begin{pmatrix}
            M_v^{-1} & 0 \\
            0 & (M_p + \omega_c^2 D M_v^{-1} D^T)^{-1}\\
          \end{pmatrix}
        \end{equation*}
      \end{block}
      \begin{itemize}
      \item $H = M_p +  \omega_c^2 D M_v^{-1} D^T$ is \emph{dense}, but elliptic
      \item need preconditioner, $\tilde{S}$.
      \item Can use similar methods as a \emph{distributive} smoother.
      \end{itemize}
    \end{column}
  \end{columns}
\end{frame}

\begin{frame}[allowframebreaks]
  \frametitle{Approximating $S$}
  \begin{itemize}
  \item Strong coupling in vertical direction, need to treat this
    exactly.  
  \item Use a multigrid cycle with horizontal coarsening and
    vertical line relaxation.
  \item Exploit tensor-product structure to split horizontal and
    vertical components
  \end{itemize}
\framebreak

\begin{block}{Steps to $\tilde{S}$}
  \begin{itemize}
  \item Split horizontal and vertical $M_v^{-1} =
    (M_v^h)^{-1}\oplus (M_v^z)^{-1}$ (also $D
      = D^h \oplus D^z$)
  \item Diagonal approximation $M_v^{-1} \approx M_{v,\text{inv}} = \delta_{ij} / (M_v)_{ii}$
  \item Ignore horizontal coupling, error $\mathcal{O}((\Delta z/
    \Delta x)^2)$
  \end{itemize}
\end{block}
\begin{equation*}
  \tilde{S} = M_p + \omega_c^2 \text{bdiag}\big[D^h M_{v,\text{inv}}^h
  (D^h)^T\big] \oplus \frac{\omega_c^2}{1 + \omega_N^2} D^z
    M_{v,\text{inv}}^z (D^z)^T
\end{equation*}
\end{frame}

\begin{frame}[fragile, allowframebreaks]
  \frametitle{Implementation}
  Multigrid V-cycle using exact inverse of $\tilde{S}$ as smoother.
  Utilise column innermost numbering to perform fast banded matrix
  inverse
  \begin{center}
    \begin{tikzpicture}
      \node (A) {\includegraphics[height=2.5cm]{bandedmatrix}};
      \node (C) [right=of A.north east, anchor=north west] {\includegraphics[width=3.5cm]{columndofs}};
      \node (B) [below=0.1cm of C] {\includegraphics[height=0.6cm]{linearstorage}};
      \node (D) [above=0.1cm of A.north east, anchor=south] {\includegraphics[width=5cm]{MeshHierarchy}};
    \end{tikzpicture}    
  \end{center}

\framebreak

Banded matrix algebra implemented as short PyOP2 kernels

\begin{minted}[fontsize=\tiny]{python}
# C-kernel code for Matrix-vector product v = A.u in one vertical column
kernel_code = '''void ax(double **A, double **u, double **v) {
  for (int i=0;i<n_row;++i) {                                   
    v[0][i] = 0.0;                                                     
    int j_m = (int) ceil((alpha*i-gamma_p)/beta);   
    int j_p = (int) floor((alpha*i+gamma_m)/beta);  
    for (int j=std::max(0,j_m);j<std::min(n_col,j_p+1);++j)
      v[0][i] += A[0][bandwidth*i+(j-j_m)] * u[j];
  }'''
# Execute PyOP2 kernel over grid
kernel = op2.Kernel(kernel_code,'ax',cpp=True)
op2.par_loop(kernel,hostmesh.cell_set, A(op2.READ,Vcell.cell_node_map()),
             u.dat(op2.READ,u.cell_node_map()),
             v.dat(op2.WRITE,u.cell_node_map()))
\end{minted}
\end{frame}
\section{Results}

\begin{frame}
  \frametitle{Algorithmic performance}

  \begin{center}
    \begin{tikzpicture}
      \begin{semilogyaxis}[title={Convergence with $\text{CFL} = 8$, low-order},
        xlabel={Iterations}, ylabel={Relative residual},
        legend entries={Custom MG, Single level, Algebraic (PETSc) MG}]
        \addplot+[only marks] table [x=iterations, y=residual]
        {\jobname.figures/mf-lo.dat};

        \addplot+[only marks] table [x=iterations, y=residual]
        {\jobname.figures/single-level-lo.dat};

        \addplot+[only marks] table [x=iterations, y=residual]
        {\jobname.figures/petsc-lo.dat};
      \end{semilogyaxis}
    \end{tikzpicture}
  \end{center}

\end{frame}

\begin{frame}
  \frametitle{Algorithmic performance}

  \begin{center}
    \begin{tikzpicture}
      \begin{semilogyaxis}[title={Convergence with $\text{CFL} = 8$, high-order},
        xlabel={Iterations}, ylabel={Relative residual},
        legend entries={Custom MG, Single level, Algebraic (PETSc) MG}]
        \addplot+[only marks] table [x=iterations, y=residual]
        {\jobname.figures/mf-ho.dat};

        \addplot+[only marks] table [x=iterations, y=residual]
        {\jobname.figures/single-level-ho.dat};

        \addplot+[only marks] table [x=iterations, y=residual]
        {\jobname.figures/petsc-ho.dat};
      \end{semilogyaxis}
    \end{tikzpicture}
  \end{center}

  % CFL variation + time

  % Mesh independence

\end{frame}

\begin{frame}[t]
  \frametitle{Varying CFL}
  \begin{center}
  \begin{tikzpicture}[scale=0.65]
    \begin{groupplot}[group style={group size=2 by 1, horizontal sep=2cm}],
      \nextgroupplot[xmode=log,xlabel=CFL ($\Delta t/\Delta x$),
      ylabel=V cycles,
      log basis x=2,
      legend entries={Custom MG (LO), Custom MG (HO),
        PETSc MG (LO), PETSc MG (HO)},
      legend style={legend to name=grouplegend, legend columns=2}]
      \addplot+[line width=2pt, solid, blue, mark=*, mark options={fill=blue}] table [x=CFL, y=its] {\jobname.figures/mf-lo-cfl.dat};
      \addplot+[line width=2pt, dashed, blue, mark=*, mark options={fill=blue}] table [x=CFL, y=its] {\jobname.figures/mf-ho-cfl.dat};
      \addplot+[line width=2pt, solid, brown, mark=square*, mark options={fill=brown}] table [x=CFL, y=its] {\jobname.figures/petsc-lo-cfl.dat};
      \addplot+[line width=2pt, dashed, brown, mark=square*, mark options={fill=brown}] table [x=CFL, y=its] {\jobname.figures/petsc-ho-cfl.dat};
      \nextgroupplot[xmode=log,xlabel=CFL ($\Delta t/\Delta x$),
      ylabel=Time (s),
      log basis x=2]
      \addplot+[line width=2pt, solid, blue, mark=*, mark options={fill=blue}] table [x=CFL, y=Time] {\jobname.figures/mf-lo-cfl.dat};
      \addplot+[line width=2pt, dashed, blue, mark=*, mark options={fill=blue}] table [x=CFL, y=Time] {\jobname.figures/mf-ho-cfl.dat};
      \addplot+[line width=2pt, solid, brown, mark=square*, mark options={fill=brown}] table [x=CFL, y=Time] {\jobname.figures/petsc-lo-cfl.dat};
      \addplot+[line width=2pt, dashed, brown, mark=square*, mark options={fill=brown}] table [x=CFL, y=Time] {\jobname.figures/petsc-ho-cfl.dat};
    \end{groupplot}
    \node at ($(group c1r1)!0.5!(group c2r1) + (0,-5cm)$) {\ref{grouplegend}}; 
  \end{tikzpicture}
  \end{center}
\end{frame}

\begin{frame}
  \frametitle{Weak scaling}
  \begin{tikzpicture}[scale=0.65]
    \begin{groupplot}[group style={group size=2 by 1, horizontal sep=2cm}],
      \nextgroupplot[xmode=log,xlabel=Num processes,
      ylabel=Time (s),
      axis y line*=left,
      title={Low order},
      xtick={1, 6, 24, 96, 384, 1536},
      xticklabels={$1$, $6$, $24$, $96$, $384$, $1536$},
      legend entries={Custom MG, PETSc MG},
      legend style={legend to name=grouplegend, legend columns=2}]
      \addplot+[line width=2pt, blue, mark=*, mark options={fill=blue}] table [x=Np, y=Time] {\jobname.figures/mf-weak-lo.dat};
      \addplot+[line width=2pt, brown, mark=square, mark options={fill=brown}] table [x=Np, y=Time] {\jobname.figures/petsc-weak-lo.dat};

      \nextgroupplot[xmode=log,xlabel=Num processes,
      ylabel=Time (s),
      title={High order},
      xtick={1, 6, 24, 96, 384, 1536, 6144},
      xticklabels={$1$, $6$, $24$, $96$, $384$, $1536$, $6144$}]
      \addplot+[line width=2pt, blue, mark=*, mark options={fill=blue}] table [x=Np, y=Time] {\jobname.figures/mf-weak-ho.dat};
      \addplot+[line width=2pt, brown, mark=square, mark options={fill=brown}] table [x=Np, y=Time] {\jobname.figures/petsc-weak-ho.dat};
    \end{groupplot}
    \node at ($(group c1r1)!0.5!(group c2r1) + (0,-5cm)$) {\ref{grouplegend}}; 
  \end{tikzpicture}
\end{frame}


\section{Concluding}

\begin{frame}
  \frametitle{Conclusions}
  Not so bad

  Programming this stuff is still painful

  Hybridisation?

  Auxiliary space \parencite{Hiptmair:2007} or $H(\div)$ multigrid?
\end{frame}

\begin{frame}[standout]
  
  Thanks!
  
\end{frame}
% \begin{frame}
%   \frametitle{Implementation}
%   Firedrake: hierarchy of extruded meshes

%   Banded matrix algebra and line smoothers with pyop2 -- foo
% \end{frame}

% \section{Results}

% \begin{frame}
%   \frametitle{Weak scaling}
  
% \end{frame}
\appendix
\begin{frame}[t, allowframebreaks]
  \frametitle{References}
  \printbibliography[heading=none]
\end{frame}
\end{document}
