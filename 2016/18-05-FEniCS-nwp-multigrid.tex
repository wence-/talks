\documentclass[presentation]{beamer}

\usepackage{tikz}
\usepackage{appendixnumberbeamer}
\usepackage{amsmath}
\usetheme{metropolis}

\renewcommand{\vec}[1]{\ensuremath{\boldsymbol{#1}}}
\newcommand{\ddt}[1]{\frac{\partial #1}{\partial t}}
\newcommand{\zhat}{\hat{\vec{z}}}
\newcommand{\W}{\ensuremath{\mathbb{W}}}

\DeclareMathOperator{\Grad}{grad}
\DeclareMathOperator{\Div}{div}
\author{Lawrence Mitchell\inst{1} \and Eike M\"uller\inst{2}}
\institute{\inst{1}Departments of Computing and Mathematics, Imperial College London 
  \and
  \inst{2}Department of Mathematical Sciences, University of Bath}
\title{Multigrid for numerical weather prediction}
\subtitle{A subtitle}

\begin{document}
\maketitle

\section{Introduction}

\begin{frame}
  \frametitle{Weather prediction}
  
  For scalability, most weather centres are moving to
  \emph{non-orthogonal} meshes of the sphere

  Rules out, for numerics reasons, traditional staggered finite
  difference discretisations

  Extension to non-orthogonal, \emph{compatible finite elements} \cite{Cotter:2012a}
\end{frame}

\begin{frame}
  \frametitle{3D compatible FE}
  Commuting diagram

  \begin{equation*}
  \W_0 \stackrel{\nabla}{\longrightarrow} \W_1 \stackrel{\nabla\times}{\longrightarrow} \W_2 \stackrel{\nabla\cdot}{\longrightarrow} \W_3,
\end{equation*}

  Tensor products
\end{frame}

\section{Model problem}

\begin{frame}
  \frametitle{Idealised test case}
  Linearised gravity wave

  \begin{align*}
    \label{eq:1}
   \ddt{\vec{u}} &= \nabla p + b \zhat, \\
   \ddt{p} &= -c^2 \nabla\cdot \vec{u}, \\
   \ddt{b} &= -N^2\vec{u}\cdot\zhat. \\
    \vec{u}\cdot \hat{\vec{n}} &= 0 \quad \textsf{on boundary}
  \end{align*}
\end{frame}

\begin{frame}
  \frametitle{Discrete system for updates}
Eliminate buoyancy pointwise (holds strongly in absence of mountains)

$\delta \vec{u}\in \W_2^0$, $\delta b\in\W_b$ and
$\delta p\in\W_3$
\begin{align*}
  \langle\vec{w},\delta\vec{u}\rangle
  - \frac{\Delta t}{2}\langle\nabla\cdot\vec{w},\delta p\rangle
  - \frac{\Delta t}{2}\langle\vec{w},\delta b\zhat\rangle
  &= \Delta t \langle\nabla\cdot\vec{w},p_0\rangle
  + \Delta t \langle\vec{w},b_0\zhat\rangle
   &\forall \vec{w}\in\W_2^0
 \\
  \langle\phi,\delta p\rangle
  +\frac{\Delta t}{2}c^2\langle\phi,\nabla\cdot\delta\vec{u}\rangle
  &= -\Delta tc^2\langle\phi,\nabla\cdot\vec{u}_0\rangle
   &\forall \phi\in\W_3
  \\
  \langle\gamma,\delta b\rangle
  + \frac{\Delta t }{2}N^2\langle\gamma,\delta\vec{u}\cdot\zhat\rangle
  &= -\Delta t N^2\langle\gamma,\vec{u}_0\cdot\zhat\rangle
   &\forall \gamma\in\W_b
\end{align*}
\end{frame}

\begin{frame}
  \frametitle{A preconditioner}
  After eliminating $b$ we obtain a block $2\times2$ elliptic problem
\begin{equation}
  A\begin{pmatrix}\vec{U}\\[1ex]\vec{P}\end{pmatrix}
  \equiv
  \begin{pmatrix}
   \tilde{M}_2 & -\frac{\Delta t}{2}D^T  \\[1ex]
   \frac{\Delta t}{2}c^2 D & M_3
  \end{pmatrix}
  \begin{pmatrix}\vec{U}\\[1ex]\vec{P}\end{pmatrix}
 =
\begin{pmatrix}M_2\tilde{\vec{R}}_u\\[1ex]M_3\vec{R}_p\end{pmatrix}
\end{equation}
\begin{equation}
  \label{eq:2}
\begin{matrix}
\tilde{M}_2 = M_2^h \oplus (1+ \omega_N^2) M_2^z, \quad \omega_N \equiv \frac{\Delta t}{2}N
\\[1ex]
\tilde{\vec{R}}_u = \vec{R}_u+\frac{\Delta t}{2}M_2^{-1}Q\vec{R}_b.
\end{matrix}
\end{equation}
\end{frame}
\begin{frame}
  \frametitle{Choices}
  Block diagonal \emph{Riesz-map}.  $(I - \Grad\Div)^{-1}, I$

  Schur complement $I, S^{-1}$

  Former has an infinite dimensional null space (so standard multigrid
  doesn't work).

  Latter requires that we can invert $S = D - C^T A^{-1} C$ which is
  dense if formed.
\end{frame}

\begin{frame}
  \frametitle{Schur complements}
  Due to anisotropy in the domain, dominant couplings are in the
  vertical direction.  It is these which we need to treat well.
  Effectively we need a line smoother in $S_z$.

  
\end{frame}

\section{Implementation}

\begin{frame}
  \frametitle{Implementation}
  Firedrake: hierarchy of extruded meshes

  Banded matrix algebra and line smoothers with pyop2 -- foo
\end{frame}

\section{Results}

\begin{frame}
  \frametitle{Weak scaling}
  
\end{frame}
\appendix
\begin{frame}[allowframebreaks]
  \frametitle{References}
  \bibliography{references}
  \bibliographystyle{abbrv}
\end{frame}
\end{document}
