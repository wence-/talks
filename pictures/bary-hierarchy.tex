\documentclass{standalone}

\usepackage{tikz}
\usepackage{expl3}
\usetikzlibrary{calc,arrows}

\begin{document}

\ExplSyntaxOn
\newcommand{\fpeval}[1]{\fp_eval:n{#1}}
\ExplSyntaxOff

\newcommand{\DrawIterated}[7]% number of iterations
{ \ifnum#1=1%
  \draw[line join=round] (#2,#3) -- (#4,#5) -- (#6,#7) -- cycle;
  \else
  \draw[line join=round] (#2,#3) -- (#4,#5) -- (#6,#7) -- cycle;
  % Middle
  \DrawIterated{\numexpr#1-1\relax}%
  {\fpeval{(#2 + #4)/2}}%
  {\fpeval{(#3 + #5)/2}}%
  {\fpeval{(#4 + #6)/2}}%
  {\fpeval{(#5 + #7)/2}}%
  {\fpeval{(#2 + #6)/2}}%
  {\fpeval{(#3 + #7)/2}}

  % Bottom left
  \DrawIterated{\numexpr#1-1\relax}%
  {#2}%
  {#3}%
  {\fpeval{(#2 + #4)/2}}%
  {\fpeval{(#3 + #5)/2}}%
  {\fpeval{(#2 + #6)/2}}%
  {\fpeval{(#3 + #7)/2}}

  % Bottom right
  \DrawIterated{\numexpr#1-1\relax}%
  {\fpeval{(#2 + #4)/2}}%
  {\fpeval{(#3 + #5)/2}}%
  {#4}%
  {#5}%
  {\fpeval{(#4 + #6)/2}}%
  {\fpeval{(#5 + #7)/2}}%

  % Top
  \DrawIterated{\numexpr#1-1\relax}%
  {\fpeval{(#2 + #6)/2}}%
  {\fpeval{(#3 + #7)/2}}
  {\fpeval{(#4 + #6)/2}}%
  {\fpeval{(#5 + #7)/2}}%
  {#6}%
  {#7}
  
  \fi
}

\newcommand{\DrawIteratedBary}[7]% number of iterations
{ \ifnum#1=1%
  \draw[line join=round] (#2,#3) -- (#4,#5) -- (#6,#7) -- cycle;
  \coordinate (0) at (#2,#3);
  \coordinate (1) at (#4,#5);
  \coordinate (2) at (#6,#7);
  \coordinate (bc) at (barycentric cs:0=1,1=1,2=1);
  \draw[dashed, line join=round] (0) -- (bc) -- (1) (bc) -- (2);
  \else
  \draw[line join=round] (#2,#3) -- (#4,#5) -- (#6,#7) -- cycle;
  % Middle
  \DrawIteratedBary{\numexpr#1-1\relax}%
  {\fpeval{(#2 + #4)/2}}%
  {\fpeval{(#3 + #5)/2}}%
  {\fpeval{(#4 + #6)/2}}%
  {\fpeval{(#5 + #7)/2}}%
  {\fpeval{(#2 + #6)/2}}%
  {\fpeval{(#3 + #7)/2}}

  % Bottom left
  \DrawIteratedBary{\numexpr#1-1\relax}%
  {#2}%
  {#3}%
  {\fpeval{(#2 + #4)/2}}%
  {\fpeval{(#3 + #5)/2}}%
  {\fpeval{(#2 + #6)/2}}%
  {\fpeval{(#3 + #7)/2}}

  % Bottom right
  \DrawIteratedBary{\numexpr#1-1\relax}%
  {\fpeval{(#2 + #4)/2}}%
  {\fpeval{(#3 + #5)/2}}%
  {#4}%
  {#5}%
  {\fpeval{(#4 + #6)/2}}%
  {\fpeval{(#5 + #7)/2}}%

  % Top
  \DrawIteratedBary{\numexpr#1-1\relax}%
  {\fpeval{(#2 + #6)/2}}%
  {\fpeval{(#3 + #7)/2}}
  {\fpeval{(#4 + #6)/2}}%
  {\fpeval{(#5 + #7)/2}}%
  {#6}%
  {#7}
  
  \fi
}

\begin{tikzpicture}[scale=4]
  \foreach \i in {1, 2, 3} {
    \begin{scope}[shift={($(\i*1.5-1.5, 0)$)}]
      \DrawIterated{\i}{0}{0}{1}{0}{0.5}{\fpeval{sqrt(3)/2}}
    \end{scope}
    \begin{scope}[shift={($(\i*1.5-1.5, -1.5)$)}]
      \DrawIteratedBary{\i}{0}{0}{1}{0}{0.5}{\fpeval{sqrt(3)/2}};
    \end{scope}

    \draw[-stealth, very thick] ($(-1 + 1.5*\i, -0.1)$) -- +(0, -0.4);
    \ifnum\i=3
    \relax
    \else
    \draw[-stealth, very thick] ($(-0.5 + 1.5*\i, 0.5)$) -- +(0.5, 0);
    \fi
  }
\end{tikzpicture}
\end{document}
