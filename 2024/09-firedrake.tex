% -*- TeX-engine: luatex -*-

\documentclass[aspectratio=169]{beamer}
\usetheme{nvidia}

\usepackage{booktabs}
\title{The modern CUDA ecosystem}
\subtitle{how to talk to your GPU}
\author{Lawrence Mitchell}
\institute{\texttt{lmitchell@nvidia.com}}
\begin{document}

\maketitle

\begin{frame}
  \frametitle{Prehistory}
  \begin{quote}
    The contributions of this paper are: a high-performance
    implementation of a finite element solver for an
    advection-diffusion problem written using NVidia CUDA (Section 3),
    a prototype implementation of a compiler that generates CUDA code
    from UFL sources (Section 4.2), [\dots]
  \end{quote}
  {
    \scriptsize
    \raggedleft
    Markall, Ham, and Kelly.
    \emph{Towards generating optimised finite
      element solvers for GPUs from high-level specifications},
    Procedia Computer Science (2010)
    \par
    GTX 480, 1.5GB @ 177GB/s, 1.3TFLOP/s (32bit), 168GFLOP/s (64bit)
    \par
  }
\end{frame}
\begin{frame}
  \frametitle{Prehistory}
  \begin{quote}
    [W]e present finite element implementations that are
    written in both CUDA, for NVidia GPUs, and OpenCL, [\dots We] show
    that the point at which it becomes profitable to switch algorithms
    differs between target architectures. Using OpenCL achieves
    functional portability across different multicore and GPU
    platforms; we demonstrate that performance portability is more
    challenging to achieve.
  \end{quote}
  {
    \scriptsize
    \raggedleft
    Markall, Slemmer, Ham, Kelly, Cantwell, and Sherwin.
    \emph{Finite element assembly strategies on multi-core and many-core architectures},
    Int.~J.~Num.~Meth.~Fluids (2011)
    \par
    \vspace{\baselineskip}
    GTX 480, 1.5GB @ 177GB/s, 1.3TFLOP/s (32bit), 168GFLOP/s (64bit)
    \par
  }
\end{frame}

\begin{frame}
  \frametitle{\textcolor{black!40}{Pre}history}
  \begin{quote}
    PyOP2 provides an API for the implementation of target-specific
    backends. At present, backends for sequential C code, C with
    OpenMP, and CUDA code are fully supported.
  \end{quote}
  {
    \scriptsize
    \raggedleft
    Markall, Rathgeber, Mitchell, Loriant, Bertolli, Ham, and Kelly.
    \emph{Performance-portable finite element assembly using PyOP2 and FEniCS},
    ISC (2013)
    \par
    Tesla M2050, 3GB @ 148GB/s, 1TFLOP/s (32bit), 515GFLOP/s (64bit)
    \par
  }
\end{frame}

\begin{frame}
  \frametitle{History}
  \begin{quote}
    This paper presents only performance results [\dots] on CPUs
    [\dots The] available hybrid parallel and GPU linear solver
    libraries are far more limited than PETSc’s MPI-only functionality
    [\dots we] have therefore given priority to achieving high
    performance and feature completeness for the CPU backend using
    MPI.\@
  \end{quote}
  {
    \scriptsize
    \raggedleft
    Rathgeber, Ham, Mitchell, Lange, Luporini, McRae, Bercea, Markall,
    and Kelly. \emph{Firedrake: automating the finite element method
      by composing abstractions}, ACM TOMS (2016)
    \par
  }
\end{frame}

\begin{frame}
  \frametitle{History}
  \begin{quote}
    Remove all code generation backends except for the sequential CPU
    backend [\dots We]
    have now reached reasonable consensus that the existing
    implementations for GPUs are not the right place to start for
    high-performance assembly.
  \end{quote}
  {
    \scriptsize
    \raggedleft
    Mitchell. \url{https://github.com/OP2/PyOP2/pull/507}, (2016)
    \par
  }
\end{frame}

\begin{frame}
  \frametitle{What about today?}

  \begin{itemize}
  \item Data center GPUs have \emph{quite a lot} of RAM
    \begin{itemize}
    \item H100 PCIe: 80GB @ 2TB/s, 51.2TFLOP/s (32bit), 51.2TFLOP/s (64bit)
    \end{itemize}
  \item They are both easier and harder to program
    \begin{itemize}
    \item Compiler is better at finding parallelism
    \item Automatic caches
    \item Best perf/watt utilises \emph{tensor cores}
    \end{itemize}
  \end{itemize}
\end{frame}


\begin{frame}
  \frametitle{Tensor cores}
  \begin{columns}[t]
    \begin{column}{0.5\textwidth}
      \begin{block}{What}
        \begin{itemize}
        \item Specialised hardware for matrix-matrix accumulate: $D \gets A B + C$
        \item Zoo of different precisions
        \item \emph{Collective} across a warp
        \end{itemize}
      \end{block}
    \end{column}
    \begin{column}{0.5\textwidth}
      \begin{block}{Why}
        \begin{itemize}
        \item Addressing power wall
        \item One instruction does more work
        \item Data access more structured: less integer addressing
        \item Low precision: power cost of multiplication scales like
          square of mantissa
        \end{itemize}
      \end{block}
    \end{column}
  \end{columns}
\end{frame}

\begin{frame}
  \frametitle{Example: relative power}
  \begin{table}
    \centering
    \begin{tabular}{ccc}
      \toprule
      Precision & Operation & Relative energy/FLOP \\
      \midrule
      FP64 & FMA & 2.5 \\
      FP32 & FMA & 1.0 \\
      FP16 & FMA & 0.5 \\
      FP64 & MMA & 1.5 \\
      FP16 & MMA & 0.12 \\
      FP8 & MMA & 0.06 \\
      \bottomrule
    \end{tabular}
    \caption{Relative energy cost per FLOP for matrix-matrix multiply
      on Hopper}
  \end{table}
\end{frame}

\begin{frame}
  \frametitle{Pretend it's the 80s (60s?)}
  \begin{itemize}
  \item My algorithm needs 64bit/32bit precision, what do all these fancy
    formats help me?
  \item Iterative refinement everywhere!\\
    {\scriptsize
      \raggedleft
      Haidar, Bayraktar, Tomov, Dongarra, Higham, Proc. R. Soc. A (2020)
      \par}
  \item Or, emulate high-precision scalar with low-precision tensor\\
    {\scriptsize
      \raggedleft
      Hiroyuki, Katsuhisa, Yokota. Int.\~J.~HPCA (2024)
      \par}
  \end{itemize}
\end{frame}
\begin{frame}
  \frametitle{Do I have to do this all myself?}
  
\end{frame}

\begin{frame}
  \frametitle{Work with us}
  
\end{frame}
% \begin{abstract}
%   Firedrake emerged, in part, from projects looking at finite elements
%   on GPUs, but that was in the depths of history and we swiftly
%   pivoted to a CPU-centric approach.

%   A decade later, there is interest in revisiting this decision, and
%   both the hardware and software have moved on since the last time
%   Firedrake tried this in anger. As well as a much broader ecosystem
%   of GPU libraries, there is also renewed interest in making the core
%   programming experience harder to misuse. To that end, I'll present
%   some thoughts on device-side libraries and programming abstractions
%   for writing fast kernels correctly.
% \end{abstract}
\end{document}
