\documentclass[presentation]{beamer}
\usepackage{mathtools}
\usepackage{minted}
\usepackage{tikz}
\usetikzlibrary{positioning,calc}
\usetikzlibrary{shapes.geometric}
\usetikzlibrary{backgrounds}% only to show the bounding box
\usetikzlibrary{shapes,arrows}
\usepackage{pgfplots}
\usepackage{pgfplotstable}
\usetikzlibrary{pgfplots.groupplots}
\pgfplotsset{compat=1.12}
\usepackage{appendixnumberbeamer}
\usepackage{amsmath}
\date{}
\renewcommand{\vec}[1]{\ensuremath{\boldsymbol{#1}}}
\newcommand{\ddt}[1]{\frac{\partial #1}{\partial t}}
\newcommand{\zhat}{\hat{\vec{z}}}
\newcommand{\W}{\ensuremath{\mathbb{W}}}

\DeclareMathOperator{\grad}{grad}
\let\div\relax
\DeclareMathOperator{\div}{div}
\DeclareMathOperator{\curl}{curl}
\newcommand{\vsubset}[1]{\rotatebox[origin=c]{90}{\ensuremath{\subset}}}
\newcommand{\inner}[2]{\ensuremath{\langle #1, #2 \rangle}}
\author{Lawrence Mitchell\inst{1}}
\institute{
\inst{1}Departments of Computing and Mathematics, Imperial College
London
}

\usetheme{metropolis}
 
\begin{document}

\begin{frame}[standout]

  \url{www.firedrakeproject.org}

  Automated solution of partial differential equations

  \vspace{2em}
  {\normalsize
    Lawrence Mitchell

    \texttt{lawrence.mitchell@imperial.ac.uk}
    }
\end{frame}

\begin{frame}[fragile]
  \frametitle{The Cahn-Hilliard equation}
  Find $\phi, \mu \in V \times V$ s.t.
  \begin{equation*}
    \begin{aligned}
      \int\!\!(\phi^{n+1} - \phi^n) q\,\text{d}x + \Delta t D \int\!\!\nabla
      ((1 - \theta)\mu^{n} + \theta \mu^{n+1}) \cdot \nabla q\,\text{d}x &= 0 \quad \forall\, q \in V\\
      \int\!\!(\mu - \partial_\phi f) v\,\text{d}x - \gamma \int\!\!\nabla \phi\cdot \nabla v\,\text{d}x &= 0 \quad \forall\, v \in V\\
      n\cdot\nabla \phi = 0 \text{ and } n\cdot\nabla \mu &= 0 \quad\text{on $\partial\Omega$}
    \end{aligned}
  \end{equation*}

  \begin{columns}
    \begin{column}[t]{0.5\textwidth}
      \begin{minted}[frame=none,xleftmargin=0em,xrightmargin=0em,fontsize=\tiny,mathescape]{python}
from firedrake import *
mesh = UnitSquareMesh(100, 100)
V = FunctionSpace(mesh, "CG", 1)
W = V*V
gamma = Constant(0.005)
D = Constant(10)
q, v = TestFunctions(W)
u = Function(W)
u0 = Function(W)
phi, mu = split(u)
phi0, mu0 = split(u0)
phi = variable(phi)
f = 10*(phi**2 - 1)**2
dfdphi = diff(f, phi)
\end{minted}
\end{column}
\hspace{-4em}
\begin{column}[t]{0.5\textwidth}
 \begin{minted}[frame=none,xleftmargin=0em,xrightmargin=0em,fontsize=\tiny,mathescape]{python}
theta = 0.5
mu_theta = (1-theta)*mu0 + theta*mu
dt = 5e-6
F = ((phi - phi0)*q +
     dt*D*dot(grad(mu_theta), grad(q)) +
     (mu - dfdphi)*v -
     gamma*dot(grad(phi), grad(v)))*dx
for _ in range(200):
    u0.assign(u)
    solve(F == 0, u)
\end{minted}
\end{column}
\end{columns}
  
\end{frame}
\end{document}
