\documentclass[presentation]{beamer}

\usepackage{tikz}
\usetikzlibrary{positioning,calc}
\usetikzlibrary{shapes.geometric}
\usetikzlibrary{backgrounds}% only to show the bounding box
\usetikzlibrary{shapes,arrows}
\usepackage{pgfplots}
\usepackage{pgfplotstable}
\usetikzlibrary{pgfplots.groupplots}
\pgfplotsset{compat=1.12}
\usepackage{appendixnumberbeamer}
\usepackage{amsmath}
\date{28th March 2017}
\usetheme{firedrake}

\renewcommand{\vec}[1]{\ensuremath{\boldsymbol{#1}}}
\newcommand{\ddt}[1]{\frac{\partial #1}{\partial t}}
\newcommand{\zhat}{\hat{\vec{z}}}
\newcommand{\W}{\ensuremath{\mathbb{W}}}

\newcommand{\inner}[1]{\left\langle #1 \right \rangle}

\newcommand{\KSP}[2]{\ensuremath{\mathcal{K}\left(#1, \mathbb{#2}\right)}}
\newcommand{\ksp}[1]{\KSP{#1}{#1}}

\newcommand{\highlight}[1]{\colorbox{red!20}{\color{black} #1}}

\author{Lawrence Mitchell\inst{1,*}}
\institute{
\inst{1}Departments of Computing and Mathematics, Imperial College
London

\inst{*}\texttt{lawrence.mitchell@imperial.ac.uk}
}

\graphicspath{{./\jobname.figures/}}

\usepackage[url=false,
            doi=true,
            isbn=false,
            style=authoryear,
            firstinits=true,
            uniquename=init,
            backend=biber]{biblatex}

\setbeamertemplate{bibliography item}{}
\renewcommand{\bibfont}{\footnotesize}
\addbibresource{references.bib}

\setlength{\bibitemsep}{1ex}
\setlength{\fboxsep}{1pt}

\renewbibmacro{in:}{}
\DeclareFieldFormat[article]{volume}{\textbf{#1}}
\DeclareFieldFormat{doi}{%
  doi\addcolon%
  {\scriptsize\ifhyperref{\href{http://dx.doi.org/#1}{\nolinkurl{#1}}}
    {\nolinkurl{#1}}}}
\AtEveryBibitem{%
\clearfield{pages}%
\clearfield{issue}%
\clearfield{number}%
}

\usepackage{minted}

\title{Firedrake: symbolic numerical computing}
\begin{document}

\maketitle

\begin{abstract}
  One of the great and enduring successes of numerical computing is
  the capturing of mathematical abstractions in software.  This allows
  the programmer to express the intent of their code, without
  specifying its low-level implementation.  We then automate the
  synthesis of efficient code by using (or writing) compilers.

  In the context of solving numerical PDEs, the finite element method
  is particularly amenable to this approach.  For many problems, the
  choice of discretisation completely specifies the mathematical
  intent.  A symbolic description of the PDE can then be manipulated
  by a domain-specific compiler to produce a high-performance
  implementation.  I this talk, I present Firedrake
  (www.firedrakeproject.org), a concrete realisation of this idea.  I
  will discuss how capturing and exploiting symbolic structure in
  numerical software greatly simplifies model development while
  simultaneously permitting the synthesis of a high performance
  implementation.  I will illustrate with examples from high-order
  finite element methods, and block preconditioning of multiphysics
  problems.
\end{abstract}

\begin{frame}
  Finite element 101

  Introduce UFL as symbolic description language

  Weave in problem specific data

  Introduce Firedrake as a concrete example

  What does the symbolic description bring us?

  Not just pretty syntax, but we can teach the computer to reason
  about the model.

  We can synthesise efficient implementations from this problem
  description.

  Including applying optimisations that are tricky to get right by
  hand (sum factorisation example)

  Weaving together symbolics also allows usntructured tiling

  Composing with linear algebra, reversing the one-way street.

  Complex preconditioning, by composing the pieces we already have.
\end{frame}
\appendix
\begin{frame}[t,allowframebreaks]
  \frametitle{References}
  \printbibliography[heading=none]
\end{frame}
\end{document}
