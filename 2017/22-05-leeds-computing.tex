\documentclass[presentation]{beamer}

\usepackage{tikz}
\usetikzlibrary{positioning,calc}
\usetikzlibrary{shapes.geometric}
\usetikzlibrary{backgrounds}% only to show the bounding box
\usetikzlibrary{shapes,arrows}
\usepackage{pgf}
\usepackage{appendixnumberbeamer}
\usepackage{amsmath}
\date{28th March 2017}
\usetheme{firedrake}

\renewcommand{\vec}[1]{\ensuremath{\boldsymbol{#1}}}
\newcommand{\ddt}[1]{\frac{\partial #1}{\partial t}}
\newcommand{\zhat}{\hat{\vec{z}}}
\newcommand{\W}{\ensuremath{\mathbb{W}}}

\newcommand{\inner}[1]{\left\langle #1 \right \rangle}

\newcommand{\KSP}[2]{\ensuremath{\mathcal{K}\left(#1, \mathbb{#2}\right)}}
\newcommand{\ksp}[1]{\KSP{#1}{#1}}

\newcommand{\highlight}[1]{\colorbox{red!20}{\color{black} #1}}

\author{Lawrence Mitchell\inst{1,*}}
\institute{
\inst{1}Departments of Computing and Mathematics, Imperial College
London

\inst{*}\texttt{lawrence.mitchell@imperial.ac.uk}
}

\graphicspath{{./\jobname.figures/}}

\usepackage[url=false,
            doi=true,
            isbn=false,
            style=authoryear,
            firstinits=true,
            uniquename=init,
            backend=biber]{biblatex}

\setbeamertemplate{bibliography item}{}
\renewcommand{\bibfont}{\footnotesize}
\addbibresource{references.bib}

\setlength{\bibitemsep}{1ex}
\setlength{\fboxsep}{1pt}

\renewbibmacro{in:}{}
\DeclareFieldFormat[article]{volume}{\textbf{#1}}
\DeclareFieldFormat{doi}{%
  doi\addcolon%
  {\scriptsize\ifhyperref{\href{http://dx.doi.org/#1}{\nolinkurl{#1}}}
    {\nolinkurl{#1}}}}
\AtEveryBibitem{%
\clearfield{pages}%
\clearfield{issue}%
\clearfield{number}%
}

\usepackage{minted}

\title{Firedrake: symbolic numerical computing}
\begin{document}

\maketitle

\begin{abstract}
  One of the great and enduring successes of numerical computing is
  the capturing of mathematical abstractions in software.  This allows
  the programmer to express the intent of their code, without
  specifying its low-level implementation.  We then automate the
  synthesis of efficient code by using (or writing) compilers.

  In the context of solving numerical PDEs, the finite element method
  is particularly amenable to this approach.  For many problems, the
  choice of discretisation completely specifies the mathematical
  intent.  A symbolic description of the PDE can then be manipulated
  by a domain-specific compiler to produce a high-performance
  implementation.  I this talk, I present Firedrake
  (www.firedrakeproject.org), a concrete realisation of this idea.  I
  will discuss how capturing and exploiting symbolic structure in
  numerical software greatly simplifies model development while
  simultaneously permitting the synthesis of a high performance
  implementation.  I will illustrate with examples from high-order
  finite element methods, and block preconditioning of multiphysics
  problems.
\end{abstract}

\begin{frame}
  \frametitle{Finite element crash course}
  \begin{align*}
    F(u) &= 0 \text{ in $\Omega$}\\
    u &= g \text{ on $\Gamma_1$}\\
    \frac{\partial u}{\partial n} &= h \text{ on $\Gamma_2$}
  \end{align*}
  Seek \emph{weak} solution in some space of functions $V(\Omega)$.

  Now we need to solve the (infinite dimensional) problem, find $u\in V$ s.t.
  \begin{equation*}
    \int_\Omega \!F(u) v\, \text{d}x = 0 \quad \forall\, v \in V
  \end{equation*}
\end{frame}
\begin{frame}
  Choose finite dimensional $V_h \subset V$, and seek a solution in
  that subspace: find $u_h \in V_h$ s.t.
  \begin{equation*}
    \int_\Omega \!F(u_h) v_h\, \text{d}x = 0 \quad \forall\, v_h \in V_h
  \end{equation*}
\end{frame}
\begin{frame}
  \begin{overprint}
    \only<1>{Divide domain $\Omega$\dots
    \begin{center}
      \begin{tikzpicture}
        \draw[very thick, line cap=rect] (0,0) -- (5, 0) (0, 0) arc
        (180:360:2.5);
      \end{tikzpicture}
    \end{center}}
  \only<2>{\dots{}into triangulation $\mathcal{T}$\dots
    \begin{center}
        \begin{tikzpicture}
          \path (0,0) arc[radius=2.5, start angle=180, end angle=360]
          node[name=E,pos=0,swap] {} node[name=F,pos=0.25,swap] {}
          node[name=G,pos=0.5,swap] {} node[name=H,pos=0.82,swap] {}
          node[name=I,pos=1,swap] {}; \node (A) at (2.5, 0) {}; \node
          (B) at (1.4, -0.7) {}; \node (C) at (3.4, -1.2) {}; \node
          (D) at (1.8, -1.5) {};

          \draw[color=black, very thick, line cap=butt, line
          join=round] (E.center) -- (A.center) -- (I.center) --
          (H.center) -- (G.center) -- (F.center) -- (E.center) --
          cycle; \draw[color=black, very thick, line cap=butt, line
          join=round] (E.center) -- (B.center) -- (D.center) --
          (F.center) -- (B.center); \draw[color=black, very thick,
          line cap=butt, line join=round] (G.center) -- (D.center) --
          (C.center) -- (G.center); \draw[color=black, very thick,
          line cap=butt, line join=round] (B.center) -- (A.center) --
          (C.center) -- (B.center); \draw[color=black, very thick,
          line cap=butt, line join=round] (H.center) -- (C.center) --
          (I.center);
        \end{tikzpicture}
    \end{center}
  }
  \only<3>{\dots{}and choose basis with finite support.
    \begin{center}
        \begin{tikzpicture}
          \path (0,0) arc[radius=2.5, start angle=180, end angle=360]
          node[name=E,pos=0,swap] {} node[name=F,pos=0.25,swap] {}
          node[name=G,pos=0.5,swap] {} node[name=H,pos=0.82,swap] {}
          node[name=I,pos=1,swap] {}; \node (A) at (2.5, 0) {}; \node
          (B) at (1.4, -0.7) {}; \node (C) at (3.4, -1.2) {}; \node
          (D) at (1.8, -1.5) {};

        \path[fill=gray!50] (E.center) -- (A.center) -- (B.center) --
        (F.center) --cycle;
          \draw[color=black, very thick, line cap=butt, line
          join=round] (E.center) -- (A.center) -- (I.center) --
          (H.center) -- (G.center) -- (F.center) -- (E.center) --
          cycle; \draw[color=black, very thick, line cap=butt, line
          join=round] (E.center) -- (B.center) -- (D.center) --
          (F.center) -- (B.center); \draw[color=black, very thick,
          line cap=butt, line join=round] (G.center) -- (D.center) --
          (C.center) -- (G.center); \draw[color=black, very thick,
          line cap=butt, line join=round] (B.center) -- (A.center) --
          (C.center) -- (B.center); \draw[color=black, very thick,
          line cap=butt, line join=round] (H.center) -- (C.center) --
          (I.center);
        \end{tikzpicture}
      \end{center}
      }
  \end{overprint}
\end{frame}

\begin{frame}
  Integrals become sum over element integrals
  \begin{equation*}
    \int_\Omega\! F(u_h) v_h \, \text{d}x =
    \sum_{e \in \mathcal{T}} \int_e\! F(u_h)v_h\, \text{d}x
  \end{equation*}

  (Usually) perform element integrals with numerical quadrature
  \begin{equation*}
    \int_e F(u_h)v_h\,\text{d}x = \sum_q w_q F(u_h(q)) v_h(q)\,\text{d}x
  \end{equation*}

  Typically, this is done by geometrically transforming from each
  \emph{physical} element to a \emph{reference} element.
\end{frame}

\begin{frame}
  \frametitle{Abstractly}
  \begin{itemize}
  \item Mathematics says ``here is the integral to compute on each
    element, do that everywhere''
  \item Doesn't specify \emph{how} to compute the integral
  \item Doesn't specify \emph{how} to gather the element contributions
  \end{itemize}
\end{frame}

\begin{frame}
  \frametitle{Specifying element integrals}

  Traditional software libraries for finite element computations give you
  \begin{itemize}
  \item methods for computing numerical quadrature
  \item methods to evaluate basis functions at quadrature points
  \item methods to evaluate fields at quadrature points
  \item methods to compute geometric transformations
  \end{itemize}
\end{frame}
\begin{frame}[fragile]
  You might write code like this.

\begin{minted}[fontsize=\tiny]{cpp}
template<typename EG, typename LFSU, typename X, typename LFSV, typename M>
void jacobian_volume(const EG& eg, const LFSU& lfsu, const X& x, 
                     const LFSV& lfsv, M& mat) const {
  const auto geo = eg.geometry();
  const auto S = geo.jacobianInverseTransposed(qp);
  RF factor = weight*geo.integrationElement(qp);
  double grad[dim][n] = {{0.0}};
  for (int i=0; i<dim; i++)
    for (int k=0; k<dim; k++)
      for (int j=0; j<n; j++)
        grad[i][j] += S[i][k] * gradhat[k][j];
  double A[n][n] = {{0.0}};
  for (int i=0; i<n; i++)
    for (int k=0; k<dim; k++)
      for (int j=0; j<n; j++)
        A[i][j] += grad[k][i]*grad[k][j];
  for (int i=0; i<n; i++)
    for (int j=0; j<n; j++)
      mat.accumulate(lfsu,i,lfsu,j,A[i][j]*factor);
}
\end{minted}
\end{frame}

\begin{frame}
  Finite element 101

  Introduce UFL as symbolic description language

  Weave in problem specific data

  Introduce Firedrake as a concrete example

  What does the symbolic description bring us?

  Not just pretty syntax, but we can teach the computer to reason
  about the model.

  We can synthesise efficient implementations from this problem
  description.

  Including applying optimisations that are tricky to get right by
  hand (sum factorisation example)

  Weaving together symbolics also allows usntructured tiling

  Composing with linear algebra, reversing the one-way street.

  Complex preconditioning, by composing the pieces we already have.
\end{frame}
\appendix
\begin{frame}[t,allowframebreaks]
  \frametitle{References}
  \printbibliography[heading=none]
\end{frame}
\end{document}
