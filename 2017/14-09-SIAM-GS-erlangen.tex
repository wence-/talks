\documentclass[presentation]{beamer}

\usepackage{tikz}
\usetikzlibrary{positioning,calc}
\usetikzlibrary{shapes.geometric}
\usetikzlibrary{backgrounds}% only to show the bounding box
\usetikzlibrary{shapes,arrows}
\usepackage{pgfplots}
\usepackage{pgfplotstable}
\usetikzlibrary{pgfplots.groupplots}
\pgfplotsset{compat=1.12}
\usepackage{appendixnumberbeamer}
\usepackage{amsmath}
\date{27th March 2017}
\usetheme{firedrake}

\pgfplotscreateplotcyclelist{decent cycle}{%
  {blue, mark=*, mark options={fill=blue},
    mark size=2pt},
  {cyan, mark=square*, mark options={fill=cyan},
    mark size=2pt},
  {magenta, mark=triangle*, mark options={fill=magenta},
    mark size=3pt},
  {blue, mark=*, mark options={fill=blue},
    mark size=2pt},
  {cyan, mark=square*, mark options={fill=cyan},
    mark size=2pt},
  {magenta, mark=triangle*, mark options={fill=magenta},
    mark size=3pt},
}

\pgfplotsset{
  decent/.style={
    cycle list name=decent cycle,
  }
}
\renewcommand{\vec}[1]{\ensuremath{\boldsymbol{#1}}}
\newcommand{\ddt}[1]{\frac{\partial #1}{\partial t}}
\newcommand{\zhat}{\hat{\vec{z}}}
\newcommand{\W}{\ensuremath{\mathbb{W}}}

\DeclareMathOperator{\grad}{grad}
\let\div\relax
\DeclareMathOperator{\div}{div}
\DeclareMathOperator{\curl}{curl}
\newcommand{\vsubset}[1]{\rotatebox[origin=c]{90}{\ensuremath{\subset}}}
\newcommand{\inner}[2]{\ensuremath{\langle #1, #2 \rangle}}
\author{Lawrence Mitchell\inst{1} \and Firedrake team}
\institute{
\inst{1}Departments of Computing and Mathematics, Imperial College
London
}

\graphicspath{{./\jobname.figures/}}

\newcommand{\arxivlink}[2]{%
  \href{http://www.arxiv.org/abs/#1}%
  {{\small\texttt{arXiv:\,#1\,[#2]}}}%
}
\newcommand{\doilink}[1]{%
  \href{http://dx.doi.org/#1}%
  {{\small\texttt{doi:\,#1}{}}}%
}
\usepackage[url=false,
            doi=true,
            isbn=false,
            style=authoryear,
            firstinits=true,
            uniquename=init,
            backend=biber]{biblatex}

\setbeamertemplate{bibliography item}{}
\renewcommand{\bibfont}{\footnotesize}
\addbibresource{references.bib}

\setlength{\bibitemsep}{1ex}

\renewbibmacro{in:}{}
\DeclareFieldFormat[article]{volume}{\textbf{#1}}
\DeclareFieldFormat{doi}{%
  doi\addcolon%
  {\scriptsize\ifhyperref{\href{http://dx.doi.org/#1}{\nolinkurl{#1}}}
    {\nolinkurl{#1}}}}
\AtEveryBibitem{%
\clearfield{pages}%
\clearfield{issue}%
\clearfield{number}%
}

\usepackage{minted}

\title{Firedrake: composable abstractions for high performance finite
  element computations}

\begin{document}

\maketitle

\begin{abstract}
  The development of complex numerical models requires a variety of
  skills.  Including, but not limited to, problem-specific knowledge,
  numerical methods, software engineering, and parallel computing.
  Polymaths that tick all of these boxes are rare.  To combat this
  complexity, traditional model design employs a separation of
  concerns using software libraries.  This separation is
  horizontal, and works best when the granularity of the API is
  large, and one-way.  Finite element computations, that contain
  user-specific variability in the inner loop, seem to preclude
  such an approach.

  In this talk, I will describe how, by teaching computers to
  manipulate mathematical descriptions of PDE problems, we address
  this problem, providing high performance finite element computations
  without requiring that the model developer be an expert low-level
  code optimisation.

  With an efficient model, we also need efficient solvers, and I will
  also discuss recent work in Firedrake to simplify the development of
  runtime-configurable block preconditioners using PETSc.
\end{abstract}

\appendix
\begin{frame}[t]
  \frametitle{References}
  \printbibliography[heading=none]
\end{frame}
\end{document}
