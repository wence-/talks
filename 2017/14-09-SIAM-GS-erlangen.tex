\documentclass[presentation]{beamer}

\usepackage{tikz}
\usetikzlibrary{positioning,calc}
\usetikzlibrary{shapes.geometric}
\usetikzlibrary{backgrounds}% only to show the bounding box
\usetikzlibrary{shapes,arrows}
\usepackage{pgfplots}
\usepackage{pgfplotstable}
\usetikzlibrary{pgfplots.groupplots}
\usepackage{appendixnumberbeamer}
\usepackage{amsmath}
\DeclareMathOperator{\tr}{tr}
\date{12th September 2017}
\usetheme{firedrake}

\pgfplotscreateplotcyclelist{decent cycle}{%
  {blue, mark=*, mark options={fill=blue},
    mark size=2pt},
  {cyan, mark=square*, mark options={fill=cyan},
    mark size=2pt},
  {magenta, mark=triangle*, mark options={fill=magenta},
    mark size=3pt},
  {blue, mark=*, mark options={fill=blue},
    mark size=2pt},
  {cyan, mark=square*, mark options={fill=cyan},
    mark size=2pt},
  {magenta, mark=triangle*, mark options={fill=magenta},
    mark size=3pt},
}

\pgfplotsset{
  decent/.style={
    cycle list name=decent cycle,
  }
}
\newcommand{\colourfiredrake}[1]{\colorbox{red!20}{#1}}
\newcommand{\colourpetsc}[1]{\colorbox{blue!20}{#1}}
\renewcommand{\vec}[1]{\ensuremath{\boldsymbol{#1}}}
\newcommand{\ddt}[1]{\frac{\partial #1}{\partial t}}
\newcommand{\zhat}{\hat{\vec{z}}}
\newcommand{\W}{\ensuremath{\mathbb{W}}}

\DeclareMathOperator{\grad}{grad}
\let\div\relax
\DeclareMathOperator{\div}{div}
\DeclareMathOperator{\curl}{curl}
\newcommand{\vsubset}[1]{\rotatebox[origin=c]{90}{\ensuremath{\subset}}}
\newcommand{\inner}[2]{\ensuremath{\langle #1, #2 \rangle}}
\author{Lawrence Mitchell\inst{1} $\in\text{ Firedrake team}$}
\institute{
\inst{1}Department of Computing and Department of Mathematics, Imperial College
London
}

\graphicspath{{./\jobname.figures/}{../pictures/}}

\newcommand{\arxivlink}[2]{%
  \href{http://www.arxiv.org/abs/#1}%
  {\texttt{arXiv:\,#1\,[#2]}}%
}
\newcommand{\doilink}[1]{%
  \href{http://dx.doi.org/#1}%
  {\texttt{doi:\,#1}}%
}
\usepackage[url=false,
            doi=true,
            isbn=false,
            style=authoryear,
            maxnames=10,
            firstinits=true,
            uniquename=init,
            backend=biber]{biblatex}

\setbeamertemplate{bibliography item}{}
\renewcommand{\bibfont}{\fontsize{8}{8}\selectfont}
\addbibresource{references.bib}

\setlength{\bibitemsep}{1ex}
\setlength{\fboxsep}{1pt}

\renewbibmacro{in:}{}
\DeclareFieldFormat[article]{volume}{\textbf{#1}}
\DeclareFieldFormat{doi}{%
  doi\addcolon%
  \ifhyperref{\href{http://dx.doi.org/#1}{\nolinkurl{#1}}}
  {\nolinkurl{#1}}}
\AtEveryBibitem{%
\clearfield{pages}%
\clearfield{issue}%
\clearfield{number}%
}

\usepackage{minted}

\title{Firedrake: composable abstractions for high performance finite
  element computations}

\begin{document}

\maketitle

% \begin{abstract}
%   The development of complex numerical models requires a variety of
%   skills.  Including, but not limited to, problem-specific knowledge,
%   numerical methods, software engineering, and parallel computing.
%   Polymaths that tick all of these boxes are rare.  To combat this
%   complexity, traditional model design employs a separation of
%   concerns using software libraries.  This separation is
%   horizontal, and works best when the granularity of the API is
%   large, and one-way.  Finite element computations, that contain
%   user-specific variability in the inner loop, seem to preclude
%   such an approach.

%   In this talk, I will describe how, by teaching computers to
%   manipulate mathematical descriptions of PDE problems, we address
%   this problem, providing high performance finite element computations
%   without requiring that the model developer be an expert low-level
%   code optimisation.

%   With an efficient model, we also need efficient solvers, and I will
%   also discuss recent work in Firedrake to simplify the development of
%   runtime-configurable block preconditioners using PETSc.
% \end{abstract}
\setbeamertemplate{background}{}
\setbeamercolor{footline}{
  use=normal text,
  fg=normal text.fg
}

\begin{frame}
  \frametitle{Firedrake \url{www.firedrakeproject.org}}

  \begin{quote}
    {\normalfont [\ldots]} an automated system for the solution of partial
    differential equations using the finite element method.
  \end{quote}

  \begin{itemize}
  \item Written in Python.
  \item Finite element problems specified with \emph{embedded} domain
    specific language, UFL \parencite{Alnaes:2014} from the FEniCS project.
  \item \emph{Runtime} compilation to low-level (C) code.
  \item Explicitly \emph{data parallel}.
  \end{itemize}

  \begin{flushright}
    {\scriptsize F.~Rathgeber, D.A.~Ham, \textbf{LM}, M.~Lange,
      F.~Luporini, A.T.T.~McRae, G.-T.~Bercea, G.R.~Markall,
      P.H.J.~Kelly. TOMS,
      2016. \arxivlink{1501.01809}{cs.MS}\nocite{Rathgeber:2016}}
  \end{flushright}
\end{frame}

\begin{frame}[fragile]
  \frametitle{A DSL for finite element computations}
  \begin{columns}
    \begin{column}{0.47\framewidth}
      {\small
        Find $(u, p, T) \in V\times W\times Q$ s.t.
        \begin{align*}
          \int\!\nabla u \cdot \nabla v + (u \cdot \nabla u) \cdot v \\
          - p\nabla\cdot v + \frac{\text{Ra}}{\text{Pr}} Tg \hat{z} \cdot v\,\text{d}x &= 0 \\
          \int\!\nabla\cdot u q\,\text{d}x&= 0\\
          \int\! (u\cdot \nabla T) S + \text{Pr}^{-1} \nabla T \cdot \nabla
          S\,\text{d}x &= 0\\
          \quad \forall\, (v,q,T) \in V\times W \times Q
        \end{align*}
        }
    \end{column}
    \begin{column}{0.52\framewidth}
\begin{minted}[fontsize=\tiny]{python}
from firedrake import *
mesh = Mesh(...)
V = VectorFunctionSpace(mesh, "CG", 2)
W = FunctionSpace(mesh, "CG", 1)
Q = FunctionSpace(mesh, "CG", 1)
Z = V * W * Q
Ra = Constant(200)
Pr = Constant(6.18)
upT = Function(Z)
u, p, T = split(upT)
v, q, S = TestFunctions(Z)
bcs = [...] # no-flow + temp gradient
nullspace = MixedVectorSpaceBasis(
   Z, [Z.sub(0), VectorSpaceBasis(constant=True),
       Z.sub(2)])
F = (inner(grad(u), grad(v))
     + inner(dot(grad(u), u), v)
     - inner(p, div(v))
     + (Ra/Pr)*inner(T*g, v)
     + inner(div(u), q)
     + inner(dot(grad(T), u), S)
     + (1/Pr) * inner(grad(T), grad(S)))*dx

solve(F == 0, upT, bcs=bcs, nullspace=nullspace)
\end{minted}
    \end{column}
  \end{columns}  
\end{frame}

\begin{frame}
  \frametitle{Automating expertise}
  \begin{lemma}
    Most research groups do not have the expertise to produce high
    performance simulations.
  \end{lemma}
  \begin{corollary}
    If we want high performance expertise to be available to all model
    developers, we need a way of scaling the expertise.

    \uncover<2->{In Firedrake, we do this by synthesising
    \uncover<3->{\alert{hopefully}} efficient code with
    domain-specific compilers.}
  \end{corollary}
\end{frame}

\begin{frame}
  \frametitle{Two stages of compilation}
  \begin{block}{Local kernels: TSFC}
    Synthesise \emph{element local} kernel from weak form.
  \end{block}
  \begin{block}{Global iteration: PyOP2}
    Weave together \emph{local kernel} with global iteration over some
    set of mesh entities (e.g.\@ cells, exterior facets).
  \end{block}
\end{frame}

\begin{frame}
  \frametitle{Global iteration}
  \begin{block}{Local computation}
    Kernels do not know about global data layout.
    \begin{itemize}
    \item Kernel defines contract on local, packed, ordering.
    \item Global-to-local reordering/packing applied by runtime library.
    \end{itemize}
  \end{block}
  \begin{block}{Data parallel API}
    Application code does not specify explicit iteration order.
    \begin{itemize}
    \item Define data structures, then just ``iterate''
    \item Lazy evaluation, permits loop tiling and fusion without
      changing application code.
    \end{itemize}
  \end{block}
\end{frame}

\begin{frame}
  \frametitle{Maintainability}
  With good abstractions, you write little code.
  \begin{overlayarea}{\textwidth}{0.8\textheight}
    \begin{onlyenv}<1>
      \begin{block}{Library usability}
        \begin{itemize}
        \item High-level language enables rapid model development
        \item Ease of experimentation
        \item Small model code base
        \end{itemize}
      \end{block}

      \begin{block}{Library development}
        \begin{itemize}
        \item Automation of complex optimisations
        \item Exploit expertise across disciplines
        \item Small library code base
        \end{itemize}
      \end{block}
    \end{onlyenv}
    \begin{onlyenv}<2>
      \begin{columns}
        \begin{column}[t]{0.5\textwidth}
          \begin{block}{Core Firedrake}
            \begin{table}
              \centering
              \begin{tabular}{lc}
                Component & LOC   \\
                \hline
                Firedrake & 12000 \\
                PyOP2     & 5200  \\
                TSFC      & 4000  \\
                finat     & 1300   \\
                \hline
                Total     & 22500
              \end{tabular}
            \end{table}
          \end{block}
        \end{column}
        \begin{column}[t]{0.5\textwidth}
          \begin{block}{Shared with FEniCS}
            \begin{table}
              \centering
              \begin{tabular}{lc}
                Component & LOC   \\
                \hline
                FIAT      & 4000  \\
                UFL       & 13000 \\
                \hline
                Total     & 17000
              \end{tabular}
            \end{table}        
          \end{block}
        \end{column}
      \end{columns}
    \end{onlyenv}
    \begin{onlyenv}<3>
      \begin{columns}
        \begin{column}[t]{0.5\textwidth}
          \begin{block}{Thetis}
            {\small
              {\scriptsize\url{github.com/thetisproject/thetis}}
              \begin{itemize}
              \item 3D unstructured coastal ocean model written with
                Firedrake
              \item 7000 LOC
              \item 4-8x faster than previous code in group (same numerics)
              \end{itemize}
            }
          \end{block}
        \end{column}
        \begin{column}[t]{0.5\textwidth}
          \begin{block}{Gusto}
            {\small
              {\scriptsize\url{www.firedrakeproject.org/gusto/}}
              \begin{itemize}
              \item 3D atmospheric dynamical core using compatible FE
              \item Implements Met Office ``Gung Ho'' numerics
              \item 2000 LOC
              \end{itemize}
            }
          \end{block}
        \end{column}
      \end{columns}
    \end{onlyenv}
  \end{overlayarea}
\end{frame}

\begin{frame}[fragile]
  \frametitle{Compiling finite element kernels}
  Transforming an element integral to reference space.
  \begin{equation*}
    \int_e u v\, \text{d}x \rightarrow \int_e \tilde{u} \tilde{v} \det{\frac{\partial x}{\partial
        X}}\,\text{d} X
  \end{equation*}
  With quadrature
  \begin{equation*}
    \sum_{q,i,j} w_q E_{q,i} E_{q,j} \det{%
      \begin{bmatrix}
        \sum_r C_{q,r}^{1}c_r & \sum_r C_{q,r}^{1}c_r \\
        \sum_r C_{q,r}^{2}c_r & \sum_r C_{q,r}^{2}c_r
      \end{bmatrix}
    }
  \end{equation*}
  \begin{itemize}
  \item $E$ the tabulation matrix of the finite element of $\tilde{v}$.
  \item $c$ the vector of basis function coefficients of the
    coordinate field.
  \item $C^a$ the tabulation matrix of the first derivative of the
    $a$th component of the coordinate element.
  \end{itemize}
\end{frame}
\begin{frame}
  \frametitle{Compiling finite element kernels}
  \begin{equation*}
    \sum_{q, i, j} w_q E_{q,i} E_{q,j} \det{%
      \begin{bmatrix}
        \sum_r C_{q,r}^{1}c_r & \sum_r C_{q,r}^{1}c_r \\
        \sum_r C_{q,r}^{2}c_r & \sum_r C_{q,r}^{2}c_r
      \end{bmatrix}
    }
  \end{equation*}

  \begin{itemize}
  \item Na\"ive code generation transforms this tensor algebra
    expression into low-level C code.
  \item But, there might be opportunities for optimisation.
  \item For example, if $C_{q,r} = C_r$ (affine coordinates)
  \begin{equation*}
    \det{%
      \begin{bmatrix}
        \sum_r C_{r}^{1}c_r & \sum_r C_{r}^{1}c_r \\
        \sum_r C_{r}^{2}c_r & \sum_r C_{r}^{2}c_r
      \end{bmatrix}
    }\sum_{q, i, j} w_q E_{q,i} E_{q,j}
  \end{equation*}
\item Others available, depending on structure in $E$, $C$, $\{q\}$.
  \end{itemize}
\end{frame}

\begin{frame}
  \frametitle{Optimisations in TSFC}
  \begin{block}{Generic}
    \begin{itemize}
    \item Flop reduction via factorisation, code motion, and CSE.
      \begin{flushright}
        {\scriptsize
        Luporini, Ham, and Kelly.  TOMS
        2017. \arxivlink{1604.05872}{cs.MS}\nocite{Luporini:2017}}
      \end{flushright}
    \item Alignment and padding for vectorisation (either intrinsics
      or rely on C compiler).
      \begin{flushright}
        {\scriptsize
        Luporini, Varbanescu, et al. TACO
        2015. \doilink{10.1145/2687415}\nocite{Luporini:2015}}
      \end{flushright}
    \end{itemize}
  \end{block}
  \begin{block}{Structured basis}
    \begin{itemize}
    \item Structure (e.g.\@ tensor products) preserved in intermediate
      representation in TSFC, enables new optimisation passes.
      \begin{flushright}
        {\scriptsize Homolya, \textbf{LM}, Luporini, Ham. \arxivlink{1705.03667}{cs.MS}\nocite{Homolya:2017}}
      \end{flushright}
    \item Sum factorisation and spectral underintegration
      \begin{flushright}
        {\scriptsize Homolya, Kirby, Ham. In preparation.}
      \end{flushright}
    \end{itemize}
  \end{block}
\end{frame}

\begin{frame}
  \frametitle{Sum factorisation}
  On many elements, there is \emph{structure} in the basis functions:
  \begin{equation*}
    \phi_{i,q} := \phi_{(j,k),(p,r)} = \varphi_{j,p}\varphi_{k,r}
  \end{equation*}
  and so
  \begin{align*}
    \mathcal{F}_{(p,r)} &= \sum_{j,k} \phi_{(j,k),(p,r)} f_{j,k} \\
                        &= \sum_{j,k} \varphi_{j,p}\varphi_{k,r} f_{j,k} \\
                        &= \sum_j \varphi_{j,p} \sum_k \varphi_{k,r} f_{j,k}
  \end{align*}
  Automating this transformation is enabled by providing
  \emph{symbolic code} for the tabulation
  $\varphi_{j,p}\varphi_{k,r}$.
\end{frame}

\begin{frame}
  \frametitle{Sum factorisation II}
  \begin{itemize}
  \item Improves complexity $\mathcal{O}((p+1)^{d-1})$-fold.
  \item Gives \emph{optimal complexity} evaluation for matrix
    assembly, matrix-vector products, and residual evaluation.
  \item For a degree $p$ approximation on a $d$-dimensional tensor
    product cell we have
    \vskip\baselineskip
    \begin{tabular}{l|c|c|c}
      Method              & Build operator            & MatVec                 & Mem refs              \\
                          & (FLOPs)                   & (FLOPs)                & (bytes)               \\
      \hline

      Na\"ive assembled   & $\mathcal{O}(p^{3d})$     & $\mathcal{O}(p^{2d})$  & $\mathcal{O}(p^{2d})$ \\ 
      SF assembled        & $\mathcal{O}(p^{2d + 1})$ & $\mathcal{O}(p^{2d})$  & $\mathcal{O}(p^{2d})$ \\
      Na\"ive matrix free & 0                         & $\mathcal{O}(p^{2d})$  & $\mathcal{O}(p^d)$    \\
      SF matrix free      & 0                         & $\mathcal{O}(p^{d+1})$ & $\mathcal{O}(p^d)$    \\
    \end{tabular}
  \end{itemize}
\end{frame}

\begin{frame}[fragile]
  \frametitle{Not just for $Q$ and $dQ$}
  \begin{columns}
    \begin{column}{0.5\textwidth}
      \begin{block}{}
        Find $u \in V \subset H(curl)$ s.t.
        {\scriptsize \begin{equation*}
          \int\!\!\curl u \cdot \curl v \,\text{d}x = \int\!\!B\cdot
          v\,\text{d}x \quad \forall v \in V.
        \end{equation*}}
\begin{minted}[fontsize=\tiny,mathescape]{python}
NCE = FiniteElement("NCE", hexahedron, degree)
Q = VectorElement("Q", hexahedron, degree)
u = Coefficient(NCE) # Solution variable
B = Coefficient(Q)   # Coefficient in $H^1$
v = TestFunction(NCE)
F = (dot(curl(u), curl(v)) - dot(B, v))*dx
\end{minted}
    \end{block}
    {\scriptsize
      \begin{uncoverenv}<2>
        \begin{itemize}
        \item TSFC obtains optimal \emph{complexity} evaluation
        \item In progress: is the constant factor good?
        \item Much still to be done in terms of vectorisation.
        \end{itemize}
      \end{uncoverenv}
      }
    \end{column}
  \begin{column}{0.6\textwidth}
  \begin{center}
    \begin{tikzpicture}[scale=0.9]
      \begin{loglogaxis}[name=plot, small, title=FLOPs for single-cell residual,
        xlabel=Polynomial degree,
        ylabel=FLOPs, xtick={1,2,4,8,16,32},
        xticklabels={$1$,$2$,$4$,$8$,$16$,$32$}, axis lines=left, axis
        line style={-}, log basis x=2,
        legend entries={Na\"ive (no sum fact), With sum factorisation},
        legend style={at={(0.5,-0.3)},anchor=north,draw=none}]
        \pgfplotstableread[row sep=crcr]{
degree flops_vanilla flops_opt\\
1 9.78100e+03 6.48100e+03\\
2 1.05624e+05 2.78280e+04\\
3 6.04110e+05 7.50860e+04\\
4 2.36102e+06 1.63395e+05\\
5 7.20724e+06 3.09060e+05\\
6 1.85105e+07 5.33483e+05\\
7 4.18644e+07 8.60322e+05\\
8 8.59101e+07 1.31626e+06\\
9 1.63287e+08 1.93100e+06\\
10 2.91712e+08 2.73728e+06\\
11 4.95190e+08 3.77084e+06\\
12 8.05352e+08 5.07047e+06\\
13 1.26293e+09 6.67796e+06\\
14 1.91933e+09 8.63816e+06\\
15 2.83841e+09 1.09989e+07\\
16 4.09829e+09 1.38110e+07\\
17 5.79335e+09 1.71285e+07\\
18 8.03637e+09 2.10082e+07\\
19 1.09608e+10 2.55101e+07\\
20 1.47229e+10 3.06972e+07\\
21 1.95048e+10 3.66353e+07\\
22 2.55164e+10 4.33937e+07\\
23 3.29987e+10 5.10443e+07\\
24 4.22266e+10 5.96621e+07\\
25 5.35115e+10 6.93253e+07\\
26 6.72053e+10 8.01150e+07\\
27 8.37026e+10 9.21153e+07\\
28 1.03445e+11 1.05413e+08\\
29 1.26924e+11 1.20099e+08\\
30 1.54686e+11 1.36267e+08\\
31 1.87334e+11 1.54011e+08\\
32 2.25533e+11 1.73433e+08\\
}\data; \pgfplotstableset{create on use/vanilla/.style={create
            col/expr={1e3*pow(\thisrow{degree},6)}}};
        \pgfplotstableset{create on use/spectral/.style={create
            col/expr={5e2*pow(\thisrow{degree},4)}}};
  
        \addplot+[mark=none, color=black, line width=1.5pt] table
        [x=degree,y=flops_vanilla] \data; \addplot+[mark=none,
        color=black, dashed, line width=1.5pt] table
        [x=degree,y=flops_opt] \data; \addplot+[mark=none, color=black,
        dotted, line width=1pt] table [x=degree,y=vanilla] \data
        coordinate [pos=0.67] (A); \node at (A) [anchor=south east]
        {$\mathcal{O}(p^6)$}; \addplot+[mark=none, color=black,
        dotted, line width=1pt] table [x=degree,y=spectral] \data
        coordinate [pos=0.67] (B); \node at (B) [anchor=north west]
        {$\mathcal{O}(p^4)$};
      \end{loglogaxis}
    \end{tikzpicture}
  \end{center}    
  \end{column}
  \end{columns}
\end{frame}

\begin{frame}
  \frametitle{Conclusions}
  \begin{itemize}
  \item Firedrake provides a layered set of abstractions for finite
    element computations.
  \item By capturing mathematical structure in code, we can
    \emph{automate} many transformations that people do by hand.
  \item Enables automated provision of ``HPC expertise'' to model
    developers.
  \item Good for experimentation from laptop to supercomputer.
  \end{itemize}
  \begin{block}{Future developments}
    \begin{itemize}
    \item Better support for subdomains and domain-decomposition PCs
      \begin{flushright}
        {\scriptsize
          Extending ideas from Kirby and \textbf{LM}. \arxivlink{1706.01346}{cs.MS}\nocite{Kirby:2017}}
      \end{flushright}
    \item Code generation for wide vector lanes
    \item ...
    \end{itemize}
  \end{block}
  \begin{tikzpicture}[remember picture,overlay]
    \node[at=(current page.south east), anchor=south east] {\includegraphics[width=2.5cm]{epsrc-logo}};
  \end{tikzpicture}
\end{frame}

\appendix
\begin{frame}
  \frametitle{References}
  \printbibliography[heading=none]
\end{frame}
\end{document}
