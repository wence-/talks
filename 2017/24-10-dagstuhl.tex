\documentclass[presentation]{beamer}
\usetheme{metropolis}
\metroset{progressbar=frametitle}
\usepackage{tikz}

\date{24th October 2017}
\title{If you've scheduled loops, you've gone too far}
\author{Lawrence Mitchell\inst{1,*}}
\institute{
\inst{1}Department\textbf{s} of Computing and Mathematics, Imperial College
London

\inst{*}\texttt{lawrence.mitchell@imperial.ac.uk}
}
\usepackage{pifont}
\usepackage{minted}
\newcommand{\cmark}{\ding{51}}
\newcommand{\xmark}{\ding{55}}
\newcommand{\arxivlink}[2]{%
  \href{http://www.arxiv.org/abs/#1}%
  {\texttt{arXiv:\,#1\,[#2]}}%
}
\newcommand{\doilink}[1]{%
  \href{http://dx.doi.org/#1}%
  {\texttt{doi:\,#1}}%
}
\usepackage[url=false,
            doi=true,
            isbn=false,
            style=authoryear,
            maxnames=10,
            firstinits=true,
            uniquename=init,
            backend=biber]{biblatex}

\setbeamertemplate{bibliography item}{}
\renewcommand{\bibfont}{\fontsize{8}{8}\selectfont}
\addbibresource{references.bib}
\setlength{\bibitemsep}{1ex}
\setlength{\fboxsep}{1pt}

\renewbibmacro{in:}{}
\DeclareFieldFormat[article]{volume}{\textbf{#1}}
\DeclareFieldFormat{doi}{%
  doi\addcolon%
  \ifhyperref{\href{http://dx.doi.org/#1}{\nolinkurl{#1}}}
  {\nolinkurl{#1}}}
\AtEveryBibitem{%
\clearfield{pages}%
\clearfield{issue}%
\clearfield{number}%
}

\DeclareMathOperator{\tr}{tr}

\graphicspath{{./\jobname.figures/}{../pictures/}}

\begin{document}

\begin{frame}[plain,noframenumbering]
  \maketitle
  \begin{tikzpicture}[remember picture,overlay]
    \node[at=(current page.south west), anchor=south west] {\includegraphics[height=0.9cm]{epsrc-logo}};
    \node[at=(current page.south east), anchor=south east] {\includegraphics[height=0.9cm]{imperial-two-tone}};
  \end{tikzpicture}
\end{frame}

% \begin{abstract}
%   If you've scheduled loops, you've gone too far
%   ==============================================

%   The optimal loop schedule for a given algorithm is typically
%   hardware dependent, even with all other parameters of the algorithm
%   fixed.  When we manually port code to a new hardware platform, we
%   must understand the loop structure and then perform, in tandem, data
%   layout and loop reordering to achieve good performance.  This is a
%   difficult task.  I argue that requiring a compiler system to perform
%   the same task will never work: the scheduled loop nest does not
%   offer enough information to the compiler for it to determine the
%   algorithmic structure.  Instead, we should strive for program
%   transformation steps that operate on /unscheduled/ DAGs.  This is
%   most easily achieved with DSLs, since no analysis is required.  I
%   will say some things about how we achieve this in the context of
%   finite element codes, but will mostly be full of questions.

% \end{abstract}

\begin{frame}[standout]
  Write data\uncover<2>{/task} parallel code

  {\small as any fule kno}
\end{frame}

\begin{frame}
  \frametitle{I've got 99 problems}
  \begin{lemma}
    You can't trust computational scientists to write good code.
  \end{lemma}

  \begin{corollary}
    Make it ``impossible'' to \emph{not} write good code.
  \end{corollary}
\end{frame}

\begin{frame}[fragile]
  \frametitle{DSLs for finite elements}
  \begin{columns}
    \begin{column}{0.47\framewidth}
      {\footnotesize
        Find $(u, p, T) \in V\times W\times Q$ s.t.
        \begin{align*}
          \int\!\nabla u \cdot \nabla v + (u \cdot \nabla u) \cdot v \\
          - p\nabla\cdot v + \frac{\text{Ra}}{\text{Pr}} Tg \hat{z} \cdot v\,\text{d}x &= 0 \\
          \int\!\nabla\cdot u q\,\text{d}x&= 0\\
          \int\! (u\cdot \nabla T) S + \text{Pr}^{-1} \nabla T \cdot \nabla
          S\,\text{d}x &= 0\\
          \quad \forall\, (v,q,T) \in V\times W \times Q
        \end{align*}
        }
    \end{column}
    \begin{uncoverenv}<2->
      \begin{column}{0.52\framewidth}
\begin{minted}[fontsize=\tiny]{python}
from firedrake import *
mesh = Mesh(...)
V = VectorFunctionSpace(mesh, "CG", 2)
W = FunctionSpace(mesh, "CG", 1)
Q = FunctionSpace(mesh, "CG", 1)
Z = V * W * Q
Ra = Constant(200)
Pr = Constant(6.18)
upT = Function(Z)
u, p, T = split(upT)
v, q, S = TestFunctions(Z)
bcs = [...] # no-flow + temp gradient
nullspace = MixedVectorSpaceBasis(
   Z, [Z.sub(0), VectorSpaceBasis(constant=True),
       Z.sub(2)])
F = (inner(grad(u), grad(v))
     + inner(dot(grad(u), u), v)
     - inner(p, div(v))
     + (Ra/Pr)*inner(T*g, v)
     + inner(div(u), q)
     + inner(dot(grad(T), u), S)
     + (1/Pr) * inner(grad(T), grad(S)))*dx

solve(F == 0, upT, bcs=bcs, nullspace=nullspace)
\end{minted}
      \end{column}
    \end{uncoverenv}
  \end{columns}
\end{frame}
\begin{frame}[fragile]
  \frametitle{ + a DSL for solver configuration}
  \begin{columns}
    \begin{column}{0.5\textwidth}
      {\fontsize{4.5}{4.5}\selectfont
\begin{verbatim}
-snes_type newtonls
-snes_rtol 1e-8
-snes_linesearch_type basic
-ksp_type fgmres
-ksp_gmres_modifiedgramschmidt
-mat_type matfree
-pc_type fieldsplit
-pc_fieldsplit_type multiplicative
-pc_fieldsplit_0_fields 0,1
-pc_fieldsplit_1_fields 2
-prefix_push fieldsplit_1_
  -ksp_type gmres
  -ksp_rtol 1e-4,
  -pc_type python
  -pc_python_type firedrake.AssembledPC
  -assembled_mat_type aij
  -assembled_pc_type telescope
  -assembled_pc_telescope_reduction_factor 6
  -assembled_telescope_pc_type hypre
  -assembled_telescope_pc_hypre_boomeramg_P_max 4
  -assembled_telescope_pc_hypre_boomeramg_agg_nl 1
  -assembled_telescope_pc_hypre_boomeramg_agg_num_paths 2
  -assembled_telescope_pc_hypre_boomeramg_coarsen_type HMIS
  -assembled_telescope_pc_hypre_boomeramg_interp_type ext+i
  -assembled_telescope_pc_hypre_boomeramg_no_CF True
-prefix_pop
-prefix_push fieldsplit_0_
  -ksp_type gmres
  -ksp_gmres_modifiedgramschmidt
  -ksp_rtol 1e-2
  -pc_type fieldsplit
  -pc_fieldsplit_type schur
  -pc_fieldsplit_schur_fact_type lower
\end{verbatim}
        }
      \end{column}
      \begin{column}{0.5\textwidth}
        {\fontsize{4.5}{4.5}\selectfont
\begin{verbatim}
  -prefix_push fieldsplit_0_
    -ksp_type preonly
    -pc_type python
    -pc_python_type firedrake.AssembledPC
    -assembled_mat_type aij
    -assembled_pc_type hypre
    -assembled_pc_hypre_boomeramg_P_max 4
    -assembled_pc_hypre_boomeramg_agg_nl 1
    -assembled_pc_hypre_boomeramg_agg_num_paths 2
    -assembled_pc_hypre_boomeramg_coarsen_type HMIS
    -assembled_pc_hypre_boomeramg_interp_type ext+i
    -assembled_pc_hypre_boomeramg_no_CF
  -prefix_pop
  -prefix_push fieldsplit_1_
    -ksp_type preonly
    -pc_type python
    -pc_python_type firedrake.PCDPC
    -pcd_Fp_mat_type matfree
    -pcd_Kp_ksp_type preonly
    -pcd_Kp_mat_type aij
    -pcd_Kp_pc_type telescope
    -pcd_Kp_pc_telescope_reduction_factor 6
    -pcd_Kp_telescope_pc_type ksp
    -pcd_Kp_telescope_ksp_ksp_max_it 3
    -pcd_Kp_telescope_ksp_ksp_type richardson
    -pcd_Kp_telescope_ksp_pc_type hypre
    -pcd_Kp_telescope_ksp_pc_hypre_boomeramg_P_max 4
    -pcd_Kp_telescope_ksp_pc_hypre_boomeramg_agg_nl 1
    -pcd_Kp_telescope_ksp_pc_hypre_boomeramg_agg_num_paths 2
    -pcd_Kp_telescope_ksp_pc_hypre_boomeramg_coarsen_type HMIS
    -pcd_Kp_telescope_ksp_pc_hypre_boomeramg_interp_type ext+i
    -pcd_Kp_telescope_ksp_pc_hypre_boomeramg_no_CF
    -pcd_Mp_mat_type aij
    -pcd_Mp_ksp_type richardson
    -pcd_Mp_pc_type sor
    -pcd_Mp_ksp_max_it 2
  -prefix_pop
-prefix_pop
\end{verbatim}
}
      \end{column}
  \end{columns}
\end{frame}

\begin{frame}
  \frametitle{Firedrake \url{www.firedrakeproject.org}}

  \begin{quote}
    {\normalfont [\ldots]} an automated system for the solution of partial
    differential equations using the finite element method.
  \end{quote}

  \begin{itemize}
  \item Written in Python.
  \item Finite element problems specified with \emph{embedded} domain
    specific language, UFL \parencite{Alnaes:2014} from the FEniCS project.
  \item \emph{Runtime} compilation to low-level (C) code.
  \item Explicitly \emph{data parallel} API.
  \end{itemize}

  \begin{flushright}
    {\scriptsize F.~Rathgeber, D.A.~Ham, \textbf{LM}, M.~Lange,
      F.~Luporini, A.T.T.~McRae, G.-T.~Bercea, G.R.~Markall,
      P.H.J.~Kelly. TOMS,
      2016. \arxivlink{1501.01809}{cs.MS}\nocite{Rathgeber:2016}}
  \end{flushright}
\end{frame}


\begin{frame}
  \frametitle{Code transformation}
  \begin{itemize}
  \item Represent fields as expansion in some basis $\{\phi_i\}$ for
    the discrete space.
  \item Integrals are computed by numerical quadrature on mesh elements.
    \begin{equation*}
      \int_\Omega F(\phi_i) \phi_j \,\text{d}x \rightarrow \sum_{e \in
        \mathcal{T}} \sum_q w_q F(\phi_i(q)) \phi_j(q)
    \end{equation*}
    \item Need to evaluate $\phi_j$ and $F(\phi_i)$, defined by basis coefficients
      $\{f_i\}_{i=1}^{N}$, at quadrature points $\{q_j\}_{j=1}^{Q}$.

    \begin{equation*}
      \mathcal{F}_q = \big[\Phi f\,\big]_q = \sum_i \phi_{i,q} f_i
    \end{equation*}
  \item $\Phi$ is a $Q\times N$ matrix of basis functions evaluated at
    quadrature points.
  \end{itemize}
\end{frame}

\begin{frame}
  \frametitle{How fast can you do dgemv?}
  \begin{itemize}
  \item For degree $p$ elements in $d$ dimensions. $N, Q = \mathcal{O}(p^d)$.
  \item<2-> So I need $\mathcal{O}(p^{2d})$ operations.  Right?
  \item<3-> Well not always$\dots$.
  \end{itemize}
\end{frame}

\begin{frame}
  \frametitle{Exploiting structure}
  Often, $\phi$ might have a tensor decomposition.
  \begin{equation*}
    \phi_{i,q}(x,y,\dots) :=  \varphi_{j,p}(x)\varphi_{k,r}(y)\dots
  \end{equation*}
  and so
  \begin{align*}
    \mathcal{F}_{(p,r)} &= \sum_{j,k} \phi_{(j,k),(p,r)} f_{j,k} \\
                        &= \sum_{j,k} \varphi_{j,p}\varphi_{k,r} f_{j,k} \\
                        &= \sum_j \varphi_{j,p} \sum_k \varphi_{k,r} f_{j,k}
  \end{align*}
  at the cost of some temporary storage, this requires only
  $\mathcal{O}(d p^{d+1})$ operations.
\end{frame}

\begin{frame}[fragile]
  \frametitle{Tensions}
  \begin{itemize}
  \item You want the granularity of the data parallel operation to be
    small
  \item That way the programmer has less chance to get it wrong
  \item But, can you then get the ``good'' algorithm?
  \end{itemize}
  \hspace{0.5\baselineskip}
  \begin{uncoverenv}<2->
    \begin{columns}
      \begin{column}{0.5\textwidth}
        Will your compiler hoist into array temporaries?
\begin{minted}[fontsize=\scriptsize]{c}
for (p = 0; p < L; p++)
 for (r = 0; r < L; r++)
  for (j = 0; j < M; j++)
   for (k = 0; k < M; k++)
    F[L*p+r][j*M+k] += f(p,j)*g(r,k)
\end{minted}
    \end{column}
    \begin{column}{0.5\textwidth}
      Will your blas library notice if \texttt{PHI} has
      Kronecker-product structure?
\begin{minted}[fontsize=\scriptsize]{c}
double PHI[L][M] = {{ ... }};
dgemv(PHI, Fi, Fq)
\end{minted}
    \end{column}
  \end{columns}
\end{uncoverenv}
\end{frame}
\begin{frame}
  \frametitle{Scheduling}
  \begin{itemize}
  \item Front-end DSL matches finite elements
  \item Compiler frontend removes finite element specific constructs $\rightarrow$
    DAG representation of tensor-algebra.
  \item These operations can have structure (e.g.~tensor-product
    decomposition)
  \item Transformations on the DAG to minimise op-count, perhaps
    promote vectorisation.
  \item Scheduling $\leftrightarrow$ topological sort of DAG
  \item Opportunity to introduce hardware- and problem-guided
    heuristics, and optimisation passes
  \end{itemize}
  \begin{flushright}
    {\scriptsize Homolya, \textbf{LM}, Luporini,
      Ham. \arxivlink{1705.03667}{cs.MS}\nocite{Homolya:2017} \\

      Homolya, Kirby, Ham.  In preparation}
  \end{flushright}
\end{frame}

\begin{frame}
  \frametitle{DSLs are handcuffs}

  \begin{itemize}
  \item[\cmark] A DSL should elegantly capture mathematical structure
  \item[\cmark] things expressible in the mathematics can be compiled
    to efficient code \emph{and algorithms!}
  \item[\xmark] All else cannot be compiled, need graceful
    degradation.
  \item[\xmark/\cmark] Greatest advantages come when you incorporate
    them \emph{at the top level}.
  \end{itemize}
\end{frame}

\begin{frame}[standout]
  Thanks!
\end{frame}
\appendix
\begin{frame}
  \frametitle{References}
  \printbibliography[heading=none]
\end{frame}
\end{document}
